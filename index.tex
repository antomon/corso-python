% Options for packages loaded elsewhere
\PassOptionsToPackage{unicode}{hyperref}
\PassOptionsToPackage{hyphens}{url}
\PassOptionsToPackage{dvipsnames,svgnames,x11names}{xcolor}
%
\documentclass[
  letterpaper,
  DIV=11,
  numbers=noendperiod]{scrreprt}

\usepackage{amsmath,amssymb}
\usepackage{iftex}
\ifPDFTeX
  \usepackage[T1]{fontenc}
  \usepackage[utf8]{inputenc}
  \usepackage{textcomp} % provide euro and other symbols
\else % if luatex or xetex
  \usepackage{unicode-math}
  \defaultfontfeatures{Scale=MatchLowercase}
  \defaultfontfeatures[\rmfamily]{Ligatures=TeX,Scale=1}
\fi
\usepackage{lmodern}
\ifPDFTeX\else  
    % xetex/luatex font selection
\fi
% Use upquote if available, for straight quotes in verbatim environments
\IfFileExists{upquote.sty}{\usepackage{upquote}}{}
\IfFileExists{microtype.sty}{% use microtype if available
  \usepackage[]{microtype}
  \UseMicrotypeSet[protrusion]{basicmath} % disable protrusion for tt fonts
}{}
\makeatletter
\@ifundefined{KOMAClassName}{% if non-KOMA class
  \IfFileExists{parskip.sty}{%
    \usepackage{parskip}
  }{% else
    \setlength{\parindent}{0pt}
    \setlength{\parskip}{6pt plus 2pt minus 1pt}}
}{% if KOMA class
  \KOMAoptions{parskip=half}}
\makeatother
\usepackage{xcolor}
\setlength{\emergencystretch}{3em} % prevent overfull lines
\setcounter{secnumdepth}{5}
% Make \paragraph and \subparagraph free-standing
\makeatletter
\ifx\paragraph\undefined\else
  \let\oldparagraph\paragraph
  \renewcommand{\paragraph}{
    \@ifstar
      \xxxParagraphStar
      \xxxParagraphNoStar
  }
  \newcommand{\xxxParagraphStar}[1]{\oldparagraph*{#1}\mbox{}}
  \newcommand{\xxxParagraphNoStar}[1]{\oldparagraph{#1}\mbox{}}
\fi
\ifx\subparagraph\undefined\else
  \let\oldsubparagraph\subparagraph
  \renewcommand{\subparagraph}{
    \@ifstar
      \xxxSubParagraphStar
      \xxxSubParagraphNoStar
  }
  \newcommand{\xxxSubParagraphStar}[1]{\oldsubparagraph*{#1}\mbox{}}
  \newcommand{\xxxSubParagraphNoStar}[1]{\oldsubparagraph{#1}\mbox{}}
\fi
\makeatother


\providecommand{\tightlist}{%
  \setlength{\itemsep}{0pt}\setlength{\parskip}{0pt}}\usepackage{longtable,booktabs,array}
\usepackage{calc} % for calculating minipage widths
% Correct order of tables after \paragraph or \subparagraph
\usepackage{etoolbox}
\makeatletter
\patchcmd\longtable{\par}{\if@noskipsec\mbox{}\fi\par}{}{}
\makeatother
% Allow footnotes in longtable head/foot
\IfFileExists{footnotehyper.sty}{\usepackage{footnotehyper}}{\usepackage{footnote}}
\makesavenoteenv{longtable}
\usepackage{graphicx}
\makeatletter
\def\maxwidth{\ifdim\Gin@nat@width>\linewidth\linewidth\else\Gin@nat@width\fi}
\def\maxheight{\ifdim\Gin@nat@height>\textheight\textheight\else\Gin@nat@height\fi}
\makeatother
% Scale images if necessary, so that they will not overflow the page
% margins by default, and it is still possible to overwrite the defaults
% using explicit options in \includegraphics[width, height, ...]{}
\setkeys{Gin}{width=\maxwidth,height=\maxheight,keepaspectratio}
% Set default figure placement to htbp
\makeatletter
\def\fps@figure{htbp}
\makeatother
% definitions for citeproc citations
\NewDocumentCommand\citeproctext{}{}
\NewDocumentCommand\citeproc{mm}{%
  \begingroup\def\citeproctext{#2}\cite{#1}\endgroup}
\makeatletter
 % allow citations to break across lines
 \let\@cite@ofmt\@firstofone
 % avoid brackets around text for \cite:
 \def\@biblabel#1{}
 \def\@cite#1#2{{#1\if@tempswa , #2\fi}}
\makeatother
\newlength{\cslhangindent}
\setlength{\cslhangindent}{1.5em}
\newlength{\csllabelwidth}
\setlength{\csllabelwidth}{3em}
\newenvironment{CSLReferences}[2] % #1 hanging-indent, #2 entry-spacing
 {\begin{list}{}{%
  \setlength{\itemindent}{0pt}
  \setlength{\leftmargin}{0pt}
  \setlength{\parsep}{0pt}
  % turn on hanging indent if param 1 is 1
  \ifodd #1
   \setlength{\leftmargin}{\cslhangindent}
   \setlength{\itemindent}{-1\cslhangindent}
  \fi
  % set entry spacing
  \setlength{\itemsep}{#2\baselineskip}}}
 {\end{list}}
\usepackage{calc}
\newcommand{\CSLBlock}[1]{\hfill\break\parbox[t]{\linewidth}{\strut\ignorespaces#1\strut}}
\newcommand{\CSLLeftMargin}[1]{\parbox[t]{\csllabelwidth}{\strut#1\strut}}
\newcommand{\CSLRightInline}[1]{\parbox[t]{\linewidth - \csllabelwidth}{\strut#1\strut}}
\newcommand{\CSLIndent}[1]{\hspace{\cslhangindent}#1}

\KOMAoption{captions}{tableheading}
\makeatletter
\@ifpackageloaded{bookmark}{}{\usepackage{bookmark}}
\makeatother
\makeatletter
\@ifpackageloaded{caption}{}{\usepackage{caption}}
\AtBeginDocument{%
\ifdefined\contentsname
  \renewcommand*\contentsname{Table of contents}
\else
  \newcommand\contentsname{Table of contents}
\fi
\ifdefined\listfigurename
  \renewcommand*\listfigurename{List of Figures}
\else
  \newcommand\listfigurename{List of Figures}
\fi
\ifdefined\listtablename
  \renewcommand*\listtablename{List of Tables}
\else
  \newcommand\listtablename{List of Tables}
\fi
\ifdefined\figurename
  \renewcommand*\figurename{Figure}
\else
  \newcommand\figurename{Figure}
\fi
\ifdefined\tablename
  \renewcommand*\tablename{Table}
\else
  \newcommand\tablename{Table}
\fi
}
\@ifpackageloaded{float}{}{\usepackage{float}}
\floatstyle{ruled}
\@ifundefined{c@chapter}{\newfloat{codelisting}{h}{lop}}{\newfloat{codelisting}{h}{lop}[chapter]}
\floatname{codelisting}{Listing}
\newcommand*\listoflistings{\listof{codelisting}{List of Listings}}
\makeatother
\makeatletter
\makeatother
\makeatletter
\@ifpackageloaded{caption}{}{\usepackage{caption}}
\@ifpackageloaded{subcaption}{}{\usepackage{subcaption}}
\makeatother
\ifLuaTeX
  \usepackage{selnolig}  % disable illegal ligatures
\fi
\usepackage{bookmark}

\IfFileExists{xurl.sty}{\usepackage{xurl}}{} % add URL line breaks if available
\urlstyle{same} % disable monospaced font for URLs
\hypersetup{
  pdftitle={Corso Python},
  pdfauthor={Antonio Montano},
  colorlinks=true,
  linkcolor={blue},
  filecolor={Maroon},
  citecolor={Blue},
  urlcolor={Blue},
  pdfcreator={LaTeX via pandoc}}

\title{Corso Python}
\author{Antonio Montano}
\date{2024-05-24}

\begin{document}
\maketitle

\renewcommand*\contentsname{Table of contents}
{
\hypersetup{linkcolor=}
\setcounter{tocdepth}{2}
\tableofcontents
}
\bookmarksetup{startatroot}

\chapter*{Preface}\label{preface}
\addcontentsline{toc}{chapter}{Preface}

\markboth{Preface}{Preface}

This is a Quarto book.

To learn more about Quarto books visit
\url{https://quarto.org/docs/books}.

\part{Prima parte: I fondamenti}

\chapter{I linguaggi di programmazione, i programmi e i
programmatori}\label{i-linguaggi-di-programmazione-i-programmi-e-i-programmatori}

Partiamo da alcuni concetti basilari a cui collegare quelli che
approfondiremo nel corso.

\section{Cosa sono?}\label{cosa-sono}

La programmazione è il processo di progettazione e scrittura di
istruzioni che un computer può ricevere per eseguire compiti
predefiniti. Queste istruzioni sono codificate in un linguaggio di
programmazione, che traduce le idee e gli algoritmi del programmatore in
un formato che il computer può comprendere ed eseguire.

Cos'è un programma informatico?

Un programma informatico è una sequenza di istruzioni scritte per
eseguire una specifica operazione o un insieme di operazioni su un
computer. Queste istruzioni sono codificate in un linguaggio che il
computer può comprendere e seguire per eseguire attività come calcoli,
manipolazione di dati, controllo di dispositivi e interazione con
l'utente.

Pensate a un programma come a una ricetta di cucina. La ricetta elenca
gli ingredienti necessari (dati) e fornisce istruzioni passo-passo
(algoritmo) per preparare un piatto. Allo stesso modo, un programma
informatico specifica i dati da usare e le istruzioni da seguire per
ottenere un risultato desiderato.

Cos'è un linguaggio di programmazione?

Un linguaggio di programmazione è un linguaggio formale che fornisce un
insieme di regole e sintassi per scrivere programmi informatici. Questi
linguaggi permettono ai programmatori di comunicare con i computer e di
creare software. Alcuni esempi di linguaggi di programmazione includono
Python, Java, C++ e JavaScript.

I linguaggi di programmazione differiscono dai linguaggi naturali (come
l'italiano o l'inglese) in diversi modi:

\begin{enumerate}
\def\labelenumi{\arabic{enumi}.}
\tightlist
\item
  Precisione e rigidità: I linguaggi di programmazione sono estremamente
  precisi e rigidi. Ogni istruzione deve essere scritta in un modo
  specifico affinché il computer possa comprenderla ed eseguirla
  correttamente. Anche un piccolo errore di sintassi può impedire il
  funzionamento di un programma.
\item
  Ambiguità: I linguaggi naturali sono spesso ambigui e aperti a
  interpretazioni. Le stesse parole possono avere significati diversi a
  seconda del contesto. I linguaggi di programmazione, invece, sono
  progettati per essere privi di ambiguità; ogni istruzione ha un
  significato preciso e univoco.
\item
  Vocabolario limitato: I linguaggi naturali hanno un vocabolario
  vastissimo e in continua espansione. I linguaggi di programmazione, al
  contrario, hanno un vocabolario limitato costituito da parole chiave e
  comandi definiti dal linguaggio stesso.
\end{enumerate}

Come un programma produce azioni in un calcolatore?

Quando un programma è scritto e salvato, il computer deve eseguirlo per
produrre le azioni desiderate. Questo processo avviene in diverse fasi:

\begin{enumerate}
\def\labelenumi{\arabic{enumi}.}
\tightlist
\item
  Compilazione o interpretazione: La maggior parte dei programmi deve
  essere trasformata da un linguaggio di alto livello (leggibile
  dall'uomo) a un linguaggio macchina (comprensibile dal computer).
  Questo avviene attraverso un processo chiamato compilazione (per
  linguaggi come C++ o Java) o interpretazione (per linguaggi come
  Python o JavaScript).
\item
  Esecuzione: Una volta che il programma è stato compilato o
  interpretato, il computer può eseguire le istruzioni una per una. La
  CPU (central processing unit) del computer legge le istruzioni e le
  esegue, manipolando i dati e producendo i risultati desiderati.
\item
  Interazione con componenti hardware: Durante l'esecuzione, il
  programma può interagire con vari componenti hardware del computer,
  come la memoria, i dischi rigidi, la rete, e i dispositivi di
  input/output (come tastiere e monitor).
\end{enumerate}

\section{L'Impatto dell'intelligenza artificiale generativa sulla
programmazione}\label{limpatto-dellintelligenza-artificiale-generativa-sulla-programmazione}

Con l'avvento dell'intelligenza artificiale (IA) generativa, la
programmazione ha subito una trasformazione significativa. Prima dell'IA
generativa, i programmatori dovevano tutti scrivere manualmente ogni
riga di codice, seguendo rigorosamente la sintassi e le regole del
linguaggio di programmazione scelto. Questo processo richiedeva una
conoscenza approfondita degli algoritmi, delle strutture dati e delle
migliori pratiche di programmazione.

Inoltre, i programmatori dovevano creare ogni funzione, classe e modulo
a mano, assicurandosi che ogni dettaglio fosse corretto, identificavano
e correggevano gli errori nel codice con un processo lungo e laborioso,
che comportava anche la scrittura di casi di test e l'esecuzione di
sessioni di esecuzione di tali casi. Infine, dovebano scrivere
documentazione dettagliata per spiegare il funzionamento del codice e
facilitare la manutenzione futura.

\subsection{Attività del programmatore con l'IA
Generativa}\label{attivituxe0-del-programmatore-con-lia-generativa}

L'IA generativa ha introdotto nuovi strumenti e metodologie che stanno
cambiando il modo in cui i programmatori lavorano:

\begin{enumerate}
\def\labelenumi{\arabic{enumi}.}
\tightlist
\item
  Generazione automatica del codice: Gli strumenti di IA generativa
  possono creare porzioni di codice basate su descrizioni ad alto
  livello fornite dai programmatori. Questo permette di velocizzare
  notevolmente lo sviluppo iniziale e ridurre gli errori di sintassi.
\item
  Assistenza nel debugging: L'IA può identificare potenziali bug e
  suggerire correzioni, rendendo il processo di debugging più efficiente
  e meno dispendioso in termini di tempo.
\item
  Ottimizzazione automatica: Gli algoritmi di IA possono analizzare il
  codice e suggerire o applicare automaticamente ottimizzazioni per
  migliorare le prestazioni.
\item
  Generazione di casi di test: L'IA può creare casi di test per
  verificare la correttezza del codice, coprendo una gamma più ampia di
  scenari di quanto un programmatore potrebbe fare manualmente.
\item
  Documentazione automatica: L'IA può generare documentazione leggendo e
  interpretando il codice, riducendo il carico di lavoro manuale e
  garantendo una documentazione coerente e aggiornata.
\end{enumerate}

\subsection{L'Importanza di imparare a programmare nell'era dell'IA
generativa}\label{limportanza-di-imparare-a-programmare-nellera-dellia-generativa}

Nonostante l'avvento dell'IA generativa, imparare a programmare rimane
fondamentale per diverse ragioni. La programmazione non è solo una
competenza tecnica, ma anche un modo di pensare e risolvere problemi.
Comprendere i fondamenti della programmazione è essenziale per
utilizzare efficacemente gli strumenti di IA generativa. Senza una
solida base, è difficile sfruttare appieno queste tecnologie. Inoltre,
la programmazione insegna a scomporre problemi complessi in parti più
gestibili e a trovare soluzioni logiche e sequenziali, una competenza
preziosa in molti campi.

Anche con l'IA generativa, esisteranno sempre situazioni in cui sarà
necessario personalizzare o ottimizzare il codice per esigenze
specifiche. La conoscenza della programmazione permette di fare queste
modifiche con sicurezza. Inoltre, quando qualcosa va storto, è
indispensabile sapere come leggere e comprendere il codice per
identificare e risolvere i problemi. L'IA può assistere, ma la
comprensione umana rimane cruciale per interventi mirati.

Imparare a programmare consente di sperimentare nuove idee e prototipare
rapidamente soluzioni innovative. La creatività è potenziata dalla
capacità di tradurre idee in codice funzionante. Sapere programmare
aiuta anche a comprendere i limiti e le potenzialità degli strumenti di
IA generativa, permettendo di usarli in modo più strategico ed efficace.

La tecnologia evolve rapidamente, e con una conoscenza della
programmazione si è meglio preparati ad adattarsi alle nuove tecnologie
e metodologie che emergeranno in futuro. Inoltre, la programmazione è
una competenza trasversale applicabile in numerosi settori, dalla
biologia computazionale alla finanza, dall'ingegneria all'arte digitale.
Avere questa competenza amplia notevolmente le opportunità di carriera.

Infine, la programmazione è una porta d'accesso a ruoli più avanzati e
specializzati nel campo della tecnologia, come l'ingegneria del
software, la scienza dei dati e la ricerca sull'IA. Conoscere i principi
della programmazione aiuta a comprendere meglio come funzionano gli
algoritmi di IA, permettendo di contribuire attivamente allo sviluppo di
nuove tecnologie.

\cleardoublepage
\phantomsection
\addcontentsline{toc}{part}{Appendices}
\appendix

\chapter*{References}\label{references}
\addcontentsline{toc}{chapter}{References}

\markboth{References}{References}

\phantomsection\label{refs}
\begin{CSLReferences}{0}{1}
\end{CSLReferences}



\end{document}
