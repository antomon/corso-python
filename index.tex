% Options for packages loaded elsewhere
\PassOptionsToPackage{unicode}{hyperref}
\PassOptionsToPackage{hyphens}{url}
%
\documentclass[
  letterpaper,
]{scrbook}

\usepackage{amsmath,amssymb}
\usepackage{iftex}
\ifPDFTeX
  \usepackage[T1]{fontenc}
  \usepackage[utf8]{inputenc}
  \usepackage{textcomp} % provide euro and other symbols
\else % if luatex or xetex
  \usepackage{unicode-math}
  \defaultfontfeatures{Scale=MatchLowercase}
  \defaultfontfeatures[\rmfamily]{Ligatures=TeX,Scale=1}
\fi
\usepackage{lmodern}
\ifPDFTeX\else  
    % xetex/luatex font selection
\fi
% Use upquote if available, for straight quotes in verbatim environments
\IfFileExists{upquote.sty}{\usepackage{upquote}}{}
\IfFileExists{microtype.sty}{% use microtype if available
  \usepackage[]{microtype}
  \UseMicrotypeSet[protrusion]{basicmath} % disable protrusion for tt fonts
}{}
\makeatletter
\@ifundefined{KOMAClassName}{% if non-KOMA class
  \IfFileExists{parskip.sty}{%
    \usepackage{parskip}
  }{% else
    \setlength{\parindent}{0pt}
    \setlength{\parskip}{6pt plus 2pt minus 1pt}}
}{% if KOMA class
  \KOMAoptions{parskip=half}}
\makeatother
\usepackage{xcolor}
\usepackage[left=3cm,right=3cm,top=2cm,bottom=2cm]{geometry}
\setlength{\emergencystretch}{3em} % prevent overfull lines
\setcounter{secnumdepth}{5}
% Make \paragraph and \subparagraph free-standing
\makeatletter
\ifx\paragraph\undefined\else
  \let\oldparagraph\paragraph
  \renewcommand{\paragraph}{
    \@ifstar
      \xxxParagraphStar
      \xxxParagraphNoStar
  }
  \newcommand{\xxxParagraphStar}[1]{\oldparagraph*{#1}\mbox{}}
  \newcommand{\xxxParagraphNoStar}[1]{\oldparagraph{#1}\mbox{}}
\fi
\ifx\subparagraph\undefined\else
  \let\oldsubparagraph\subparagraph
  \renewcommand{\subparagraph}{
    \@ifstar
      \xxxSubParagraphStar
      \xxxSubParagraphNoStar
  }
  \newcommand{\xxxSubParagraphStar}[1]{\oldsubparagraph*{#1}\mbox{}}
  \newcommand{\xxxSubParagraphNoStar}[1]{\oldsubparagraph{#1}\mbox{}}
\fi
\makeatother

\usepackage{color}
\usepackage{fancyvrb}
\newcommand{\VerbBar}{|}
\newcommand{\VERB}{\Verb[commandchars=\\\{\}]}
\DefineVerbatimEnvironment{Highlighting}{Verbatim}{commandchars=\\\{\}}
% Add ',fontsize=\small' for more characters per line
\usepackage{framed}
\definecolor{shadecolor}{RGB}{241,243,245}
\newenvironment{Shaded}{\begin{snugshade}}{\end{snugshade}}
\newcommand{\AlertTok}[1]{\textcolor[rgb]{0.68,0.00,0.00}{#1}}
\newcommand{\AnnotationTok}[1]{\textcolor[rgb]{0.37,0.37,0.37}{#1}}
\newcommand{\AttributeTok}[1]{\textcolor[rgb]{0.40,0.45,0.13}{#1}}
\newcommand{\BaseNTok}[1]{\textcolor[rgb]{0.68,0.00,0.00}{#1}}
\newcommand{\BuiltInTok}[1]{\textcolor[rgb]{0.00,0.23,0.31}{#1}}
\newcommand{\CharTok}[1]{\textcolor[rgb]{0.13,0.47,0.30}{#1}}
\newcommand{\CommentTok}[1]{\textcolor[rgb]{0.37,0.37,0.37}{#1}}
\newcommand{\CommentVarTok}[1]{\textcolor[rgb]{0.37,0.37,0.37}{\textit{#1}}}
\newcommand{\ConstantTok}[1]{\textcolor[rgb]{0.56,0.35,0.01}{#1}}
\newcommand{\ControlFlowTok}[1]{\textcolor[rgb]{0.00,0.23,0.31}{\textbf{#1}}}
\newcommand{\DataTypeTok}[1]{\textcolor[rgb]{0.68,0.00,0.00}{#1}}
\newcommand{\DecValTok}[1]{\textcolor[rgb]{0.68,0.00,0.00}{#1}}
\newcommand{\DocumentationTok}[1]{\textcolor[rgb]{0.37,0.37,0.37}{\textit{#1}}}
\newcommand{\ErrorTok}[1]{\textcolor[rgb]{0.68,0.00,0.00}{#1}}
\newcommand{\ExtensionTok}[1]{\textcolor[rgb]{0.00,0.23,0.31}{#1}}
\newcommand{\FloatTok}[1]{\textcolor[rgb]{0.68,0.00,0.00}{#1}}
\newcommand{\FunctionTok}[1]{\textcolor[rgb]{0.28,0.35,0.67}{#1}}
\newcommand{\ImportTok}[1]{\textcolor[rgb]{0.00,0.46,0.62}{#1}}
\newcommand{\InformationTok}[1]{\textcolor[rgb]{0.37,0.37,0.37}{#1}}
\newcommand{\KeywordTok}[1]{\textcolor[rgb]{0.00,0.23,0.31}{\textbf{#1}}}
\newcommand{\NormalTok}[1]{\textcolor[rgb]{0.00,0.23,0.31}{#1}}
\newcommand{\OperatorTok}[1]{\textcolor[rgb]{0.37,0.37,0.37}{#1}}
\newcommand{\OtherTok}[1]{\textcolor[rgb]{0.00,0.23,0.31}{#1}}
\newcommand{\PreprocessorTok}[1]{\textcolor[rgb]{0.68,0.00,0.00}{#1}}
\newcommand{\RegionMarkerTok}[1]{\textcolor[rgb]{0.00,0.23,0.31}{#1}}
\newcommand{\SpecialCharTok}[1]{\textcolor[rgb]{0.37,0.37,0.37}{#1}}
\newcommand{\SpecialStringTok}[1]{\textcolor[rgb]{0.13,0.47,0.30}{#1}}
\newcommand{\StringTok}[1]{\textcolor[rgb]{0.13,0.47,0.30}{#1}}
\newcommand{\VariableTok}[1]{\textcolor[rgb]{0.07,0.07,0.07}{#1}}
\newcommand{\VerbatimStringTok}[1]{\textcolor[rgb]{0.13,0.47,0.30}{#1}}
\newcommand{\WarningTok}[1]{\textcolor[rgb]{0.37,0.37,0.37}{\textit{#1}}}

\providecommand{\tightlist}{%
  \setlength{\itemsep}{0pt}\setlength{\parskip}{0pt}}\usepackage{longtable,booktabs,array}
\usepackage{calc} % for calculating minipage widths
% Correct order of tables after \paragraph or \subparagraph
\usepackage{etoolbox}
\makeatletter
\patchcmd\longtable{\par}{\if@noskipsec\mbox{}\fi\par}{}{}
\makeatother
% Allow footnotes in longtable head/foot
\IfFileExists{footnotehyper.sty}{\usepackage{footnotehyper}}{\usepackage{footnote}}
\makesavenoteenv{longtable}
\usepackage{graphicx}
\makeatletter
\def\maxwidth{\ifdim\Gin@nat@width>\linewidth\linewidth\else\Gin@nat@width\fi}
\def\maxheight{\ifdim\Gin@nat@height>\textheight\textheight\else\Gin@nat@height\fi}
\makeatother
% Scale images if necessary, so that they will not overflow the page
% margins by default, and it is still possible to overwrite the defaults
% using explicit options in \includegraphics[width, height, ...]{}
\setkeys{Gin}{width=\maxwidth,height=\maxheight,keepaspectratio}
% Set default figure placement to htbp
\makeatletter
\def\fps@figure{htbp}
\makeatother
% definitions for citeproc citations
\NewDocumentCommand\citeproctext{}{}
\NewDocumentCommand\citeproc{mm}{%
  \begingroup\def\citeproctext{#2}\cite{#1}\endgroup}
\makeatletter
 % allow citations to break across lines
 \let\@cite@ofmt\@firstofone
 % avoid brackets around text for \cite:
 \def\@biblabel#1{}
 \def\@cite#1#2{{#1\if@tempswa , #2\fi}}
\makeatother
\newlength{\cslhangindent}
\setlength{\cslhangindent}{1.5em}
\newlength{\csllabelwidth}
\setlength{\csllabelwidth}{3em}
\newenvironment{CSLReferences}[2] % #1 hanging-indent, #2 entry-spacing
 {\begin{list}{}{%
  \setlength{\itemindent}{0pt}
  \setlength{\leftmargin}{0pt}
  \setlength{\parsep}{0pt}
  % turn on hanging indent if param 1 is 1
  \ifodd #1
   \setlength{\leftmargin}{\cslhangindent}
   \setlength{\itemindent}{-1\cslhangindent}
  \fi
  % set entry spacing
  \setlength{\itemsep}{#2\baselineskip}}}
 {\end{list}}
\usepackage{calc}
\newcommand{\CSLBlock}[1]{\hfill\break\parbox[t]{\linewidth}{\strut\ignorespaces#1\strut}}
\newcommand{\CSLLeftMargin}[1]{\parbox[t]{\csllabelwidth}{\strut#1\strut}}
\newcommand{\CSLRightInline}[1]{\parbox[t]{\linewidth - \csllabelwidth}{\strut#1\strut}}
\newcommand{\CSLIndent}[1]{\hspace{\cslhangindent}#1}

\makeatletter
\@ifpackageloaded{tcolorbox}{}{\usepackage[skins,breakable]{tcolorbox}}
\@ifpackageloaded{fontawesome5}{}{\usepackage{fontawesome5}}
\definecolor{quarto-callout-color}{HTML}{909090}
\definecolor{quarto-callout-note-color}{HTML}{0758E5}
\definecolor{quarto-callout-important-color}{HTML}{CC1914}
\definecolor{quarto-callout-warning-color}{HTML}{EB9113}
\definecolor{quarto-callout-tip-color}{HTML}{00A047}
\definecolor{quarto-callout-caution-color}{HTML}{FC5300}
\definecolor{quarto-callout-color-frame}{HTML}{acacac}
\definecolor{quarto-callout-note-color-frame}{HTML}{4582ec}
\definecolor{quarto-callout-important-color-frame}{HTML}{d9534f}
\definecolor{quarto-callout-warning-color-frame}{HTML}{f0ad4e}
\definecolor{quarto-callout-tip-color-frame}{HTML}{02b875}
\definecolor{quarto-callout-caution-color-frame}{HTML}{fd7e14}
\makeatother
\makeatletter
\@ifpackageloaded{bookmark}{}{\usepackage{bookmark}}
\makeatother
\makeatletter
\@ifpackageloaded{caption}{}{\usepackage{caption}}
\AtBeginDocument{%
\ifdefined\contentsname
  \renewcommand*\contentsname{Indice}
\else
  \newcommand\contentsname{Indice}
\fi
\ifdefined\listfigurename
  \renewcommand*\listfigurename{Elenco delle Figure}
\else
  \newcommand\listfigurename{Elenco delle Figure}
\fi
\ifdefined\listtablename
  \renewcommand*\listtablename{Elenco delle Tabelle}
\else
  \newcommand\listtablename{Elenco delle Tabelle}
\fi
\ifdefined\figurename
  \renewcommand*\figurename{Figura}
\else
  \newcommand\figurename{Figura}
\fi
\ifdefined\tablename
  \renewcommand*\tablename{Tabella}
\else
  \newcommand\tablename{Tabella}
\fi
}
\@ifpackageloaded{float}{}{\usepackage{float}}
\floatstyle{ruled}
\@ifundefined{c@chapter}{\newfloat{codelisting}{h}{lop}}{\newfloat{codelisting}{h}{lop}[chapter]}
\floatname{codelisting}{Lista}
\newcommand*\listoflistings{\listof{codelisting}{Elenco degli Elenchi}}
\makeatother
\makeatletter
\makeatother
\makeatletter
\@ifpackageloaded{caption}{}{\usepackage{caption}}
\@ifpackageloaded{subcaption}{}{\usepackage{subcaption}}
\makeatother
\makeatletter
\@ifpackageloaded{tikz}{}{\usepackage{tikz}}
\makeatother
        \newcommand*\circled[1]{\tikz[baseline=(char.base)]{
          \node[shape=circle,draw,inner sep=1pt] (char) {{\scriptsize#1}};}}  
                  
\ifLuaTeX
\usepackage[bidi=basic]{babel}
\else
\usepackage[bidi=default]{babel}
\fi
\babelprovide[main,import]{italian}
% get rid of language-specific shorthands (see #6817):
\let\LanguageShortHands\languageshorthands
\def\languageshorthands#1{}
\ifLuaTeX
  \usepackage{selnolig}  % disable illegal ligatures
\fi
\usepackage{bookmark}

\IfFileExists{xurl.sty}{\usepackage{xurl}}{} % add URL line breaks if available
\urlstyle{same} % disable monospaced font for URLs
\hypersetup{
  pdftitle={Da neofita di Python a campione},
  pdfauthor={Antonio Montano},
  pdflang={it},
  hidelinks,
  pdfcreator={LaTeX via pandoc}}

\title{Da neofita di Python a campione}
\author{Antonio Montano}
\date{2024-05-24}

\begin{document}
\frontmatter
\maketitle

\renewcommand*\contentsname{Indice}
{
\setcounter{tocdepth}{2}
\tableofcontents
}
\mainmatter
\bookmarksetup{startatroot}

\chapter*{Prefazione}\label{prefazione}
\addcontentsline{toc}{chapter}{Prefazione}

\markboth{Prefazione}{Prefazione}

\part{Prima parte: I fondamenti della programmazione}

\chapter{I linguaggi di programmazione, i programmi e i
programmatori}\label{i-linguaggi-di-programmazione-i-programmi-e-i-programmatori}

Partiamo da alcuni concetti basilari che ci permette di contestualizzare
quelli che intridurremo via via nel corso.

\section{Definizioni}\label{definizioni}

La \textbf{programmazione} è il processo di progettazione e scrittura di
\textbf{istruzioni}, nella forma statica identificate come
\textbf{codice sorgente}, che un computer può ricevere per eseguire
compiti predefiniti. Queste istruzioni sono codificate in un
\textbf{linguaggio di programmazione}, che traduce le idee e gli
algoritmi del programmatore, in un formato comprensibile ed eseguibile
dal computer.

Un \textbf{programma} informatico è una sequenza di istruzioni scritte
per eseguire una specifica operazione o un insieme di operazioni su un
computer. Queste istruzioni sono codificate in un linguaggio che il
computer può comprendere e seguire per eseguire attività come calcoli,
manipolazione di dati, controllo di dispositivi e interazione con
l'utente. Pensate a un programma come a una ricetta di cucina. La
ricetta elenca gli ingredienti necessari (dati) e fornisce istruzioni
passo-passo (algoritmo) per preparare un piatto. Allo stesso modo, un
programma informatico specifica i dati da usare e le istruzioni da
seguire per ottenere un risultato desiderato.

Un linguaggio di programmazione è un linguaggio formale che fornisce un
insieme di regole e sintassi per scrivere programmi informatici. Questi
linguaggi permettono ai programmatori di comunicare con i computer e di
creare software. Alcuni esempi di linguaggi di programmazione includono
Python, Java, C++, SQL, Rust, Haskell, Prolog, C, Assembly, Fortran,
JavaScript e altre centinaia (o forse migliaia).

\section{Linguaggi naturali e di
programmazione}\label{linguaggi-naturali-e-di-programmazione}

I linguaggi di programmazione differiscono dai linguaggi naturali (come
l'italiano o l'inglese) in diversi modi:

\begin{enumerate}
\def\labelenumi{\arabic{enumi}.}
\item
  Precisione e rigidità: I linguaggi di programmazione sono estremamente
  precisi e rigidi. Ogni istruzione deve essere scritta in un modo
  specifico affinché il computer possa comprenderla ed eseguirla
  correttamente. Anche un piccolo errore di sintassi può impedire il
  funzionamento di un programma.
\item
  Ambiguità: I linguaggi naturali sono spesso ambigui e aperti a
  interpretazioni. Le stesse parole possono avere significati diversi a
  seconda del contesto. I linguaggi di programmazione, invece, sono
  progettati per essere privi di ambiguità; ogni istruzione ha un
  significato preciso e univoco.
\item
  Vocabolario limitato: I linguaggi naturali hanno un vocabolario
  vastissimo e in continua espansione. I linguaggi di programmazione, al
  contrario, hanno un vocabolario limitato costituito da parole chiave e
  comandi definiti dal linguaggio stesso.
\end{enumerate}

\section{Interpreti e compilatori}\label{interpreti-e-compilatori}

Quando un programma viene scritto e salvato in un file di testo, il
computer deve eseguirlo per produrre le azioni desiderate. Questo
processo si svolge in diverse fasi:

\begin{itemize}
\item
  Compilazione o interpretazione: Il codice sorgente, scritto in un
  linguaggio di alto livello leggibile dall'uomo, deve essere
  trasformato in un linguaggio macchina comprensibile dal computer.
  Questo avviene attraverso due possibili processi:

  \begin{itemize}
  \tightlist
  \item
    Compilazione: In linguaggi come C++ o Java, un compilatore traduce
    tutto il codice sorgente in linguaggio macchina, creando un file
    eseguibile. Questo file può poi essere eseguito direttamente dalla
    CPU.
  \item
    Interpretazione: In linguaggi come Python o JavaScript, un
    interprete legge ed esegue il codice sorgente riga per riga,
    traducendolo in linguaggio macchina al momento dell'esecuzione.
  \end{itemize}
\item
  Esecuzione: Una volta che il programma è stato compilato (nel caso dei
  linguaggi compilati) o viene interpretato (nel caso dei linguaggi
  interpretati), il computer può iniziare ad eseguire le istruzioni. La
  CPU (central processing unit) legge queste istruzioni dal file
  eseguibile o dall'interprete e le esegue una per una. Durante questa
  fase, la CPU manipola i dati e produce i risultati desiderati.
\item
  Interazione con i componenti hardware: Durante l'esecuzione, il
  programma può interagire con vari componenti hardware del computer. Ad
  esempio, può leggere e scrivere dati nella memoria, accedere ai dischi
  rigidi per salvare o recuperare informazioni, comunicare attraverso la
  rete, e interagire con dispositivi di input/output come tastiere e
  monitor. Questa interazione permette al programma di eseguire compiti
  complessi e di fornire output all'utente.
\end{itemize}

\section{L'Impatto dell'intelligenza artificiale generativa sulla
programmazione}\label{limpatto-dellintelligenza-artificiale-generativa-sulla-programmazione}

Con l'avvento dell'\textbf{intelligenza artificiale generativa} (IA
generativa), la programmazione ha subito una trasformazione
significativa. Prima dell'IA generativa, i programmatori dovevano tutti
scrivere manualmente ogni riga di codice, seguendo rigorosamente la
sintassi e le regole del linguaggio di programmazione scelto. Questo
processo richiedeva una conoscenza approfondita degli algoritmi, delle
strutture dati e delle migliori pratiche di programmazione.

Inoltre, i programmatori dovevano creare ogni funzione, classe e modulo
a mano, assicurandosi che ogni dettaglio fosse corretto, identificavano
e correggevano gli errori nel codice con un processo lungo e laborioso,
che comportava anche la scrittura di casi di test e l'esecuzione di
sessioni di esecuzione di tali casi. Infine, dovebano scrivere
documentazione dettagliata per spiegare il funzionamento del codice e
facilitare la manutenzione futura.

\subsection{Attività del programmatore con l'IA
Generativa}\label{attivituxe0-del-programmatore-con-lia-generativa}

L'IA generativa ha introdotto nuovi strumenti e metodologie che stanno
cambiando il modo in cui i programmatori lavorano:

\begin{enumerate}
\def\labelenumi{\arabic{enumi}.}
\tightlist
\item
  Generazione automatica del codice: Gli strumenti di IA generativa
  possono creare porzioni di codice basate su descrizioni ad alto
  livello fornite dai programmatori. Questo permette di velocizzare
  notevolmente lo sviluppo iniziale e ridurre gli errori di sintassi.
\item
  Assistenza nel debugging: L'IA può identificare potenziali bug e
  suggerire correzioni, rendendo il processo di debugging più efficiente
  e meno dispendioso in termini di tempo.
\item
  Ottimizzazione automatica: Gli algoritmi di IA possono analizzare il
  codice e suggerire o applicare automaticamente ottimizzazioni per
  migliorare le prestazioni.
\item
  Generazione di casi di test: L'IA può creare casi di test per
  verificare la correttezza del codice, coprendo una gamma più ampia di
  scenari di quanto un programmatore potrebbe fare manualmente.
\item
  Documentazione automatica: L'IA può generare documentazione leggendo e
  interpretando il codice, riducendo il carico di lavoro manuale e
  garantendo una documentazione coerente e aggiornata.
\end{enumerate}

\subsection{L'Importanza di imparare a programmare nell'era dell'IA
generativa}\label{limportanza-di-imparare-a-programmare-nellera-dellia-generativa}

Nonostante l'avvento dell'IA generativa, imparare a programmare rimane
fondamentale per diverse ragioni. La programmazione non è solo una
competenza tecnica, ma anche un modo di pensare e risolvere problemi.
Comprendere i fondamenti della programmazione è essenziale per
utilizzare efficacemente gli strumenti di IA generativa. Senza una
solida base, è difficile sfruttare appieno queste tecnologie. Inoltre,
la programmazione insegna a scomporre problemi complessi in parti più
gestibili e a trovare soluzioni logiche e sequenziali, una competenza
preziosa in molti campi.

Anche con l'IA generativa, esisteranno sempre situazioni in cui sarà
necessario personalizzare o ottimizzare il codice per esigenze
specifiche. La conoscenza della programmazione permette di fare queste
modifiche con sicurezza. Inoltre, quando qualcosa va storto, è
indispensabile sapere come leggere e comprendere il codice per
identificare e risolvere i problemi. L'IA può assistere, ma la
comprensione umana rimane cruciale per interventi mirati.

Imparare a programmare consente di sperimentare nuove idee e prototipare
rapidamente soluzioni innovative. La creatività è potenziata dalla
capacità di tradurre idee in codice funzionante. Sapere programmare
aiuta anche a comprendere i limiti e le potenzialità degli strumenti di
IA generativa, permettendo di usarli in modo più strategico ed efficace.

La tecnologia evolve rapidamente, e con una conoscenza della
programmazione si è meglio preparati ad adattarsi alle nuove tecnologie
e metodologie che emergeranno in futuro. Inoltre, la programmazione è
una competenza trasversale applicabile in numerosi settori, dalla
biologia computazionale alla finanza, dall'ingegneria all'arte digitale.
Avere questa competenza amplia notevolmente le opportunità di carriera.

Infine, la programmazione è una porta d'accesso a ruoli più avanzati e
specializzati nel campo della tecnologia, come l'ingegneria del
software, la scienza dei dati e la ricerca sull'IA. Conoscere i principi
della programmazione aiuta a comprendere meglio come funzionano gli
algoritmi di IA, permettendo di contribuire attivamente allo sviluppo di
nuove tecnologie.

\chapter{Paradigmi di programmazione}\label{paradigmi-di-programmazione}

I linguaggi di programmazione possono essere classificati in diversi
tipi in base al loro scopo e alla loro struttura. Una delle
classificazioni più importanti è quella del \textbf{paradigma di
programmazione}, che definisce il modello e gli stili di risoluzione dei
problemi che un linguaggio supporta. Tuttavia, è importante notare che
molti linguaggi moderni supportano più di un paradigma di
programmazione, rendendo difficile assegnare un linguaggio a una sola
categoria. Come ha affermato Bjarne Stroustrup, il creatore di C++,
\emph{un linguaggio di programmazione non è semplicemente supportare un
certo paradigma, ma abilitare un certo stile di programmazione}
(Stroustrup 1997).

\section{L'importanza dei paradigmi di
programmazione}\label{limportanza-dei-paradigmi-di-programmazione}

Comprendere i paradigmi di programmazione è fondamentale per diversi
motivi:

\begin{itemize}
\tightlist
\item
  Approccio alla risoluzione dei problemi: Ogni paradigma offre una
  visione diversa su come affrontare e risolvere problemi. Conoscere
  vari paradigmi permette ai programmatori di scegliere l'approccio più
  adatto in base al problema specifico. Ad esempio, per problemi che
  richiedono una manipolazione di stati, la programmazione imperativa
  può essere più intuitiva. Al contrario, per problemi che richiedono
  trasformazioni di dati senza effetti collaterali, la programmazione
  funzionale potrebbe essere più adatta.
\item
  Versatilità e adattabilità: I linguaggi moderni che supportano più
  paradigmi permettono ai programmatori di essere più versatili e
  adattabili. Possono utilizzare il paradigma più efficiente per diverse
  parti del progetto, migliorando sia la leggibilità che le prestazioni
  del codice.
\item
  Manutenzione del codice: La comprensione dei paradigmi aiuta nella
  scrittura di codice più chiaro e manutenibile. Ad esempio, il
  paradigma orientato agli oggetti può essere utile per organizzare
  grandi basi di codice in moduli e componenti riutilizzabili,
  migliorando la gestione del progetto.
\item
  Evoluzione professionale: La conoscenza dei vari paradigmi arricchisce
  le competenze di un programmatore, rendendolo più competitivo nel
  mercato del lavoro. Conoscere più paradigmi permette di comprendere e
  lavorare con una gamma più ampia di linguaggi di programmazione e
  tecnologie.
\item
  Ottimizzazione del codice: Alcuni paradigmi sono più efficienti in
  determinate situazioni. Ad esempio, la programmazione concorrente è
  essenziale per lo sviluppo di software che richiede alta prestazione e
  scalabilità, come nei sistemi distribuiti. Comprendere come
  implementare la concorrenza in vari paradigmi permette di scrivere
  codice più efficiente.
\end{itemize}

\section{Paradigma imperativo}\label{paradigma-imperativo}

La \textbf{programmazione imperativa}, a differenza della programmazione
dichiarativa, è un paradigma di programmazione che descrive l'esecuzione
di un programma come una serie di istruzioni che cambiano il suo stato.
In modo simile al modo imperativo nelle lingue naturali, che esprime
comandi per compiere azioni, i programmi imperativi sono una sequenza di
comandi che il computer deve eseguire. Un caso particolare di
programmazione imperativa è quella procedurale.

I linguaggi di programmazione imperativa si contrappongono ad altri tipi
di linguaggi, come quelli funzionali e logici. I linguaggi di
programmazione funzionale, come Haskell, non producono sequenze di
istruzioni e non hanno uno stato globale come i linguaggi imperativi. I
linguaggi di programmazione logica, come Prolog, sono caratterizzati
dalla definizione di cosa deve essere calcolato, piuttosto che come deve
avvenire il calcolo, a differenza di un linguaggio di programmazione
imperativo.

L'implementazione hardware di quasi tutti i computer è imperativa perché
è progettata per eseguire il codice macchina, che è scritto in stile
imperativo. Da questa prospettiva a basso livello, lo stato del
programma è definito dal contenuto della memoria e dalle istruzioni nel
linguaggio macchina nativo del processore. I linguaggi imperativi di
alto livello utilizzano variabili e istruzioni più complesse, ma seguono
ancora lo stesso paradigma per mantenere la coerenza nella traduzione
(compilazione o interpretazione) del codice.

\subsection{Esempio in Assembly}\label{esempio-in-assembly}

Assembly è un linguaggio a basso livello strettamente legato
all'hardware del computer. Ecco un esempio di un semplice programma
scritto per l'architettura x86, utilizzando la sintassi dell'assembler
NASM (Netwide Assembler). Il codice somma due numeri e stampa il
risultato:

\begin{Shaded}
\begin{Highlighting}[]
\NormalTok{section .data}
\NormalTok{    num1 db 5           ; Definisce il primo numero}
\NormalTok{    num2 db 3           ; Definisce il secondo numero}
\NormalTok{    result db 0         ; Variabile per memorizzare il risultato}

\NormalTok{section .text}
\NormalTok{    global \_start}

\NormalTok{\_start:}
\NormalTok{    mov al, [num1]      ; Carica il primo numero in AL}
\NormalTok{    add al, [num2]      ; Aggiunge il secondo numero a AL}
\NormalTok{    mov [result], al    ; Memorizza il risultato in result}

\NormalTok{    ; Print result (pseudo{-}syscall for simplicity)}
\NormalTok{    ; In realtà, si dovrebbe convertire il risultato in ASCII e chiamare le syscall appropriate.}
    
\NormalTok{    ; Terminazione del programma}
\NormalTok{    mov eax, 1          ; Codice di sistema per l\textquotesingle{}uscita}
\NormalTok{    int 0x80            ; Interruzione per chiamare il kernel}
\end{Highlighting}
\end{Shaded}

Le sezioni del codice:

\begin{itemize}
\tightlist
\item
  Dati: La sezione \texttt{.data} è utilizzata per dichiarare variabili
  statiche inizializzate.
\item
  Testo: La sezione \texttt{.text} contiene il codice eseguibile.
  L'etichetta \texttt{\_start} indica il punto di ingresso del
  programma.
\item
  Registri: I registri come \texttt{al} e \texttt{eax} sono utilizzati
  per operazioni aritmetiche e per la memorizzazione temporanea dei
  dati.
\item
  Chiamate di sistema: L'interruzione \texttt{int\ 0x80} è usata per
  eseguire chiamate di sistema in Linux.
\end{itemize}

L'Assembly x86 è usato nello sviluppo di:

\begin{itemize}
\tightlist
\item
  Sistemi operativi: Utilizzato nello sviluppo di kernel e driver.
\item
  Applicazioni embedded (microcontrollori nei dispositivi medici,
  sistemi di controllo nei veicoli, dispositivi IoT, ecc.): Dove è
  necessaria un'ottimizzazione estrema delle risorse computazionali.
\item
  Applicazioni HPC (high performance computing): Il focus qui è eseguire
  calcoli intensivi e complessi in tempi relativamente brevi. Queste
  applicazioni richiedono un numero di operazioni per unità di tempo
  elevato e sono ottimizzate per sfruttare al massimo le risorse
  hardware disponibili, come CPU, GPU e memoria.
\end{itemize}

\subsection{Esempio in Python}\label{esempio-in-python}

All'altro estremo della immediatezza di comprensione del testo del
codice troviamo Python, un linguaggio di alto livello noto per la
leggibilità ed eleganza. Ecco il medesimo esempio:

\begin{Shaded}
\begin{Highlighting}[]
\CommentTok{\# Definizione delle variabili}
\NormalTok{num1 }\OperatorTok{=} \DecValTok{5}
\NormalTok{num2 }\OperatorTok{=} \DecValTok{3}

\CommentTok{\# Somma dei due numeri}
\NormalTok{result }\OperatorTok{=}\NormalTok{ num1 }\OperatorTok{+}\NormalTok{ num2}

\CommentTok{\# Stampa del risultato}
\BuiltInTok{print}\NormalTok{(}\StringTok{"Il risultato è:"}\NormalTok{, result)}
\end{Highlighting}
\end{Shaded}

\subsection{Analisi comparativa}\label{analisi-comparativa}

Assembly:

\begin{itemize}
\tightlist
\item
  Basso livello di astrazione: Assembly lavora direttamente con i
  registri della CPU e la memoria, quindi non astrae granché della
  complessità dell'hardware.
\item
  Versatilità: Il linguaggio è progettato per una ben definita
  architettura e, quindi, ha una scarsa applicabilità ad altre.
\item
  Precisione: Il programmatore ha un controllo dettagliato su ogni
  singola operazione.
\item
  Complessità: Ogni operazione deve essere definita esplicitamente e in
  sequenza, il che rende il codice più lungo e difficile da leggere.
\end{itemize}

Python:

\begin{itemize}
\tightlist
\item
  Alto livello di astrazione: Python fornisce un'astrazione più elevata,
  permettendo di ignorare i dettagli dell'hardware.
\item
  Semplicità: Il codice è più breve e leggibile, facilitando la
  comprensione e la manutenzione.
\item
  Versatilità: Il linguaggio è applicabile senza modifiche a un elevato
  numero di architetture hardware-software.
\item
  Produttività: I programmatori possono concentrarsi sulla logica del
  problema senza preoccuparsi dei dettagli implementativi.
\end{itemize}

\section{Paradigma procedurale}\label{paradigma-procedurale}

La \textbf{programmazione procedurale} è un paradigma di programmazione,
derivato da quella imperativa, che organizza il codice in unità chiamate
procedure o funzioni. Ogni procedura o funzione è un blocco di codice
che può essere richiamato da altre parti del programma, promuovendo la
riutilizzabilità e la modularità del codice.

La programmazione procedurale è una naturale evoluzione della imperativa
e uno dei paradigmi più antichi e ampiamente utilizzati. Ha avuto
origine negli anni '60 e '70 con linguaggi come Fortan, COBOL e C,
tutt'oggi rilevanti. Questi linguaggi hanno introdotto concetti
fondamentali come funzioni, sottoprogrammi e la separazione tra codice e
dati. Il C, in particolare, ha avuto un impatto duraturo sulla
programmazione procedurale, diventando uno standard de facto per lo
sviluppo di sistemi operativi e software di sistema.

I vantaggi principali sono:

\begin{itemize}
\tightlist
\item
  Modularità: La programmazione procedurale incoraggia la suddivisione
  del codice in funzioni o procedure più piccole e gestibili. Questo
  facilita la comprensione, la manutenzione e il riutilizzo del codice.
\item
  Riutilizzabilità: Le funzioni possono essere riutilizzate in diverse
  parti del programma o in progetti diversi, riducendo la duplicazione
  del codice e migliorando l'efficienza dello sviluppo.
\item
  Struttura e organizzazione: Il codice procedurale è generalmente più
  strutturato e organizzato, facilitando la lettura e la gestione del
  progetto software.
\item
  Facilità di debug e testing: La suddivisione del programma in funzioni
  isolate rende più facile individuare e correggere errori, oltre a
  testare parti specifiche del codice.
\end{itemize}

D'altro canto, presenta anche degli svantaggi che hanno spinto i
ricercatori a continuare l'innovazione:

\begin{itemize}
\tightlist
\item
  Scalabilità limitata: Nei progetti molto grandi, la programmazione
  procedurale può diventare difficile da gestire. La mancanza di
  meccanismi di astrazione avanzati, come quelli offerti dalla
  programmazione orientata agli oggetti, può complicare la gestione
  della complessità.
\item
  Gestione dello stato: La programmazione procedurale si basa spesso su
  variabili globali per condividere stato tra le funzioni, il che può
  portare a bug difficili da individuare e risolvere.
\item
  Difficoltà nell'aggiornamento: Le modifiche a una funzione possono
  richiedere aggiornamenti in tutte le parti del programma che la
  utilizzano, aumentando il rischio di introdurre nuovi errori.
\item
  Meno Adatta per Applicazioni Moderne: Per applicazioni complesse e
  moderne che richiedono la gestione di eventi, interfacce utente
  complesse e modellazione del dominio, la programmazione procedurale
  può essere meno efficace rispetto ad altri paradigmi come quello
  orientato agli oggetti.
\end{itemize}

\subsection{Funzioni e procedure}\label{funzioni-e-procedure}

Nella programmazione procedurale, il codice è suddiviso in unità
elementari chiamate \textbf{funzioni} e \textbf{procedure}. La
differenza principale tra le due è la seguente:

\begin{itemize}
\item
  Funzione: Una funzione è un blocco di codice che esegue un compito
  specifico e restituisce un valore. Le funzioni sono utilizzate per
  calcoli o operazioni che producono un risultato. Ad esempio, una
  funzione che calcola la somma di due numeri:

\begin{Shaded}
\begin{Highlighting}[]
\DataTypeTok{int}\NormalTok{ somma}\OperatorTok{(}\DataTypeTok{int}\NormalTok{ a}\OperatorTok{,} \DataTypeTok{int}\NormalTok{ b}\OperatorTok{)} \OperatorTok{\{}
    \ControlFlowTok{return}\NormalTok{ a }\OperatorTok{+}\NormalTok{ b}\OperatorTok{;}
\OperatorTok{\}}
\end{Highlighting}
\end{Shaded}
\item
  Procedura: Una procedura è simile a una funzione, ma non restituisce
  un valore. È utilizzata per eseguire azioni o operazioni che non
  necessitano di un risultato. Ad esempio, una procedura che stampa un
  messaggio:

\begin{Shaded}
\begin{Highlighting}[]
\KeywordTok{procedure}\NormalTok{ stampaMessaggio;}
\KeywordTok{begin}
\NormalTok{    writeln(}\StringTok{\textquotesingle{}Ciao, Mondo!\textquotesingle{}}\NormalTok{);}
\KeywordTok{end}\NormalTok{;}
\end{Highlighting}
\end{Shaded}
\end{itemize}

\subsection{Creazione di librerie}\label{creazione-di-librerie}

Un altro aspetto importante della programmazione procedurale è la
possibilità di creare \textbf{librerie}, che sono collezioni di funzioni
e procedure riutilizzabili. Le librerie permettono di organizzare e
condividere codice comune tra diversi progetti, aumentando la
produttività e riducendo la duplicazione del codice.

Esempio di una semplice libreria in C:

\begin{itemize}
\item
  File header (\texttt{mialibreria.h}): ```c \#ifndef MIALIBRERIA\_H
  \#define MIALIBRERIA\_H

  int somma(int a, int b); void stampaMessaggio(const char* messaggio);

  \#endif ```
\item
  File di implementazione (\texttt{mialibreria.c}): ```c \#include
  ``mialibreria.h'' \#include \textless stdio.h\textgreater{}

  int somma(int a, int b) \{ return a + b; \}

  void stampaMessaggio(const char* messaggio) \{ printf(``\%s\n'',
  messaggio); \} ```
\item
  File principale (\texttt{main.c}): ```c \#include ``mialibreria.h''

  int main() \{ int risultato = somma(5, 3); stampaMessaggio(``Il
  risultato è:''); printf(``\%d\n'', risultato); return 0; \} ```
\end{itemize}

\section{Paradigma di orientamento agli
oggetti}\label{paradigma-di-orientamento-agli-oggetti}

La \textbf{programmazione orientata agli oggetti} (in inglese
object-oriented programming, OOP) è un paradigma di programmazione che
organizza il software in termini di \emph{oggetti}, ciascuno dei quali
rappresenta un'istanza di una ``classe''. Una classe definisce un tipo
di dato che include attributi (dati) e metodi (funzionalità). Gli
oggetti interagiscono tra loro attraverso messaggi, permettendo una
struttura modulare e intuitiva.

L'OOP è emersa negli anni '60 e '70 con il linguaggio Simula, il primo
linguaggio di programmazione a supportare questo paradigma. Tuttavia, è
stato con Smalltalk, sviluppato negli anni '70 da Alan Kay e altri
presso Xerox PARC, che l'OOP ha guadagnato popolarità. Il paradigma è
stato ulteriormente consolidato con il linguaggio C++ negli anni '80 e
con Java negli anni '90, rendendolo uno dei più utilizzati per lo
sviluppo software moderno. Oggi numerosi sono i linguaggi a oggetti, ad
esempio Python, C\#, Ruby, Swift, Javascript, ecc. ed altri lo
supportano come PHP (dalla versione 5) e financo il Fortran nella
versione 2003.

Rispetto ai paradigmi precedenti, l'OOP introduce diversi concetti
chiave che ineriscono al disegno architetturale di software:

\begin{itemize}
\tightlist
\item
  Classe e oggetto: La classe è un modello o schema per creare oggetti.
  Contiene definizioni di attributi e metodi. L'oggetto è un'istanza di
  una classe e rappresenta un'entità concreta con stato e comportamento.
\item
  Incapsulamento: Nasconde i dettagli interni di un oggetto e mostra
  solo le interfacce necessarie. Migliora la modularità e protegge
  l'integrità dei dati.
\item
  Ereditarietà: Permette a una classe di estenderne un'altra,
  ereditandone attributi e metodi. Favorisce il riuso del codice e
  facilita l'estensione delle funzionalità.
\item
  Polimorfismo: Consente a oggetti di classi diverse di essere trattati
  come oggetti di una classe comune. Facilita l'uso di un'interfaccia
  uniforme per operazioni diverse.
\item
  Astrazione: Permette di definire interfacce di alto livello per
  oggetti, senza esporre i dettagli implementativi. Facilita la
  comprensione e la gestione della complessità del sistema.
\end{itemize}

I vantaggi principali dell'OOP sono:

\begin{itemize}
\tightlist
\item
  Modularità: Le classi e gli oggetti favoriscono la suddivisione del
  codice in moduli indipendenti, migliorando la manutenibilità e la
  riusabilità del software.
\item
  Riutilizzabilità: L'uso di classi e l'ereditarietà consentono di
  riutilizzare il codice in nuovi progetti senza riscriverlo.
\item
  Facilità di manutenzione: L'incapsulamento e l'astrazione riducono la
  complessità e facilitano la manutenzione del codice.
\item
  Estendibilità: Le classi possono essere estese per aggiungere nuove
  funzionalità senza modificare il codice esistente.
\item
  Affidabilità: Il polimorfismo e l'ereditarietà migliorano
  l'affidabilità del codice, poiché le modifiche possono essere fatte in
  una classe base e propagate alle classi derivate.
\end{itemize}

Anche se sussistono dei caveat:

\begin{itemize}
\tightlist
\item
  Complessità iniziale: L'OOP può essere complesso da apprendere e
  implementare correttamente per i nuovi programmatori.
\item
  Overhead di prestazioni: L'uso intensivo di oggetti può introdurre un
  overhead di memoria e prestazioni rispetto alla programmazione
  procedurale.
\item
  Abuso di ereditarietà: L'uso improprio dell'ereditarietà può portare a
  gerarchie di classi troppo complesse e difficili da gestire.
\end{itemize}

\subsection{Template}\label{template}

I template, o generics, non sono specifici dell'OOP, anche se sono
spesso associati a essa. I template permettono di scrivere funzioni,
classi, e altri costrutti di codice in modo generico, cioè indipendente
dal tipo dei dati che manipolano. Questo concetto è particolarmente
utile per creare librerie e moduli riutilizzabili e flessibili.

Ad esempio, definiamo la classe \texttt{Box} nel modo seguente:

\begin{Shaded}
\begin{Highlighting}[]
\KeywordTok{template} \OperatorTok{\textless{}}\KeywordTok{typename}\NormalTok{ T}\OperatorTok{\textgreater{}}
\KeywordTok{class}\NormalTok{ Box }\OperatorTok{\{}
\NormalTok{    T value}\OperatorTok{;}
\KeywordTok{public}\OperatorTok{:}
    \DataTypeTok{void}\NormalTok{ setValue}\OperatorTok{(}\NormalTok{T val}\OperatorTok{)} \OperatorTok{\{}\NormalTok{ value }\OperatorTok{=}\NormalTok{ val}\OperatorTok{;} \OperatorTok{\}}
\NormalTok{    T getValue}\OperatorTok{()} \OperatorTok{\{} \ControlFlowTok{return}\NormalTok{ value}\OperatorTok{;} \OperatorTok{\}}
\OperatorTok{\};}
\end{Highlighting}
\end{Shaded}

Essa può contenere un valore di qualsiasi tipo specificato al momento
della creazione dell'istanza per mezzo del template \texttt{T}:

\begin{Shaded}
\begin{Highlighting}[]
\CommentTok{// Box contiene un intero}
\NormalTok{Box}\OperatorTok{\textless{}}\DataTypeTok{int}\OperatorTok{\textgreater{}}\NormalTok{ intBox}\OperatorTok{;}
\NormalTok{intBox}\OperatorTok{.}\NormalTok{setValue}\OperatorTok{(}\DecValTok{123}\OperatorTok{);}
\DataTypeTok{int}\NormalTok{ x }\OperatorTok{=}\NormalTok{ intBox}\OperatorTok{.}\NormalTok{getValue}\OperatorTok{();}

\CommentTok{// Box contiene una stringa}
\NormalTok{Box}\OperatorTok{\textless{}}\BuiltInTok{std::}\NormalTok{string}\OperatorTok{\textgreater{}}\NormalTok{ stringBox}\OperatorTok{;}
\NormalTok{stringBox}\OperatorTok{.}\NormalTok{setValue}\OperatorTok{(}\StringTok{"Hello, World!"}\OperatorTok{);}
\BuiltInTok{std::}\NormalTok{string str }\OperatorTok{=}\NormalTok{ stringBox}\OperatorTok{.}\NormalTok{getValue}\OperatorTok{();}
\end{Highlighting}
\end{Shaded}

Anche nei linguaggi non orientati agli oggetti, i template trovano
applicazione. Ad esempio, in Rust, un linguaggio di programmazione
sistemistica non puramente OOP, il codice seguente restituisce il valore
più grande di una lista:

\begin{Shaded}
\begin{Highlighting}[]
\KeywordTok{fn}\NormalTok{ largest}\OperatorTok{\textless{}}\NormalTok{T}\OperatorTok{:} \BuiltInTok{PartialOrd}\OperatorTok{\textgreater{}}\NormalTok{(list}\OperatorTok{:} \OperatorTok{\&}\NormalTok{[T]) }\OperatorTok{{-}\textgreater{}} \OperatorTok{\&}\NormalTok{T }\OperatorTok{\{}
    \KeywordTok{let} \KeywordTok{mut}\NormalTok{ largest }\OperatorTok{=} \OperatorTok{\&}\NormalTok{list[}\DecValTok{0}\NormalTok{]}\OperatorTok{;}
    \ControlFlowTok{for}\NormalTok{ item }\KeywordTok{in}\NormalTok{ list }\OperatorTok{\{}
        \ControlFlowTok{if}\NormalTok{ item }\OperatorTok{\textgreater{}}\NormalTok{ largest }\OperatorTok{\{}
\NormalTok{            largest }\OperatorTok{=}\NormalTok{ item}\OperatorTok{;}
        \OperatorTok{\}}
    \OperatorTok{\}}
\NormalTok{    largest}
\OperatorTok{\}}

\KeywordTok{fn}\NormalTok{ main() }\OperatorTok{\{}
    \KeywordTok{let}\NormalTok{ numbers }\OperatorTok{=} \PreprocessorTok{vec!}\NormalTok{[}\DecValTok{34}\OperatorTok{,} \DecValTok{50}\OperatorTok{,} \DecValTok{25}\OperatorTok{,} \DecValTok{100}\OperatorTok{,} \DecValTok{65}\NormalTok{]}\OperatorTok{;}
    \KeywordTok{let}\NormalTok{ max }\OperatorTok{=}\NormalTok{ largest(}\OperatorTok{\&}\NormalTok{numbers)}\OperatorTok{;}
    \PreprocessorTok{println!}\NormalTok{(}\StringTok{"The largest number is \{\}"}\OperatorTok{,}\NormalTok{ max)}\OperatorTok{;}
\OperatorTok{\}}
\end{Highlighting}
\end{Shaded}

\subsection{Metaprogrammazione}\label{metaprogrammazione}

La metaprogrammazione è un paradigma che consente al programma di
trattare il codice come dati, permettendo al codice di generare,
manipolare o eseguire altro codice. Anche questo concetto non è
esclusivo dell'OOP. In C++, la metaprogrammazione è strettamente legata
ai template. Un esempio classico è la template metaprogramming (TMP),
che permette di eseguire calcoli a tempo di compilazione. Un esempio è
il codice seguente di calcolo del fattoriale:

\begin{Shaded}
\begin{Highlighting}[]
\KeywordTok{template}\OperatorTok{\textless{}}\DataTypeTok{int}\NormalTok{ N}\OperatorTok{\textgreater{}}
\KeywordTok{struct}\NormalTok{ Factorial }\OperatorTok{\{}
    \AttributeTok{static} \AttributeTok{const} \DataTypeTok{int}\NormalTok{ value }\OperatorTok{=}\NormalTok{ N }\OperatorTok{*}\NormalTok{ Factorial}\OperatorTok{\textless{}}\NormalTok{N }\OperatorTok{{-}} \DecValTok{1}\OperatorTok{\textgreater{}::}\NormalTok{value}\OperatorTok{;}
\OperatorTok{\};}
\KeywordTok{template}\OperatorTok{\textless{}\textgreater{}}
\KeywordTok{struct}\NormalTok{ Factorial}\OperatorTok{\textless{}}\DecValTok{0}\OperatorTok{\textgreater{}} \OperatorTok{\{}
    \AttributeTok{static} \AttributeTok{const} \DataTypeTok{int}\NormalTok{ value }\OperatorTok{=} \DecValTok{1}\OperatorTok{;}
\OperatorTok{\};}
\end{Highlighting}
\end{Shaded}

La metaprogrammazione è presente anche in linguaggi non OOP come Lisp,
che utilizza le macro per trasformare e generare codice. Nel codice soto
proposto è definita la macro \texttt{when} che prende due parametri in
input cioè \texttt{test} e \texttt{body}, ove \texttt{test} è
un'espressione comdizionale e \texttt{body} un insieme di iestruzioni da
eeseguire se la condizione è vera:

\begin{Shaded}
\begin{Highlighting}[]
\NormalTok{(}\KeywordTok{defmacro}\FunctionTok{ when }\NormalTok{(test \&}\KeywordTok{rest}\NormalTok{ body)}
\NormalTok{  \textasciigrave{}(}\KeywordTok{if}\NormalTok{ ,test}
\NormalTok{       (}\KeywordTok{progn}\NormalTok{ ,@body)))}
\end{Highlighting}
\end{Shaded}

Vediamo un esempio pratico di come si utilizza la macro \texttt{when}.
Il test è valutare se \texttt{x} è maggiore di \texttt{10} e, nel caso,
stampare \texttt{"x\ is\ greater\ than\ 10"} e poi assegnare \texttt{x}
a \texttt{0}. Chiamiamo la macro con i due parametri:

\begin{Shaded}
\begin{Highlighting}[]
\NormalTok{(}\KeywordTok{when}\NormalTok{ (}\OperatorTok{\textgreater{}}\NormalTok{ x }\DecValTok{10}\NormalTok{)}
\NormalTok{  (}\KeywordTok{print} \StringTok{"x is greater than 10"}\NormalTok{)}
\NormalTok{  (}\KeywordTok{setf}\NormalTok{ x }\DecValTok{0}\NormalTok{))}
\end{Highlighting}
\end{Shaded}

Questo viene espanso in:

\begin{Shaded}
\begin{Highlighting}[]
\NormalTok{(}\KeywordTok{if}\NormalTok{ (}\OperatorTok{\textgreater{}}\NormalTok{ x }\DecValTok{10}\NormalTok{)}
\NormalTok{    (}\KeywordTok{progn}
\NormalTok{      (}\KeywordTok{print} \StringTok{"x is greater than 10"}\NormalTok{)}
\NormalTok{      (}\KeywordTok{setf}\NormalTok{ x }\DecValTok{0}\NormalTok{)))}
\end{Highlighting}
\end{Shaded}

\subsection{Esempio in Java}\label{esempio-in-java}

In questo esempio, la classe \texttt{Animale} rappresenta una classe
base con un attributo \texttt{nome} e un metodo \texttt{faiVerso}. La
classe \texttt{Cane} estende \texttt{Animale} e sovrascrive il metodo
\texttt{faiVerso} per fornire un'implementazione specifica. La classe
\texttt{Main} crea un'istanza di \texttt{Cane} e chiama il metodo
\texttt{faiVerso}, dimostrando il polimorfismo e l'ereditarietà:

\begin{Shaded}
\begin{Highlighting}[]
\CommentTok{// Definizione della classe base}
\KeywordTok{class}\NormalTok{ Animale }\OperatorTok{\{}
    \BuiltInTok{String}\NormalTok{ nome}\OperatorTok{;}

    \FunctionTok{Animale}\OperatorTok{(}\BuiltInTok{String}\NormalTok{ nome}\OperatorTok{)} \OperatorTok{\{}
        \KeywordTok{this}\OperatorTok{.}\FunctionTok{nome} \OperatorTok{=}\NormalTok{ nome}\OperatorTok{;}
    \OperatorTok{\}}

    \DataTypeTok{void} \FunctionTok{faiVerso}\OperatorTok{()} \OperatorTok{\{}
        \BuiltInTok{System}\OperatorTok{.}\FunctionTok{out}\OperatorTok{.}\FunctionTok{println}\OperatorTok{(}\StringTok{"L\textquotesingle{}animale fa un verso"}\OperatorTok{);}
    \OperatorTok{\}}
\OperatorTok{\}}

\CommentTok{// Definizione della classe derivata}
\KeywordTok{class}\NormalTok{ Cane }\KeywordTok{extends}\NormalTok{ Animale }\OperatorTok{\{}

    \FunctionTok{Cane}\OperatorTok{(}\BuiltInTok{String}\NormalTok{ nome}\OperatorTok{)} \OperatorTok{\{}
        \KeywordTok{super}\OperatorTok{(}\NormalTok{nome}\OperatorTok{);}
    \OperatorTok{\}}

    \AttributeTok{@Override}
    \DataTypeTok{void} \FunctionTok{faiVerso}\OperatorTok{()} \OperatorTok{\{}
        \BuiltInTok{System}\OperatorTok{.}\FunctionTok{out}\OperatorTok{.}\FunctionTok{println}\OperatorTok{(}\StringTok{"Il cane abbaia"}\OperatorTok{);}
    \OperatorTok{\}}
\OperatorTok{\}}

\CommentTok{// Classe principale}
\KeywordTok{public} \KeywordTok{class}\NormalTok{ Main }\OperatorTok{\{}
    \KeywordTok{public} \DataTypeTok{static} \DataTypeTok{void} \FunctionTok{main}\OperatorTok{(}\BuiltInTok{String}\OperatorTok{[]}\NormalTok{ args}\OperatorTok{)} \OperatorTok{\{}
\NormalTok{        Animale mioCane }\OperatorTok{=} \KeywordTok{new} \FunctionTok{Cane}\OperatorTok{(}\StringTok{"Fido"}\OperatorTok{);}
\NormalTok{        mioCane}\OperatorTok{.}\FunctionTok{faiVerso}\OperatorTok{();} \CommentTok{// Stampa: Il cane abbaia}
    \OperatorTok{\}}
\OperatorTok{\}}
\end{Highlighting}
\end{Shaded}

Vediamo come implementare il medesimo esempio con una classe astratta in
Java. \texttt{Animale} è la classe astratta, cioè che non può essere
istanziata in un oggetto e \texttt{Cane} è una classe concreta che la
estende:

\begin{Shaded}
\begin{Highlighting}[]
\CommentTok{// Definizione della classe astratta}
\KeywordTok{abstract} \KeywordTok{class}\NormalTok{ Animale }\OperatorTok{\{}
    \BuiltInTok{String}\NormalTok{ nome}\OperatorTok{;}

    \FunctionTok{Animale}\OperatorTok{(}\BuiltInTok{String}\NormalTok{ nome}\OperatorTok{)} \OperatorTok{\{}
        \KeywordTok{this}\OperatorTok{.}\FunctionTok{nome} \OperatorTok{=}\NormalTok{ nome}\OperatorTok{;}
    \OperatorTok{\}}

    \CommentTok{// Metodo astratto}
    \KeywordTok{abstract} \DataTypeTok{void} \FunctionTok{faiVerso}\OperatorTok{();}

    \CommentTok{// Metodo concreto}
    \DataTypeTok{void} \FunctionTok{descrizione}\OperatorTok{()} \OperatorTok{\{}
        \BuiltInTok{System}\OperatorTok{.}\FunctionTok{out}\OperatorTok{.}\FunctionTok{println}\OperatorTok{(}\StringTok{"L\textquotesingle{}animale si chiama "} \OperatorTok{+}\NormalTok{ nome}\OperatorTok{);}
    \OperatorTok{\}}
\OperatorTok{\}}

\CommentTok{// Definizione della classe derivata}
\KeywordTok{class}\NormalTok{ Cane }\KeywordTok{extends}\NormalTok{ Animale }\OperatorTok{\{}

    \FunctionTok{Cane}\OperatorTok{(}\BuiltInTok{String}\NormalTok{ nome}\OperatorTok{)} \OperatorTok{\{}
        \KeywordTok{super}\OperatorTok{(}\NormalTok{nome}\OperatorTok{);}
    \OperatorTok{\}}

    \AttributeTok{@Override}
    \DataTypeTok{void} \FunctionTok{faiVerso}\OperatorTok{()} \OperatorTok{\{}
        \BuiltInTok{System}\OperatorTok{.}\FunctionTok{out}\OperatorTok{.}\FunctionTok{println}\OperatorTok{(}\StringTok{"Il cane abbaia"}\OperatorTok{);}
    \OperatorTok{\}}
\OperatorTok{\}}

\CommentTok{// Classe principale}
\KeywordTok{public} \KeywordTok{class}\NormalTok{ Main }\OperatorTok{\{}
    \KeywordTok{public} \DataTypeTok{static} \DataTypeTok{void} \FunctionTok{main}\OperatorTok{(}\BuiltInTok{String}\OperatorTok{[]}\NormalTok{ args}\OperatorTok{)} \OperatorTok{\{}
\NormalTok{        Animale mioCane }\OperatorTok{=} \KeywordTok{new} \FunctionTok{Cane}\OperatorTok{(}\StringTok{"Fido"}\OperatorTok{);}
\NormalTok{        mioCane}\OperatorTok{.}\FunctionTok{descrizione}\OperatorTok{();} \CommentTok{// Stampa: L\textquotesingle{}animale si chiama Fido}
\NormalTok{        mioCane}\OperatorTok{.}\FunctionTok{faiVerso}\OperatorTok{();}    \CommentTok{// Stampa: Il cane abbaia}
    \OperatorTok{\}}
\OperatorTok{\}}
\end{Highlighting}
\end{Shaded}

\section{Paradigma dichiarativo}\label{paradigma-dichiarativo}

La \textbf{programmazione dichiarativa} è un paradigma di programmazione
che si focalizza sul \emph{cosa} deve essere calcolato piuttosto che sul
\emph{come} calcolarlo. In altre parole, i programmi dichiarativi
descrivono il risultato desiderato senza specificare esplicitamente i
passaggi per ottenerlo. Questo è in netto contrasto con la
programmazione imperativa, dove si fornisce una sequenza dettagliata di
istruzioni per modificare lo stato del programma.

La programmazione dichiarativa ha radici nella logica e nella
matematica, ed è emersa come un importante paradigma negli anni '70 e
'80 con l'avvento di linguaggi come Prolog (per la programmazione
logica) e SQL (per la gestione dei database). La programmazione
funzionale, con linguaggi come Haskell, è anch'essa una forma di
programmazione dichiarativa.

I concetti principali associati alla programazione dichiarativa sono:

\begin{itemize}
\tightlist
\item
  Descrizione del risultato: I programmi dichiarativi descrivono le
  proprietà del risultato desiderato senza specificare l'algoritmo per
  ottenerlo. Esempio: In SQL, per ottenere tutti i record di una tabella
  con un certo valore, si scrive una query che descrive la condizione,
  non un algoritmo che scorre i record uno per uno.
\item
  Assenza di stato esplicito: La programmazione dichiarativa evita l'uso
  esplicito di variabili di stato e di aggiornamenti di stato. Ciò
  riduce i rischi di effetti collaterali e rende il codice più facile da
  comprendere e verificare.
\item
  Idempotenza: Le espressioni dichiarative sono spesso idempotenti, cioè
  possono essere eseguite più volte senza cambiare il risultato. Questo
  è particolarmente utile per la concorrenza e la parallelizzazione.
\end{itemize}

Il vantaggio principale è relativo alla sua chiarezza perché ci si
concentra sul risultato desiderato piuttosto che sui dettagli di
implementazione.

La programmazione imperativa specifica come ottenere un risultato
mediante una sequenza di istruzioni, modificando lo stato del programma.
La programmazione dichiarativa, al contrario, specifica cosa deve essere
ottenuto senza descrivere i dettagli di implementazione. In termini di
livello di astrazione, la programmazione dichiarativa si trova a un
livello superiore rispetto a quella imperativa.

\subsection{Linguaggi}\label{linguaggi}

Ecco una lista di alcuni linguaggi di programmazione dichiarativi:

\begin{enumerate}
\def\labelenumi{\arabic{enumi}.}
\tightlist
\item
  SQL (Structured Query Language): Utilizzato per la gestione e
  l'interrogazione di database relazionali.
\item
  Prolog: Un linguaggio di programmazione logica usato principalmente
  per applicazioni di intelligenza artificiale e linguistica
  computazionale.
\item
  HTML (HyperText Markup Language): Utilizzato per creare e strutturare
  pagine web.
\item
  CSS (Cascading Style Sheets): Utilizzato per descrivere la
  presentazione delle pagine web scritte in HTML o XML.
\item
  XSLT (Extensible Stylesheet Language Transformations): Un linguaggio
  per trasformare documenti XML in altri formati.
\item
  Haskell: Un linguaggio funzionale che è anche dichiarativo, noto per
  la sua pura implementazione della programmazione funzionale.
\item
  Erlang: Un linguaggio utilizzato per sistemi concorrenti e
  distribuiti, con caratteristiche dichiarative.
\item
  VHDL (VHSIC Hardware Description Language): Utilizzato per descrivere
  il comportamento e la struttura di sistemi digitali.
\item
  Verilog: Un altro linguaggio di descrizione hardware usato per la
  modellazione di sistemi elettronici.
\item
  XQuery: Un linguaggio di query per interrogare documenti XML.
\end{enumerate}

Questi linguaggi rappresentano diversi ambiti di applicazione, dai
database alla descrizione hardware, e sono accomunati dall'approccio
dichiarativo nel quale si specifica cosa ottenere piuttosto che come
ottenerlo.

\subsection{Esempi}\label{esempi}

Esempio di una query SQL che estrae tutti i nomi degli utenti con età
maggiore di 30:

\begin{Shaded}
\begin{Highlighting}[]
\KeywordTok{SELECT}\NormalTok{ nome}
\KeywordTok{FROM}\NormalTok{ utenti}
\KeywordTok{WHERE}\NormalTok{ età }\OperatorTok{\textgreater{}} \DecValTok{30}\NormalTok{;}
\end{Highlighting}
\end{Shaded}

In Prolog, si definiscono fatti e regole che descrivono relazioni
logiche. Il motore di inferenza di Prolog utilizza queste definizioni
per risolvere query, senza richiedere un algoritmo dettagliato. Di
seguito, sono definiti due fatti (le prime due righe) e due regole (la
terza e la quarta) e quindi si effettua una query che dà coem risultato
\texttt{true}:

\begin{Shaded}
\begin{Highlighting}[]
\NormalTok{genitore(padre}\KeywordTok{,}\NormalTok{ figlio)}\KeywordTok{.}
\NormalTok{genitore(madre}\KeywordTok{,}\NormalTok{ figlio)}\KeywordTok{.}
\NormalTok{antenato(}\DataTypeTok{X}\KeywordTok{,} \DataTypeTok{Y}\NormalTok{) }\KeywordTok{:{-}}\NormalTok{ genitore(}\DataTypeTok{X}\KeywordTok{,} \DataTypeTok{Y}\NormalTok{)}\KeywordTok{.}
\NormalTok{antenato(}\DataTypeTok{X}\KeywordTok{,} \DataTypeTok{Y}\NormalTok{) }\KeywordTok{:{-}}\NormalTok{ genitore(}\DataTypeTok{X}\KeywordTok{,} \DataTypeTok{Z}\NormalTok{)}\KeywordTok{,}\NormalTok{ antenato(}\DataTypeTok{Z}\KeywordTok{,} \DataTypeTok{Y}\NormalTok{)}\KeywordTok{.}

\FunctionTok{?{-}}\NormalTok{ antenato(padre}\KeywordTok{,}\NormalTok{ figlio)}\KeywordTok{.}
\end{Highlighting}
\end{Shaded}

\section{Paradigma funzionale}\label{paradigma-funzionale}

La \textbf{programmazione funzionale} è un paradigma di programmazione
che tratta il calcolo come la valutazione di funzioni matematiche ed
evita lo stato mutabile e i dati modificabili. I programmi funzionali
sono costruiti applicando e componendo funzioni. Questo paradigma è
stato ispirato dal calcolo lambda, una formalizzazione matematica del
concetto di funzione. La programmazione funzionale è un paradigma
alternativo alla programmazione imperativa, che descrive la computazione
come una sequenza di istruzioni che modificano lo stato del programma.

La programmazione funzionale ha radici storiche che risalgono agli anni
'30, con il lavoro di Alonzo Church sul calcolo lambda. I linguaggi di
programmazione funzionale hanno iniziato a svilupparsi negli anni '50 e
'60 con Lisp, ma è stato negli anni '70 e '80 che linguaggi come ML e
Haskell hanno consolidato questo paradigma. Haskell, in particolare, è
stato progettato per esplorare nuove idee in programmazione funzionale e
ha avuto un impatto significativo sulla ricerca e sulla pratica del
software.

La programmazione funzionale è una forma di programmazione dichiarativa
che si basa su funzioni pure e immutabilità. Entrambi i paradigmi
evitano stati mutabili e si concentrano sul risultato finale, ma la
programmazione funzionale utilizza funzioni matematiche come unità
fondamentali di calcolo.

Concetti fondamentali:

\begin{itemize}
\tightlist
\item
  Immutabilità: I dati sono immutabili, il che significa che una volta
  creati non possono essere modificati. Questo riduce il rischio di
  effetti collaterali e rende il codice più prevedibile.
\item
  Funzioni di prima classe e di ordine superiore: Le funzioni possono
  essere passate come argomenti a altre funzioni, ritornate da funzioni,
  e assegnate a variabili. Le funzioni di ordine superiore accettano
  altre funzioni come argomenti o restituiscono funzioni.
\item
  Purezza: Le funzioni pure sono funzioni che, dato lo stesso input,
  restituiscono sempre lo stesso output e non causano effetti
  collaterali. Questo rende il comportamento del programma più facile da
  comprendere e prevedere.
\item
  Trasparenza referenziale: Un'espressione è trasparentemente
  referenziale se può essere sostituita dal suo valore senza cambiare il
  comportamento del programma. Questo facilita l'ottimizzazione e il
  reasonig sul codice.
\item
  Ricorsione: È spesso utilizzata al posto di loop iterativi per
  eseguire ripetizioni, poiché si adatta meglio alla natura immutabile
  dei dati e alla definizione di funzioni.
\item
  Composizione di Funzioni: Consente di costruire funzioni complesse
  combinando funzioni più semplici. Questo favorisce la modularità e la
  riusabilità del codice.
\end{itemize}

Il paradigma funzionale ha diversi vantaggi:

\begin{itemize}
\tightlist
\item
  Prevedibilità e facilità di test: Le funzioni pure e l'immutabilità
  rendono il codice più prevedibile e più facile da testare, poiché non
  ci sono stati mutabili o effetti collaterali nascosti.
\item
  Concorrenza: La programmazione funzionale è ben adatta alla
  programmazione concorrente e parallela, poiché l'assenza di stato
  mutabile riduce i problemi di sincronizzazione e competizione per le
  risorse.
\item
  Modularità e riutilizzabilità: La composizione di funzioni e la
  trasparenza referenziale facilitano la creazione di codice modulare e
  riutilizzabile.
\end{itemize}

E qualche svantaggio:

\begin{itemize}
\tightlist
\item
  Curva di Apprendimento: La programmazione funzionale può essere
  difficile da apprendere per chi proviene da paradigmi imperativi o
  orientati agli oggetti, a causa dei concetti matematici sottostanti e
  della diversa mentalità necessaria.
\item
  Prestazioni: In alcuni casi, l'uso intensivo di funzioni ricorsive può
  portare a problemi di prestazioni, come il consumo di memoria per le
  chiamate ricorsive. Tuttavia, molte implementazioni moderne offrono
  ottimizzazioni come la ricorsione di coda.
\item
  Disponibilità di Librerie e Strumenti: Alcuni linguaggi funzionali
  potrebbero non avere la stessa ampiezza di librerie e strumenti
  disponibili rispetto ai linguaggi imperativi più diffusi.
\end{itemize}

\subsection{Esempio in Haskell}\label{esempio-in-haskell}

Di seguito due funzioni, la prima \texttt{sumToN} è pura e somma i primi
\texttt{n} numeri. \texttt{(*2)} è una funzione che prende un argomento
e lo moltiplica per 2 e ciò rende la seconda funzione
\texttt{applyFunction} una vera funzione di ordine superiore, poiché
accetta \texttt{(*2)} come argomento oltre ad una lista, producendo come
risultato il raddoppio di tutti i suoi elementi.

\begin{Shaded}
\begin{Highlighting}[]
\CommentTok{{-}{-} Definizione di una funzione pura che calcola la somma dei numeri da 1 a n}
\OtherTok{sumToN ::} \DataTypeTok{Integer} \OtherTok{{-}\textgreater{}} \DataTypeTok{Integer}
\NormalTok{sumToN n }\OtherTok{=} \FunctionTok{sum}\NormalTok{ [}\DecValTok{1}\OperatorTok{..}\NormalTok{n]}

\CommentTok{{-}{-} Funzione di ordine superiore che accetta una funzione e una lista}
\OtherTok{applyFunction ::}\NormalTok{ (a }\OtherTok{{-}\textgreater{}}\NormalTok{ b) }\OtherTok{{-}\textgreater{}}\NormalTok{ [a] }\OtherTok{{-}\textgreater{}}\NormalTok{ [b]}
\NormalTok{applyFunction f lst }\OtherTok{=} \FunctionTok{map}\NormalTok{ f lst}

\CommentTok{{-}{-} Utilizzo delle funzioni}
\NormalTok{main }\OtherTok{=} \KeywordTok{do}
    \FunctionTok{print}\NormalTok{ (sumToN }\DecValTok{10}\NormalTok{) }\CommentTok{{-}{-} Stampa 55}
    \FunctionTok{print}\NormalTok{ (applyFunction (}\OperatorTok{*}\DecValTok{2}\NormalTok{) [}\DecValTok{1}\NormalTok{, }\DecValTok{2}\NormalTok{, }\DecValTok{3}\NormalTok{, }\DecValTok{4}\NormalTok{]) }\CommentTok{{-}{-} Stampa [2, 4, 6, 8]}
\end{Highlighting}
\end{Shaded}

\subsection{Linguaggi}\label{linguaggi-1}

Oltre a Haskell, ci sono molti altri linguaggi funzionali, tra cui:

\begin{itemize}
\tightlist
\item
  Erlang: Utilizzato per sistemi concorrenti e distribuiti.
\item
  Elixir: Costruito a partire da Erlang, è utilizzato per applicazioni
  web scalabili.
\item
  F\#: Parte della piattaforma .NET, combina la programmazione
  funzionale con lo OOP.
\item
  Scala: Anch'esso combina programmazione funzionale e orientata agli
  oggetti ed è interoperabile con Java.
\item
  OCaml: Conosciuto per le sue prestazioni e sintassi espressiva.
\item
  Lisp: Uno dei linguaggi più antichi, multi-paradigma con forti
  influenze funzionali.
\item
  Clojure: Dialetto di Lisp per la JVM, adatto alla concorrenza.
\item
  Scheme: Dialetto di Lisp spesso usato nell'educazione.
\item
  ML: Linguaggio influente che ha portato allo sviluppo di OCaml e F\#.
\item
  Racket: Derivato da Scheme, usato nella ricerca accademica.
\end{itemize}

\chapter{Sintassi e semantica dei linguaggi di programmazione e
algoritmi dei
programmi}\label{sintassi-e-semantica-dei-linguaggi-di-programmazione-e-algoritmi-dei-programmi}

\section{Sintassi e semantica dei linguaggi di
programmazione}\label{sintassi-e-semantica-dei-linguaggi-di-programmazione}

I linguaggi di programmazione sono strumenti utilizzati per implementare
algoritmi in modo che possano essere eseguiti da un computer. Un
linguaggio di programmazione ha due componenti principali: la
\textbf{sintassi} e la \textbf{semantica}.

\section{Sintassi}\label{sintassi}

La \textbf{sintassi} di un linguaggio di programmazione è l'insieme di
regole che definiscono come devono essere scritte le istruzioni, cioè le
unità logiche di esecuzione del programma. È come la grammatica in una
lingua naturale e stabilisce quali combinazioni di simboli sono
considerate costrutti validi nel linguaggio.

Partiamo dagli elementi atomici della sintassi per poi risalire fino al
programma. Gli elementi sono:

\begin{itemize}
\tightlist
\item
  Parole chiave: Sono termini riservati del linguaggio che hanno
  significati specifici e non possono essere utilizzati per altri scopi,
  come \texttt{if}, \texttt{else}, \texttt{while}, \texttt{for}, ecc.
\item
  Operatori: Simboli utilizzati per eseguire operazioni su
  identificatori e letterali, come \texttt{+}, \texttt{-}, \texttt{*},
  \texttt{/}, \texttt{=}, \texttt{==}, ecc.
\item
  Separatori: Caratteri utilizzati per separare elementi del codice,
  come punto e virgola (\texttt{;}), parentesi tonde (\texttt{()}),
  parentesi quadre (\texttt{{[}{]}}), parentesi graffe (\texttt{\{\}}),
  ecc.
\item
  Identificatori: Nomi utilizzati per identificare variabili, funzioni,
  classi, e altri oggetti.
\item
  Letterali: Rappresentazioni di valori costanti nel codice, come numeri
  (\texttt{123}), stringhe (\texttt{"hello"}), caratteri
  (\texttt{\textquotesingle{}a\textquotesingle{}}), ecc.
\item
  Commento: Non fanno parte della logica del programma e sono ignorati
  nell'esecuzione.
\item
  Spazi: Sono gruppi di caratteri non visualizzabili spesso ignorati.
\end{itemize}

Un \textbf{lessema} è una sequenza di caratteri nel programma sorgente
che corrisponde al pattern di un token ed è identificata
dall'analizzatore lessicale come un'istanza di quel token. Un
\textbf{token} è una coppia composta da un nome di token e un valore
attributo opzionale. Il nome del token è un simbolo astratto che
rappresenta un tipo di unità lessicale, ad esempio, una particolare
parola chiave o una sequenza di caratteri di input che denota un
identificatore. I nomi dei token sono i simboli di input che il parser
elabora. Un \textbf{pattern} è una descrizione della forma che possono
assumere i lessemi di un token. Nel caso di una parola chiave come
token, il pattern è semplicemente la sequenza di caratteri che forma la
parola chiave. Per gli identificatori e alcuni altri token, il pattern è
una struttura più complessa che corrisponde a molte stringhe.

\subsection{Espressioni}\label{espressioni}

Un'espressione è una combinazione di lessemi che viene valutata per
produrre un risultato. Esempi di espressioni includono:

\begin{itemize}
\tightlist
\item
  \texttt{5\ +\ 3}
\item
  \texttt{x\ *\ 2}
\item
  \texttt{y\ /\ 4.0}
\item
  \texttt{max(a,\ b)}
\item
  \texttt{"Hello,\ "\ +\ "world!"}
\end{itemize}

\subsection{Istruzioni semplici}\label{istruzioni-semplici}

Le \textbf{istruzioni semplici} sono operazioni atomiche secondo il
linguaggio e sono costituite da lessemi ed espressioni per compiere
operazioni di base. Gli esempi principali includono:

\begin{itemize}
\item
  Assegnazione: Utilizza un operatore di assegnazione (ad esempio,
  \texttt{=}) per attribuire un valore a una variabile, che possiamo
  pensare come un nome simbolico rappresentante una posizione dove è
  memorizzato un valore. Esempio:

\begin{Shaded}
\begin{Highlighting}[]
\NormalTok{x }\OperatorTok{=} \DecValTok{5}
\end{Highlighting}
\end{Shaded}

  \texttt{x}: Identificatore della variabile. \texttt{=}: Operatore di
  assegnazione. \texttt{5}: Letterale numerico intero.
\item
  Input/output: Utilizza parole chiave o funzioni di libreria per
  leggere valori dall'input o scrivere valori all'output. Esempio:

\begin{Shaded}
\begin{Highlighting}[]
\BuiltInTok{print}\NormalTok{(}\StringTok{"Hello, World!"}\NormalTok{)}
\end{Highlighting}
\end{Shaded}

  \texttt{print}: Parola chiave o identificatore di funzione di
  libreria. \texttt{"Hello,\ World!"}: Letterale stringa. L'esecuzione
  dell'istruzione produce \texttt{"Hello,\ World!"} in output.
\item
  Assegnazione ad espressione: Combinazione di variabili, operatori e
  valori che producono un risultato assegnato ad una variabile. Esempio:
  \texttt{python\ \ \ \ \ z\ =\ (x\ *\ 2)\ +\ (y\ /\ 2)} \texttt{z}:
  Identificatore della variabile. \texttt{=}: Operatore di assegnazione.
  \texttt{(x\ *\ 2)}: Espressione che moltiplica \texttt{x} per
  \texttt{2}. \texttt{(y\ /\ 2)}: Espressione che divide \texttt{y} per
  \texttt{2}. \texttt{+}: Operatore aritmetico che somma i risultati
  delle due espressioni in una più complessa. L'esecuzione
  dell'istruzione produce un risultato valido solo se \texttt{x} e
  \texttt{y} sono associate a valori numerici e ciò perché non tutte le
  istruzioni sintatticamente corrette sono semanticamente corrette.
  D'altronde ciò non deve essere preso come regola, perché se \texttt{*}
  fosse un operatore che ripete quanto a sinistra un numero di volte
  definito dal valore di destra e \texttt{/} la divisione del valore di
  sinistra in parti di numero pari a quanto a destra, allora \texttt{x}
  e \texttt{y} potrebbero essere stringhe.
\end{itemize}

\subsection{Istruzioni complesse e blocchi di
codice}\label{istruzioni-complesse-e-blocchi-di-codice}

Le \textbf{istruzioni complesse} sono costituite da più istruzioni
semplici e possono includere strutture di controllo del flusso, come
condizioni (\texttt{if}), cicli (\texttt{for}, \texttt{while}) ed
eccezioni (\texttt{try}, \texttt{catch}). Queste istruzioni sono
utilizzate per organizzare il flusso di esecuzione del programma e
possono contenere altre istruzioni semplici o complesse al loro interno.

Un \textbf{blocco di codice} è una sezione del codice che raggruppa una
serie di istruzioni che devono essere eseguite insieme. I blocchi di
codice sono spesso utilizzati all'interno delle istruzioni complesse per
delimitare il gruppo di istruzioni che devono essere eseguite in
determinate condizioni o iterazioni.

In molti linguaggi di programmazione, i blocchi di codice sono
delimitati da parentesi graffe (\texttt{\{\}}), mentre in altri
linguaggi, come Python, l'indentazione è utilizzata per indicare
l'inizio e la fine di un blocco di codice.

Alcuni esempi di istruzione e blocco di codice:

\begin{itemize}
\item
  Esempio in C:

\begin{Shaded}
\begin{Highlighting}[]
\ControlFlowTok{if} \OperatorTok{(}\NormalTok{x }\OperatorTok{\textgreater{}} \DecValTok{0}\OperatorTok{)} \OperatorTok{\{}
\NormalTok{    printf}\OperatorTok{(}\StringTok{"x è positivo}\SpecialCharTok{\textbackslash{}n}\StringTok{"}\OperatorTok{);}
\NormalTok{    y }\OperatorTok{=}\NormalTok{ x }\OperatorTok{*} \DecValTok{2}\OperatorTok{;}
\OperatorTok{\}}
\end{Highlighting}
\end{Shaded}

  In questo esempio:

  \begin{itemize}
  \tightlist
  \item
    \texttt{if\ (x\ \textgreater{}\ 0)} è un'istruzione complessa.
  \item
    \texttt{\{\ printf("x\ è\ positivo\textbackslash{}n");\ y\ =\ x\ *\ 2;\ \}}
    è un blocco di codice che viene eseguito se la condizione
    dell'istruzione \texttt{if} è vera.
  \end{itemize}
\item
  Esempio in Python:
\end{itemize}

\begin{Shaded}
\begin{Highlighting}[]
  \ControlFlowTok{if}\NormalTok{ x }\OperatorTok{\textgreater{}} \DecValTok{0}\NormalTok{:}
      \BuiltInTok{print}\NormalTok{(}\StringTok{"x è positivo"}\NormalTok{)}
\NormalTok{      y }\OperatorTok{=}\NormalTok{ x }\OperatorTok{*} \DecValTok{2}
\end{Highlighting}
\end{Shaded}

In questo esempio: - \texttt{if\ x\ \textgreater{}\ 0:} è un'istruzione
complessa. - Le righe indentate sotto l'istruzione \texttt{if}
(\texttt{print("x\ è\ positivo")} e \texttt{y\ =\ x\ *\ 2})
costituiscono un blocco di codice che viene eseguito se la condizione
dell'istruzione \texttt{if} è vera.

Altri esempi:

\begin{itemize}
\item
  Condizioni: Istruzioni che eseguono un blocco di codice solo se una
  condizione è vera. Esempio: ```c \#include
  \textless stdio.h\textgreater{}

  x = 42;

  if (x \textgreater{} 0) \{ printf(``x è positivo\n''); \}
  ``\texttt{}if\texttt{:\ Parola\ chiave\ che\ introduce\ la\ condizione.}(x
  \textgreater{}
  0)\texttt{:\ Condizione\ composta\ da:}x\texttt{identificatore\ di\ variabile,}\textgreater{}\texttt{operatore\ di\ confronto\ e}0\texttt{letterale\ numerico\ intero.}\{
  \ldots{}
  \}\texttt{:\ Delimitatori\ che\ racchiudono\ il\ blocco\ di\ codice.}print(``x
  è positivo'')`: Istruzione di output.
\item
  Cicli: Istruzioni che ripetono un blocco di codice. Esempio: ```c
  \#include \textless stdio.h\textgreater{}

  int n = 42; int somma = 0; int i;

  for (i = 0; i \textless{} n; i++) \{ somma = somma + i; \}
  ``\texttt{}for\texttt{:\ Parola\ chiave\ che\ introduce\ il\ ciclo.}(i
  = 0; i \textless{} n;
  i++)\texttt{:\ Espressione\ di\ controllo\ del\ ciclo\ composta\ da:}i
  = 0\texttt{assegnazione\ iniziale,}i \textless{}
  n\texttt{condizione\ di\ ciclo\ e}i++\texttt{incremento\ della\ variabile}i\texttt{.}\{
  \ldots{}
  \}\texttt{:\ Delimitatori\ che\ racchiudono\ il\ blocco\ di\ codice.}somma
  = somma + i`: Operazione aritmetica.
\end{itemize}

\subsection{Organizzazione del codice in un
programma}\label{organizzazione-del-codice-in-un-programma}

Il programma è solitamente salvato in un file di testo in righe. Queste
righe possono essere classificate in righe fisiche e righe logiche.

Una \textbf{riga fisica} è una linea di testo nel file sorgente del
programma, terminata da un carattere di a capo.

Esempio:

\begin{Shaded}
\begin{Highlighting}[]
\DataTypeTok{int}\NormalTok{ x }\OperatorTok{=} \DecValTok{10}\OperatorTok{;} \CommentTok{// Questa è una riga fisica}
\end{Highlighting}
\end{Shaded}

Una \textbf{riga logica} è una singola istruzione, che può estendersi su
una o più righe fisiche.

Esempi: - Una riga fisica terminata è una riga logica. - Esempio su più
righe fisiche:

\begin{Shaded}
\begin{Highlighting}[]
\DataTypeTok{int}\NormalTok{ y }\OperatorTok{=} \OperatorTok{(}\DecValTok{10} \OperatorTok{+} \DecValTok{20} \OperatorTok{+} \DecValTok{30} \OperatorTok{+} \CommentTok{// Prima riga fisica della riga logica}
         \DecValTok{40} \OperatorTok{+} \DecValTok{50}\OperatorTok{);} \CommentTok{// Seconda riga fisica della riga logica}
\end{Highlighting}
\end{Shaded}

L'interprete o il compilatore hanno un modulo particolare che si occupa
della scansione dei file costituenti il programma, con l'obiettivo di
individuare righe e lessemi, quindi le istruzioni. A seguire,
l'interprete eseguirà le istruzioni, il compilatore le tradurrà in
linguaggio macchina.

\section{Semantica}\label{semantica}

La \textbf{semantica} di un linguaggio di programmazione definisce il
significato delle istruzioni sintatticamente corrette. In altre parole,
la semantica specifica cosa fa un programma quando viene eseguito,
descrivendo l'effetto delle istruzioni sullo stato del sistema. Gli
elementi semantici sono numerosi, possono essere anche molto complessi e
non tutti presenti in uno specifico linguaggio. Di seguito ne sono
elencati alcuni tra i più diffusi:

\begin{itemize}
\tightlist
\item
  Variabile: È un nome simbolico associato a locazione di memoria che
  può contenere uno o più valori. È fondamentale per la manipolazione di
  dati perché sono un mezzo per astrarre dalla costante memorizzata. Le
  variabili possono essere associate a diversi tipi di dati e durate di
  vita. La semantica delle variabili include la loro dichiarazione,
  inizializzazione, uso e visibilità:

  \begin{itemize}
  \tightlist
  \item
    Dichiarazione: La dichiarazione di una variabile è il processo
    mediante il quale si introduce una variabile nel programma,
    specificandone il nome e, in molti casi, il tipo di dato che essa
    può contenere. La dichiarazione informa il compilatore o
    l'interprete che una certa variabile esiste e può essere utilizzata
    nel codice.
  \item
    Inizializzazione: L'inizializzazione di una variabile è il processo
    di assegnare un valore iniziale alla variabile. L'inizializzazione
    può avvenire contestualmente alla dichiarazione o in un'istruzione
    separata successiva.
  \item
    Visibilità: Indica dove la variabile può essere utilizzata
    all'interno del codice (ad esempio, variabili locali o globali).
  \item
    Durata di Vita: Descrive per quanto tempo la variabile rimane in
    memoria durante l'esecuzione del programma (ad esempio, automatica,
    statica, dinamica).
  \end{itemize}
\item
  Tipo di dati: I tipi di dati definiscono il dominio dei valori che una
  variabile può assumere e le operazioni che possono essere eseguite su
  quei valori. Un tipo di dato determina la natura del valore (ad
  esempio, numero intero, carattere, booleano) e le operazioni che
  possono essere effettuate su di esso. Generalmente si distinguono in:

  \begin{itemize}
  \tightlist
  \item
    Tipo primitivo: I tipi di dati fondamentali forniti da un
    linguaggio, come integer, float, boolean e character.
  \item
    Tipo complesso: Tipo di dati costituiti da più tipi primitivi, come
    array, struct e oggetti.
  \item
    Tipo di dati utente: Tipo definito dall'utente, come classi e tipi
    personalizzati, che permette di creare strutture dati più complesse
    e specifiche per il problema da risolvere.
  \end{itemize}
\item
  Ambito (in inglese, scope): L'ambito rappresenta la porzione del
  codice in cui un identificatore (come una variabile o una funzione) è
  definito e, quindi, esiste. L'ambito determina dove un identificatore
  può essere dichiarato e utilizzato. Tipicamente gli ambiti sono:

  \begin{itemize}
  \tightlist
  \item
    Globale: Identificatori dichiarati a livello globale, accessibili
    ovunque nel programma.
  \item
    Locale: Identificatori dichiarati all'interno di un blocco, come una
    funzione o un loop, e accessibili solo all'interno di quel blocco.
  \item
    Statico e dinamico: L'ambito statico è determinato a tempo di
    compilazione, mentre l'ambito dinamico è determinato a runtime,
    influenzando come e dove gli identificatori possono essere
    utilizzati.
  \end{itemize}
\item
  Visibilità: La visibilità si riferisce a dove nel codice un
  identificatore può essere visto e utilizzato. Anche se correlata
  all'ambito, la visibilità può essere influenzata da altri fattori come
  la modularità e i namespace, che organizzano e separano gli
  identificatori per evitare conflitti di nome. La visibilità è
  generalmente:

  \begin{itemize}
  \tightlist
  \item
    Globale: Un identificatore dichiarato con visibilità globale può
    essere utilizzato in qualsiasi parte del programma.
  \item
    Locale: Un identificatore dichiarato con visibilità locale è
    visibile solo all'interno del blocco di codice in cui è stato
    dichiarato.
  \end{itemize}
\item
  Funzioni e metodi: Le funzioni e i metodi sono blocchi di codice
  riutilizzabili che eseguono una serie di istruzioni. Alcuni concetti
  collegati sono:

  \begin{itemize}
  \tightlist
  \item
    Parametri e argomenti: Valori passati alle funzioni per influenzarne
    il comportamento. I parametri sono definiti nella dichiarazione
    della funzione, mentre gli argomenti sono i valori effettivi passati
    quando la funzione è chiamata.
  \item
    Valore di ritorno: Il risultato prodotto da una funzione, che può
    essere utilizzato nell'istruzione chiamante.
  \item
    Overloading: Definizione di più funzioni con lo stesso nome ma
    diversi parametri, consentendo diverse implementazioni basate sui
    tipi e il numero di argomenti.
  \item
    Ricorsione: Capacità di una funzione di chiamare se stessa, utile
    per risolvere problemi che possono essere suddivisi in sottoproblemi
    simili.
  \item
    Funzioni di prima classe: Le funzioni possono essere assegnate a
    variabili, passate come argomenti e ritornate da altre funzioni.
  \item
    Funzioni di ordine superiore: Funzioni che accettano altre funzioni
    come argomenti e/o ritornano funzioni come risultati.
  \end{itemize}
\item
  Durata di vita delle variabili: La durata di vita delle variabili si
  riferisce a quanto tempo una variabile rimane in memoria durante
  l'esecuzione del programma. Alcune tipologie di durata:

  \begin{itemize}
  \tightlist
  \item
    Automatica: Variabili che esistono solo durante l'esecuzione del
    blocco in cui sono dichiarate.
  \item
    Statica: Variabili che esistono per tutta la durata del programma e
    mantengono il loro valore tra diverse chiamate di funzione.
  \item
    Dinamica: Variabili allocate dinamicamente durante l'esecuzione del
    programma, solitamente gestite manualmente dall'utente (ad esempio,
    usando \texttt{malloc}/\texttt{free} in C) o automaticamente tramite
    garbage collection.
  \end{itemize}
\item
  Durata di vita di altri identificatori:

  \begin{itemize}
  \tightlist
  \item
    Funzioni: Le funzioni stesse generalmente hanno una durata di vita
    che coincide con la durata del programma. Tuttavia, i puntatori a
    funzione e le chiusure (in inglese, closures) possono avere durate
    di vita diverse in alcuni linguaggi.
  \item
    Classi e oggetti: Le classi hanno una durata di vita che coincide
    con la durata del programma, mentre gli oggetti (istanze di classi)
    hanno durate di vita dinamiche, determinate dalla loro allocazione e
    deallocazione.
  \item
    Moduli: In linguaggi come Python, i moduli hanno una durata di vita
    che coincide con la durata del programma o del processo di
    importazione.
  \end{itemize}
\item
  Controllo di flusso: Determina l'ordine in cui le istruzioni vengono
  eseguite e alcuni esempi sono:

  \begin{itemize}
  \tightlist
  \item
    Condizionali: Strutture che permettono al programma di prendere
    decisioni (\texttt{if}, \texttt{else},
    \texttt{switch}/\texttt{case}).
  \item
    Cicli: Strutture che ripetono un blocco di codice (\texttt{for},
    \texttt{while}, \texttt{do}-\texttt{while}).
  \item
    Eccezioni: Meccanismi per gestire errori e condizioni anomale
    (\texttt{try}, \texttt{catch}, \texttt{throw}), permettendo al
    programma di continuare l'esecuzione in modo controllato.
  \end{itemize}
\item
  Classi e oggetti: Le classi sono trutture che definiscono proprietà
  (variabili) e comportamenti (metodi) comuni a tutti gli oggetti di
  quel tipo. Le classi rappresentano il modello o il blueprint da cui
  vengono creati gli oggetti. L'oggetto è l'istanza concreta di una
  classe. Gli oggetti sono entità che combinano dati e comportamenti
  secondo la struttura definita dalla loro classe. Si applicano i
  seguenti:

  \begin{itemize}
  \tightlist
  \item
    Encapsulamento: Nasconde i dettagli interni di un oggetto e mostra
    solo le interfacce necessarie, migliorando la modularità e la
    manutenzione del codice.
  \item
    Ereditarietà: Permette di creare nuove classi basate su classi
    esistenti, riutilizzando e estendendo il comportamento delle classi
    base.
  \item
    Polimorfismo: Consente a metodi di comportarsi diversamente a
    seconda dell'oggetto su cui vengono invocati, fornendo flessibilità
    e estendibilità.
  \end{itemize}
\item
  Gestione della memoria utilizzata dal programma: La gestione della
  memoria è fondamentale per il funzionamento efficiente di un
  programma. Ne esistono diverse modalità:

  \begin{itemize}
  \tightlist
  \item
    Allocazione dinamica: La memoria è allocata e deallocata a runtime,
    permettendo una gestione flessibile delle risorse.
  \item
    Garbage collection: Automatizza la deallocazione della memoria non
    utilizzata, riducendo il rischio di sfruttamento non ottimale
    (memory leak) e semplificando la gestione della memoria.
  \end{itemize}
\item
  Spazio di nomi (in inglese namespace): I namespace organizzano
  variabili, funzioni e altri identificatori per evitare conflitti di
  nome.
\item
  Moduli e librerie: I moduli e le librerie suddividono il codice in
  unità riutilizzabili e organizzate, da importare in programmi. I
  moduli possono definire degli spazi di nomi.
\item
  Concorrenza: La concorrenza permette l'esecuzione parallela di più
  sequenze di istruzioni, migliorando le prestazioni e la reattività.
  Alcuni concetti relativi sono:

  \begin{itemize}
  \tightlist
  \item
    Thread: Un thread è la più piccola unità di elaborazione che può
    essere eseguita in modo indipendente. I thread consentono
    l'esecuzione parallela di codice all'interno di un programma.
  \item
    Sincronizzazione: Meccanismi per gestire l'accesso concorrente alle
    risorse condivise, prevenendo condizioni di gara e garantendo la
    consistenza dei dati.
  \item
    Lock e mutex: Meccanismi per prevenire condizioni di corsa (in
    inglese race condition), cioè un fenomeno che si presenta nei
    sistemi concorrenti quando, in presenza di una sequenza di processi
    multipli, il risultato finale dell'esecuzione dei processi dipende
    dalla temporizzazione o dalla sequenza con cui vengono eseguiti.
    Garantiscono, pertanto, l'accesso sicuro alle risorse condivise.
  \item
    Async/await: Gestione di operazioni asincrone, migliorando
    l'efficienza e la reattività delle applicazioni.
  \end{itemize}
\item
  Input/output (I/O): L'input/output gestisce la comunicazione tra il
  programma e l'ambiente esterno.

  \begin{itemize}
  \tightlist
  \item
    File I/O: Lettura e scrittura su file per memorizzare e recuperare
    dati persistenti.
  \item
    Network I/O: Comunicazione attraverso reti per inviare e ricevere
    dati tra sistemi diversi.
  \item
    Standard I/O: Interazione con l'utente tramite input da tastiera e
    output su schermo.
  \end{itemize}
\item
  Annotazioni e metadati: Le annotazioni e i metadati forniscono
  informazioni aggiuntive al compilatore o al runtime, influenzando il
  comportamento del programma o fornendo dettagli utili per la
  documentazione e l'analisi del codice.

  \begin{itemize}
  \tightlist
  \item
    Annotazioni: Informazioni extra utilizzate per specificare
    comportamenti speciali o configurazioni. Ad esempio, in Java, le
    annotazioni possono essere utilizzate per indicare che un metodo è
    obsoleto (\texttt{@Deprecated}), per sovrascrivere un metodo della
    superclasse (\texttt{@Override}), o per specificare la relazione tra
    entità nel contesto di framework come JPA (\texttt{@Entity},
    \texttt{@Table}). In Python, le annotazioni sono utilizzate
    principalmente per indicare i tipi di variabili, parametri di
    funzione e valori di ritorno (type hint). Non influenzano
    direttamente il comportamento del programma, ma sono utili per la
    documentazione e il type checking anche automatico.
  \item
    Docstring: Commenti strutturati che documentano il codice, spesso
    utilizzati per generare documentazione automatica. In Python, ad
    esempio, le docstring possono essere utilizzate per descrivere il
    funzionamento di moduli, classi, metodi e funzioni, rendendo il
    codice più leggibile e comprensibile.
  \end{itemize}
\item
  Macro e metaprogrammazione: Le macro e la metaprogrammazione
  permettono di scrivere codice che manipola altre porzioni di codice.

  \begin{itemize}
  \tightlist
  \item
    Macro: Sequenze di istruzioni predefinite che possono essere
    inserite nel codice durante la fase di precompilazione. In C, sono
    utilizzate con il preprocessore per definire costanti, funzioni
    inline e codice condizionale.
  \item
    Metaprogrammazione: Tecniche per scrivere codice che genera o
    modifica altre parti del codice a runtime o a compile-time,
    migliorando la flessibilità e il riutilizzo del codice. In Python
    include l'uso di decoratori, metaclassi e introspezione.
  \end{itemize}
\end{itemize}

Questi elementi semantici combinati determinano il comportamento e la
logica di un programma, influenzando il modo in cui il codice viene
scritto, eseguito e mantenuto.

\section{Algoritmi}\label{algoritmi}

Una definizione informale di \textbf{algoritmo} è un insieme di
istruzioni precise e finite che descrivono come eseguire un compito
specifico o risolvere un problema. Pensate a un algoritmo come a una
ricetta di cucina: la ricetta fornisce una serie di passaggi chiari da
seguire per preparare un piatto.

Formalmente, un algoritmo è una sequenza ben definita di passi o
operazioni che, a partire da un input, produce un output in un tempo
finito. Le proprietà principali di un algoritmo includono:

\begin{enumerate}
\def\labelenumi{\arabic{enumi}.}
\tightlist
\item
  Finitudine L'algoritmo deve terminare dopo un numero finito di passi.
\item
  Determinismo: Ogni passo dell'algoritmo deve essere definito in modo
  preciso e non ambiguo.
\item
  Input L'algoritmo riceve zero o più dati in ingresso.
\item
  Output L'algoritmo produce uno o più risultati.
\item
  Effettività: Ogni operazione dell'algoritmo deve essere fattibile ed
  eseguibile in un tempo finito.
\end{enumerate}

\subsection{Esempio}\label{esempio}

Consideriamo un esempio semplice di algoritmo per calcolare la somma dei
numeri da 1 a n.~In input si avrà un numero intero \texttt{n} e in
output il risultato. In pseudocodice si può rappresentare così:

\begin{Shaded}
\begin{Highlighting}[]
\NormalTok{Inizializza somma a 0}
\NormalTok{Per ogni numero i da 1 a n:}
\NormalTok{  Aggiungi i a somma}
\NormalTok{Restituisci somma}
\end{Highlighting}
\end{Shaded}

O, in alternativa, possiamo definire una funzione che implementa
l'algoritmo:

\begin{Shaded}
\begin{Highlighting}[]
\NormalTok{funzione calcola\_somma(n):}
\NormalTok{  somma = 0}
\NormalTok{  per i da 1 a n:}
\NormalTok{    somma = somma + i}
\NormalTok{  ritorna somma}
\end{Highlighting}
\end{Shaded}

Effettuaimo una analisi dettagliata della funzione per indicare quali
elementi sintattici e semantici sono presenti. Partiamo dalla prima
riga:

\begin{Shaded}
\begin{Highlighting}[]
\NormalTok{funzione calcola\_somma(n):}
\end{Highlighting}
\end{Shaded}

\begin{itemize}
\tightlist
\item
  Sintassi:

  \begin{itemize}
  \tightlist
  \item
    \texttt{funzione}: Parola chiave che introduce la definizione di una
    funzione.
  \item
    \texttt{calcola\_somma}: Identificatore della funzione.
  \item
    \texttt{(n)}: Delimitatori che contengono un identificatore.
  \end{itemize}
\item
  Semantica:

  \begin{itemize}
  \tightlist
  \item
    Definisce una funzione chiamata \texttt{calcola\_somma} che accetta
    un parametro \texttt{n}.
  \end{itemize}
\end{itemize}

La seconda riga ha l'inizializzazione della variabile che conterrà il
risultato:

\begin{Shaded}
\begin{Highlighting}[]
\NormalTok{somma = 0}
\end{Highlighting}
\end{Shaded}

\begin{itemize}
\tightlist
\item
  Sintassi:

  \begin{itemize}
  \tightlist
  \item
    \texttt{somma}: Identificatore della variabile.
  \item
    \texttt{=}: Operatore di assegnazione.
  \item
    \texttt{0}: Letterale numero intero.
  \end{itemize}
\item
  Semantica:

  \begin{itemize}
  \tightlist
  \item
    Inizializza la variabile \texttt{somma} a \texttt{0}.
  \end{itemize}
\end{itemize}

A seguire la definizione di un ciclo:

\begin{Shaded}
\begin{Highlighting}[]
\NormalTok{per i da 1 a n:}
\end{Highlighting}
\end{Shaded}

\begin{itemize}
\tightlist
\item
  Sintassi:

  \begin{itemize}
  \tightlist
  \item
    \texttt{per}: Parola chiave che introduce il ciclo.
  \item
    \texttt{i}: Identificatore della variabile di controllo del ciclo.
  \item
    \texttt{da\ 1\ a\ n}: Espressione di controllo del ciclo che indica
    l'intervallo.
  \end{itemize}
\item
  Semantica:

  \begin{itemize}
  \tightlist
  \item
    Itera la variabile \texttt{i} da \texttt{1} a \texttt{n}.
  \end{itemize}
\end{itemize}

Un assegnamento per accumulare i valori nella variabile di ritorno:

\begin{Shaded}
\begin{Highlighting}[]
\NormalTok{somma = somma + i}
\end{Highlighting}
\end{Shaded}

\begin{itemize}
\tightlist
\item
  Sintassi:

  \begin{itemize}
  \tightlist
  \item
    \texttt{somma}: Identificatore della variabile.
  \item
    \texttt{=}: Operatore di assegnazione.
  \item
    \texttt{somma\ +\ i}: Espressione aritmetica composta da:
    \texttt{somma} identificatore di variabile, \texttt{+} operatore
    aritmetico e \texttt{i} identificatore di variabile.
  \end{itemize}
\item
  Semantica:

  \begin{itemize}
  \tightlist
  \item
    Aggiunge il valore di \texttt{i} alla variabile \texttt{somma} e
    assegna il risultato a \texttt{somma}.
  \end{itemize}
\end{itemize}

E finalmente il risultato del calcolo viene restituito al chiamante:

\begin{Shaded}
\begin{Highlighting}[]
\NormalTok{ritorna somma}
\end{Highlighting}
\end{Shaded}

\begin{itemize}
\tightlist
\item
  Sintassi:

  \begin{itemize}
  \tightlist
  \item
    \texttt{ritorna}: Parola chiave che indica la restituzione di un
    valore.
  \item
    \texttt{somma}: Identificatore della variabile.
  \end{itemize}
\item
  Semantica:

  \begin{itemize}
  \tightlist
  \item
    Restituisce il valore della variabile \texttt{somma} come risultato
    della funzione.
  \end{itemize}
\end{itemize}

Abbiamo così dissezionato un algoritmo molto semplice per illustrare
come sintassi e semantica di un linguaggio abbiano ruoli distinti e
complementari in un programma. È importante comprendere che un buon
programmatore deve avere tutte e tre le competenze, cioè conoscere le
specificità formali del linguaggio (o di più linguaggi), quindi, la sua
sintassi e semantica e saper comporre algoritmi, che potrà realizzare
grazie proprio a quelle.

\part{Seconda parte: Le basi di Python}

\chapter{Introduzione a Python}\label{introduzione-a-python}

Python è un linguaggio di programmazione multiparadigma rilasciato da
Guido van Rossum nel 1991, quindi dopo il C++ e prima di Java e PHP. È
multiparadigma, cioè abilita o supporta più paradigmi di programmazione,
e multipiattaforma, potendo essere installato e utilizzato su gran parte
dei sistemi operativi e hardware.

Python offre una combinazione unica di eleganza, semplicità, praticità e
versatilità. Questa eleganza e semplicità derivano dal fatto che è stato
progettato per essere molto simile al linguaggio naturale inglese,
rendendo il codice leggibile e comprensibile. La sintassi di Python è
pulita e minimalista, evitando simboli superflui come parentesi graffe e
punti e virgola, e utilizzando indentazioni per definire blocchi di
codice, il che forza una struttura coerente e leggibile. La semantica
del linguaggio è intuitiva e coerente, il che riduce la curva di
apprendimento e minimizza gli errori.

Diventerai rapidamente produttivo con Python grazie alla sua coerenza e
regolarità, alla sua ricca libreria standard e ai numerosi pacchetti e
strumenti di terze parti prontamente disponibili. Python è facile da
imparare, quindi è molto adatto se sei nuovo alla programmazione, ma è
anche potente abbastanza per i più sofisticati esperti. Questa
semplicità ha attratto una comunità ampia e attiva che ha contribuito
sia alle librerie di programmi incluse nell'implementazione ufficiale
che a molte librerie scaricabili liberamente, ampliando ulteriormente
l'ecosistema di Python.

\section{Perché Python è un linguaggio di alto
livello?}\label{perchuxe9-python-uxe8-un-linguaggio-di-alto-livello}

Python è considerato un linguaggio di programmazione di alto livello,
cioè utilizza un livello di astrazione elevato rispetto alla complessità
dell'ambiente in cui i suoi programmi sono eseguiti. Il programmatore ha
a disposizione una sintassi che è più intuitiva rispetto ad altri
linguaggi come Java, C++, PHP tradizionalmente anch'essi definiti di
alto livello.

Infatti, consente ai programmatori di scrivere codice in modo più
concettuale e indipendente dalle caratteristiche degli hardware, anche
molto diversi, su cui è disponibile. Ad esempio, invece di preoccuparsi
di allocare e deallocare memoria manualmente, Python gestisce queste
operazioni automaticamente. Questo libera il programmatore dai dettagli
del sistema operativo e dell'elettronica, permettendogli di concentrarsi
sulla logica del problema da risolvere.

Ciò ha un effetto importante sulla versatilità perché spesso è
utilizzato come \emph{interfaccia utente} per linguaggi di livello più
basso come C, C++ o Fortran. Questo permette a Python di sfruttare le
prestazioni dei linguaggi compilati per le parti critiche e
computazionalmente intensive del codice, mantenendo al contempo una
sintassi semplice e leggibile per la maggior parte del programma. Buoni
compilatori per i linguaggi compilati classici possono sì generare
codice binario che gira più velocemente di Python, tuttavia, nella
maggior parte dei casi, le prestazioni delle applicazioni codificate in
Python sono sufficienti.

\section{Python come linguaggio
multiparadigma}\label{python-come-linguaggio-multiparadigma}

Python è un linguaggio di programmazione multiparadigma, il che
significa che supporta diversi paradigmi di programmazione, permettendo
di mescolare e combinare gli stili a seconda delle necessità
dell'applicazione. Ecco alcuni dei paradigmi supportati da Python:

\begin{itemize}
\tightlist
\item
  Programmazione imperativa: Puoi scrivere ed eseguire script Python
  direttamente dalla linea di comando, permettendo un approccio
  interattivo e immediato alla programmazione, come se fosse una
  calcolatrice.
\item
  Programmazione procedurale: In Python, è possibile organizzare il
  codice in funzioni e moduli, rendendo più semplice la gestione e la
  riutilizzabilità del codice. Puoi raccogliere il codice in file
  separati e importarli come moduli, migliorando la struttura e la
  leggibilità del programma.
\item
  Programmazione orientata agli oggetti: Python supporta pienamente la
  programmazione orientata agli oggetti, consentendo la definizione di
  classi e oggetti. Questo paradigma è utile per modellare dati
  complessi e relazioni tra essi. Le caratteristiche orientate agli
  oggetti di Python sono concettualmente simili a quelle di C++, ma più
  semplici da usare.
\item
  Programmazione funzionale: Python include funzionalità di
  programmazione funzionale, come funzioni di prima classe e di ordine
  superiore, lambda e strumenti come map, filter e reduce.
\end{itemize}

Questa flessibilità rende Python adatto a una vasta gamma di
applicazioni e consente ai programmatori di scegliere l'approccio più
adatto al problema da risolvere.

\section{Regole formali e
esperienziali}\label{regole-formali-e-esperienziali}

Python non è solo un linguaggio con regole sintattiche precise e ben
progettate, ma possiede anche una propria filosofia, un insieme di
regole di buon senso esperienziali che sono complementari alla sintassi
formale. Questa filosofia è spesso riassunta nel \textbf{zen di Python},
una raccolta di aforismi che catturano i principi fondamentali del
design di Python. Tali principi aiutano i programmatori a comprendere e
utilizzare al meglio le potenzialità del linguaggio e dell'ecosistema
Python.

Ecco alcuni dei principi dello zen di Python\footnote{\href{https://peps.python.org/pep-0020/}{PEP
  20 -- The Zen of Python}}:

\begin{itemize}
\tightlist
\item
  La leggibilità conta: Il codice dovrebbe essere scritto in modo che
  sia facile da leggere e comprendere.
\item
  Esplicito è meglio di implicito: È preferibile scrivere codice chiaro
  e diretto piuttosto che utilizzare scorciatoie criptiche.
\item
  Semplice è meglio di complesso: Il codice dovrebbe essere il più
  semplice possibile per risolvere il problema.
\item
  Complesso è meglio di complicato: Quando la semplicità non è
  sufficiente, la complessità è accettabile, ma il codice non dovrebbe
  mai essere complicato.
\item
  Pratico batte puro: Le soluzioni pragmatiche sono preferibili alle
  soluzioni eleganti ma poco pratiche.
\end{itemize}

Questi principi, insieme alle regole sintattiche, guidano il
programmatore nell'adottare buone pratiche di sviluppo e nel creare
codice che sia non solo funzionale ma anche mantenibile e comprensibile
da altri.

\section{L'ecosistema}\label{lecosistema}

Fino ad ora abbiamo visto Python come linguaggio, ma è molto di più:
Python è anche una vasta collezione di strumenti e risorse a
disposizione degli sviluppatori, strutturata in un ecosistema completo,
di cui il linguaggio ne rappresenta la parte formale. Questo ecosistema
è disponibile completamente e anche come sorgente sul sito ufficiale
\href{https://www.python.org/}{python.org}.

\subsection{L'interprete}\label{linterprete}

L'interprete Python è lo strumento di esecuzione dei programmi. È il
software che legge ed esegue il codice Python. Python è un linguaggio
interpretato, il che significa che il codice viene eseguito direttamente
dall'interprete, senza bisogno di essere compilato in un linguaggio
macchina. Esistono diverse implementazioni dell'interprete Python:

\begin{itemize}
\tightlist
\item
  CPython: L'implementazione di riferimento dell'interprete Python,
  scritta in C. È la versione più utilizzata e quella ufficiale.
\item
  PyPy: Un interprete alternativo che utilizza tecniche di compilazione
  just-in-time (JIT) per migliorare le prestazioni.
\item
  Jython: Un'implementazione di Python che gira sulla JVM (Java Virtual
  Machine).
\item
  IronPython: Un'implementazione di Python integrata col .NET Framework
  della Microsoft.
\end{itemize}

\subsection{L'ambiente di sviluppo}\label{lambiente-di-sviluppo}

IDLE (integrated development and learning environment) è l'ambiente di
sviluppo integrato ufficiale per Python. È incluso nell'installazione
standard di Python ed è progettato per essere semplice e facile da
usare, ideale per i principianti. Offre diverse funzionalità utili:

\begin{itemize}
\tightlist
\item
  Editor di codice: Con evidenziazione della sintassi, indentazione
  automatica e controllo degli errori.
\item
  Shell interattiva: Permette di eseguire codice Python in modo
  interattivo.
\item
  Strumenti di debug: Include un debugger integrato con punti di
  interruzione e stepping.
\end{itemize}

\subsection{Le librerie standard}\label{le-librerie-standard}

Una delle caratteristiche più potenti di Python è il vasto insieme di
librerie\footnote{\href{https://docs.python.org/3/library/index.html}{Documentazione
  delle librerie standard di Python}} utilizzabili in CPython e IDLE,
che fornisce moduli e pacchetti per quasi ogni necessità di
programmazione. Alcuni esempi, solo allo scopo di illustrarne la
varietà, includono:

\begin{itemize}
\tightlist
\item
  \texttt{os}: Fornisce funzioni per interagire con il sistema
  operativo.
\item
  \texttt{sys}: Offre accesso a funzioni e oggetti del runtime di
  Python.
\item
  \texttt{datetime}: Consente di lavorare con date e orari.
\item
  \texttt{json}: Permette di leggere e scrivere dati in formato JSON.
\item
  \texttt{re}: Supporta la manipolazione di stringhe tramite espressioni
  regolari.
\item
  \texttt{http}: Include moduli per l'implementazione di client e server
  HTTP.
\item
  \texttt{unittest}: Fornisce un framework per il testing del codice.
\item
  \texttt{math} e \texttt{cmath}: Contengono funzioni matematiche di
  base e complesse.
\item
  \texttt{itertools}, \texttt{functools}, \texttt{operator}: Offrono
  supporto per il paradigma di programmazione funzionale.
\item
  \texttt{csv}: Gestisce la lettura e scrittura di file CSV.
\item
  \texttt{typing}: Fornisce supporto per l'annotazione dei tipi di
  variabili, funzioni e classi.
\item
  \texttt{email}: Permette di creare, gestire e inviare email,
  facilitando la manipolazione di messaggi email MIME.
\item
  \texttt{hashlib}: Implementa algoritmi di hash sicuri come SHA-256 e
  MD5.
\item
  \texttt{asyncio}: Supporta la programmazione asincrona per la
  scrittura di codice concorrente e a bassa latenza.
\item
  \texttt{wave}: Fornisce strumenti per leggere e scrivere file audio
  WAV.
\end{itemize}

\subsection{Moduli di estensione}\label{moduli-di-estensione}

Python supporta l'estensione del suo core tramite moduli scritti in C,
C++ o altri linguaggi. Questi moduli permettono di ottimizzare parti
critiche del codice o di interfacciarsi con librerie e API esterne:

\begin{itemize}
\tightlist
\item
  Cython: Permette di scrivere moduli C estesi utilizzando una sintassi
  simile a Python. Cython è ampiamente utilizzato per migliorare le
  prestazioni di parti critiche del codice, specialmente in applicazioni
  scientifiche e di calcolo numerico. Ad esempio, molte librerie
  scientifiche popolari come SciPy e scikit-learn utilizzano Cython per
  accelerare le operazioni computazionalmente intensive.
\item
  ctypes: Permette di chiamare funzioni in librerie dinamiche C
  direttamente da Python. È utile per interfacciarsi con librerie
  esistenti scritte in C, rendendo Python estremamente versatile per
  l'integrazione con altre tecnologie. Ciò è utile in applicazioni che
  devono interfacciarsi con hardware specifico o utilizzare librerie
  legacy.
\item
  CFFI (C foreign function interface): Un'altra interfaccia per chiamare
  librerie C da Python. È progettata per essere facile da usare e per
  supportare l'uso di librerie C complesse con Python. CFFI è utilizzato
  in progetti come PyPy e gevent, permettendo di scrivere codice ad alte
  prestazioni e di gestire le chiamate a funzioni C in modo efficiente.
\end{itemize}

\subsection{Utility e strumenti
aggiuntivi}\label{utility-e-strumenti-aggiuntivi}

Python include anche una serie di strumenti e utility che facilitano lo
sviluppo e la gestione dei progetti:

\begin{itemize}
\tightlist
\item
  pip: Il gestore dei pacchetti di Python. Permette di installare e
  gestire moduli aggiuntivi, cioè non inclusi nello standard.
\item
  venv: Uno strumento per creare ambienti virtuali isolati, che
  permettono di gestire separatamente le dipendenze di diversi progetti.
\item
  Documentazione: Python include una documentazione dettagliata,
  accessibile tramite il comando pydoc o attraverso il sito ufficiale.
\end{itemize}

\section{L'algoritmo di ordinamento bubble
sort}\label{lalgoritmo-di-ordinamento-bubble-sort}

Per chiudere il capitolo sul primo approccio a Python, possiamo
confrontare un algoritmo di bassa complessità in diversi linguaggi di
programmazione. Un buon esempio potrebbe essere l'implementazione
dell'algoritmo di ordinamento \emph{bubble sort} di una lista di valori.
Vediamo come viene scritto in Python, C, C++, Java, Rust e Scala:

\begin{itemize}
\item
  Python (versione procedurale): ```python def bubble\_sort(arr): n =
  len(arr) for i in range(n): for j in range(0, n-i-1): if arr{[}j{]}
  \textgreater{} arr{[}j+1{]}: arr{[}j{]}, arr{[}j+1{]} = arr{[}j+1{]},
  arr{[}j{]}

  \# Esempio di utilizzo arr = {[}64, 34, 25, 12, 22, 11, 90{]}

  bubble\_sort(arr)

  print(``Sorted array is:'', arr) ```
\item
  Python (versione orientata agli oggetti): ```python class BubbleSort:
  (\textbf{staticmethod?}) def bubble\_sort(arr): n = len(arr) for i in
  range(n): for j in range(0, n-i-1): if arr{[}j{]} \textgreater{}
  arr{[}j+1{]}: arr{[}j{]}, arr{[}j+1{]} = arr{[}j+1{]}, arr{[}j{]}

  \# Esempio di utilizzo arr = {[}64, 34, 25, 12, 22, 11, 90{]}

  BubbleSort.bubble\_sort(arr)

  print(``Sorted array is:'', arr) ```
\item
  Python (versione funzionale): ```python def bubble\_sort(arr): def
  sort\_pass(arr, n): if n == 1: return arr new\_arr = {[} (arr{[}i +
  1{]}, arr{[}i{]}) if arr{[}i{]} \textgreater{} arr{[}i + 1{]} else
  (arr{[}i{]}, arr{[}i + 1{]}) for i in range(n - 1) {]} new\_arr =
  {[}item for pair in new\_arr for item in pair{]} return
  sort\_pass(new\_arr{[}:n - 1{]} + arr{[}n:{]}, n - 1)

\begin{verbatim}
  return sort_pass(arr, len(arr))
\end{verbatim}

  \# Esempio di utilizzo arr = {[}64, 34, 25, 12, 22, 11, 90{]}
  sorted\_arr = bubble\_sort(arr) print(``Sorted array is:'',
  sorted\_arr) ```
\item
  C: ```c \#include \textless stdio.h\textgreater{}

  void bubble\_sort(int arr{[}{]}, int n) \{ int i, j, temp; for (i = 0;
  i \textless{} n-1; i++) \{ for (j = 0; j \textless{} n-i-1; j++) \{ if
  (arr{[}j{]} \textgreater{} arr{[}j+1{]}) \{ temp = arr{[}j{]};
  arr{[}j{]} = arr{[}j+1{]}; arr{[}j+1{]} = temp; \} \} \} \}

  int main() \{ int arr{[}{]} = \{64, 34, 25, 12, 22, 11, 90\}; int n =
  sizeof(arr)/sizeof(arr{[}0{]});

\begin{verbatim}
  bubble_sort(arr, n);

  printf("Sorted array is: ");
  for (int i = 0; i < n; i++) {
      printf("%d ", arr[i]);
  }

  return 0;
\end{verbatim}

  \} ```
\item
  C++: ```cpp \#include using namespace std;

  class BubbleSort \{ public: void sort(int arr{[}{]}, int n) \{ for
  (int i = 0; i \textless{} n-1; i++) \{ for (int j = 0; j \textless{}
  n-i-1; j++) \{ if (arr{[}j{]} \textgreater{} arr{[}j+1{]}) \{ int temp
  = arr{[}j{]}; arr{[}j{]} = arr{[}j+1{]}; arr{[}j+1{]} = temp; \} \} \}
  \}

\begin{verbatim}
  void printArray(int arr[], int n) {
      for (int i = 0; i < n; i++) {
          cout << arr[i] << " ";
      }
      cout << endl;
  }
\end{verbatim}

  \};

  int main() \{ int arr{[}{]} = \{64, 34, 25, 12, 22, 11, 90\}; int n =
  sizeof(arr)/sizeof(arr{[}0{]});

\begin{verbatim}
  BubbleSort bs;
  bs.sort(arr, n);

  cout << "Sorted array is: ";
  bs.printArray(arr, n);

  return 0;
\end{verbatim}

  \} ```
\item
  Java: ```java public class BubbleSort \{

\begin{verbatim}
  public static void bubbleSort(int arr[]) {
      int n = arr.length;
      for (int i = 0; i < n-1; i++) {
          for (int j = 0; j < n-i-1; j++) {
              if (arr[j] > arr[j+1]) {
                  int temp = arr[j];
                  arr[j] = arr[j+1];
                  arr[j+1] = temp;
              }
          }
      }
  }

  public static void main(String args[]) {
      int arr[] = {64, 34, 25, 12, 22, 11, 90};

      bubbleSort(arr);

      System.out.println("Sorted array is:");
      for (int i = 0; i < arr.length; i++) {
          System.out.print(arr[i] + " ");
      }
  }
\end{verbatim}

  \} ```
\item
  Rust: ```rust fn bubble\_sort(arr: \&mut {[}i32{]}) \{ let n =
  arr.len(); for i in 0..n \{ for j in 0..n-i-1 \{ if arr{[}j{]}
  \textgreater{} arr{[}j+1{]} \{ arr.swap(j, j+1); \} \} \} \}

  fn main() \{ let mut arr = {[}64, 34, 25, 12, 22, 11, 90{]};

\begin{verbatim}
  bubble_sort(&mut arr);

  println!("Sorted array is: {:?}", arr);
\end{verbatim}

  \} ```
\item
  Scala: ```scala object BubbleSort \{ def bubbleSort(arr:
  Array{[}Int{]}): Unit = \{ val n = arr.length for (i \textless- 0
  until n) \{ for (j \textless- 0 until n - i - 1) \{ if (arr(j)
  \textgreater{} arr(j + 1)) \{ val temp = arr(j) arr(j) = arr(j + 1)
  arr(j + 1) = temp \} \} \} \}

\begin{verbatim}
  def main(args: Array[String]): Unit = {
      val arr = Array(64, 34, 25, 12, 22, 11, 90)

      bubbleSort(arr)

      println("Sorted array is: " + arr.mkString(", "))
  }
\end{verbatim}

  \} ```
\end{itemize}

Confrontando questi esempi, possiamo osservare le differenze sintattiche
e di stile tra Python ed altri, importanti, linguaggi. Python si
distingue per la sua sintassi concisa e leggibile, mentre C richiede una
gestione manuale della memoria e una sintassi più dettagliata.

Il C++ e Java aggiungono caratteristiche relative agli oggetti e
funzionalità di alto livello rispetto a C, al prezzo di una sintassi più
complessa e verbosa. Rust e Scala sono linguaggi più moderni e si
pongono nel mezzo tra C, C++ e Java e Python.

\chapter{Scaricare e installare
Python}\label{scaricare-e-installare-python}

\section{Scaricamento}\label{scaricamento}

\begin{enumerate}
\def\labelenumi{\arabic{enumi}.}
\tightlist
\item
  Visita il sito ufficiale di Python: Vai su
  \href{https://www.python.org/}{python.org}.
\item
  Naviga alla pagina di download: Clicca su \emph{Downloads} nel menu
  principale.
\item
  Scarica il pacchetto di installazione:

  \begin{itemize}
  \tightlist
  \item
    Per Windows: Cerca Python 3.12.x e fai partire il download
    (assicurati di scaricare la versione più recente).
  \item
    Per macOS: Come per Windows.
  \item
    Per Linux: Python è spesso preinstallato. Se non lo è, usa il
    gestore di pacchetti della tua distribuzione (ad esempio
    \texttt{apt} per Ubuntu: \texttt{sudo\ apt-get\ install\ python3}).
  \end{itemize}
\end{enumerate}

\section{Installazione}\label{installazione}

\begin{enumerate}
\def\labelenumi{\arabic{enumi}.}
\tightlist
\item
  Esegui il file di installazione:

  \begin{itemize}
  \tightlist
  \item
    Su Windows: Esegui il file \texttt{.exe} scaricato. Assicurati di
    selezionare l'opzione \texttt{Add\ Python\ to\ PATH} durante
    l'installazione.
  \item
    Su macOS: Apri il file \texttt{.pkg} scaricato e segui le
    istruzioni.
  \item
    Su Linux: Usa il gestore di pacchetti per installare Python.
  \end{itemize}
\item
  Verifica l'installazione:

  \begin{itemize}
  \tightlist
  \item
    Apri il terminale (Command Prompt su Windows, Terminal su macOS e
    Linux).
  \item
    Digita \texttt{python\ -\/-version} o \texttt{python3\ -\/-version}
    e premi Invio. Dovresti vedere la versione di Python installata.
  \end{itemize}
\end{enumerate}

\section{Esecuzione del primo programma: ``Hello,
World!''}\label{esecuzione-del-primo-programma-hello-world}

È consuetudine eseguire come primo programma la visualizzazione della
stringa ``Hello, World!''\footnote{La tradizione del programma ``Hello,
  World!'' ha una lunga storia che risale ai primi giorni della
  programmazione. Questo semplice programma è generalmente il primo
  esempio utilizzato per introdurre i nuovi programmatori alla sintassi
  e alla struttura di un linguaggio di programmazione. Il programma
  ``Hello, World!'' è diventato famoso grazie a Brian Kernighan, che lo
  ha incluso nel suo libro (Kernighan e Ritchie 1988) pubblicato nel
  1978. Tuttavia, il suo utilizzo risale a un testo precedente di
  Kernighan, (Kernighan 1973), pubblicato nel 1973, dove veniva
  utilizzato un esempio simile.}. Possiamo farlo in diversi modi e ciò è
una delle caratteristiche più apprezzate di Python.

\subsection{REPL}\label{repl}

Il primo modo prevede l'utilizzo del REPL di Python. Il REPL
(read-eval-print loop) è un ambiente interattivo di esecuzione di
comandi Python generato dall'interprete, secondo il ciclo:

\begin{enumerate}
\def\labelenumi{\arabic{enumi}.}
\tightlist
\item
  Read: Legge un input dell'utente.
\item
  Eval: Valuta l'input.
\item
  Print: Visualizza il risultato dell'esecuzione.
\item
  Loop: Ripete il ciclo.
\end{enumerate}

Eseguiamo il nostro primo ``Hello, World!'':

\begin{enumerate}
\def\labelenumi{\arabic{enumi}.}
\tightlist
\item
  Apri il terminale ed esegui l'interprete Python digitando
  \texttt{python} o \texttt{python3} e premi il tasto di invio della
  tastiera.
\item
  Scrivi ed esegui il programma:
\end{enumerate}

\begin{Shaded}
\begin{Highlighting}[]
\BuiltInTok{print}\NormalTok{(}\StringTok{"Hello, World!"}\NormalTok{)}
\end{Highlighting}
\end{Shaded}

Premi il tasto di invio per vedere il risultato immediatamente.

\begin{tcolorbox}[enhanced jigsaw, breakable, opacityback=0, arc=.35mm, left=2mm, rightrule=.15mm, bottomtitle=1mm, title=\textcolor{quarto-callout-warning-color}{\faExclamationTriangle}\hspace{0.5em}{Attenzione}, colframe=quarto-callout-warning-color-frame, toprule=.15mm, coltitle=black, leftrule=.75mm, colback=white, bottomrule=.15mm, opacitybacktitle=0.6, colbacktitle=quarto-callout-warning-color!10!white, toptitle=1mm, titlerule=0mm]

Il REPL e l'interprete Python sono strettamente collegati, ma non sono
esattamente la stessa cosa. Quando avvii l'interprete Python senza
specificare un file di script da eseguire (digitando semplicemente
\texttt{python} o \texttt{python3} nel terminale), entri in modalità
REPL. Nel REPL, l'interprete Python legge l'input direttamente
dall'utente, lo esegue, stampa il risultato e poi attende il prossimo
input. In sintesi, l'interprete può eseguire programmi Python completi
salvati in file, il REPL è progettato per un'esecuzione interattiva e
immediata di singole istruzioni.

\end{tcolorbox}

\subsection{Interprete}\label{interprete}

Un altro modo per eseguire il nostro programma ``Hello, World!'' è
utilizzare l'interprete Python per eseguire un file di codice sorgente.
Questo metodo è utile per scrivere programmi più complessi e per
mantenere il codice per usi futuri.

Ecco come fare sui diversi sistemi operativi.

\section{Windows}

\begin{enumerate}
\def\labelenumi{\arabic{enumi}.}
\item
  Crea un file di testo:

  \begin{enumerate}
  \def\labelenumii{\arabic{enumii}.}
  \tightlist
  \item
    Apri il tuo editor di testo preferito, come Notepad.
  \item
    Scrivi il seguente codice nel file:
  \end{enumerate}

\begin{Shaded}
\begin{Highlighting}[]
 \BuiltInTok{print}\NormalTok{(}\StringTok{"Hello, World!"}\NormalTok{)}
\end{Highlighting}
\end{Shaded}

  \begin{enumerate}
  \def\labelenumii{\arabic{enumii}.}
  \setcounter{enumii}{2}
  \tightlist
  \item
    Salva il file con il nome \texttt{hello.txt}.
  \end{enumerate}
\item
  Rinomina il file (facoltativo): se desideri mantenere il file senza
  estensione \texttt{.txt}, puoi rinominarlo in \texttt{hello}
  direttamente dall'Esplora file.
\item
  Esegui il file Python:

  \begin{enumerate}
  \def\labelenumii{\arabic{enumii}.}
  \tightlist
  \item
    Apri il prompt dei comandi.
  \item
    Naviga fino alla directory in cui hai salvato il file. Ad esempio,
    se il file si trova nella cartella \texttt{Documenti}, puoi
    digitare:
  \end{enumerate}

\begin{Shaded}
\begin{Highlighting}[]
\BuiltInTok{cd}\NormalTok{ \%HOMEPATH\%}\DataTypeTok{\textbackslash{}D}\NormalTok{ocumenti}
\end{Highlighting}
\end{Shaded}

  \begin{enumerate}
  \def\labelenumii{\arabic{enumii}.}
  \setcounter{enumii}{2}
  \tightlist
  \item
    Esegui l'interprete Python passando come argomento il file che hai
    creato:
  \end{enumerate}

\begin{Shaded}
\begin{Highlighting}[]
\NormalTok{python hello.txt}
\end{Highlighting}
\end{Shaded}

  oppure, se il tuo sistema utilizza \texttt{python3}:
  \texttt{dos\ \ \ python3\ hello.txt}
\item
  Visualizza il risultato:

\begin{Shaded}
\begin{Highlighting}[]
\NormalTok{Hello, World!}
\end{Highlighting}
\end{Shaded}
\end{enumerate}

\section{macOS}

\begin{enumerate}
\def\labelenumi{\arabic{enumi}.}
\item
  Crea un file di testo:

  \begin{enumerate}
  \def\labelenumii{\arabic{enumii}.}
  \tightlist
  \item
    Apri il tuo editor di testo preferito, come TextEdit.
  \item
    Scrivi il seguente codice nel file:
  \end{enumerate}

\begin{Shaded}
\begin{Highlighting}[]
\BuiltInTok{print}\NormalTok{(}\StringTok{"Hello, World!"}\NormalTok{)}
\end{Highlighting}
\end{Shaded}

  \begin{enumerate}
  \def\labelenumii{\arabic{enumii}.}
  \setcounter{enumii}{2}
  \tightlist
  \item
    Salva il file con il nome \texttt{hello.txt}.
  \end{enumerate}
\item
  Rinomina il file (facoltativo): se desideri mantenere il file senza
  estensione \texttt{.txt}, puoi rinominarlo in \texttt{hello}
  direttamente dal Finder.
\item
  Esegui il file Python:

  \begin{enumerate}
  \def\labelenumii{\arabic{enumii}.}
  \tightlist
  \item
    Apri il terminale del sistema operativo.
  \item
    Naviga fino alla directory in cui hai salvato il file. Ad esempio,
    se il file si trova nella cartella \texttt{Documenti}, puoi
    digitare:
  \end{enumerate}

\begin{Shaded}
\begin{Highlighting}[]
\BuiltInTok{cd}\NormalTok{ \textasciitilde{}/Documents}
\end{Highlighting}
\end{Shaded}

  \begin{enumerate}
  \def\labelenumii{\arabic{enumii}.}
  \setcounter{enumii}{2}
  \tightlist
  \item
    Esegui l'interprete Python passando come argomento il file che hai
    creato:
  \end{enumerate}

\begin{Shaded}
\begin{Highlighting}[]
\ExtensionTok{python3}\NormalTok{ hello.txt}
\end{Highlighting}
\end{Shaded}
\item
  Visualizza il risultato:

\begin{Shaded}
\begin{Highlighting}[]
\ExtensionTok{Hello,}\NormalTok{ World!}
\end{Highlighting}
\end{Shaded}
\end{enumerate}

\section{Linux}

\begin{enumerate}
\def\labelenumi{\arabic{enumi}.}
\item
  Crea un file di testo:

  \begin{enumerate}
  \def\labelenumii{\arabic{enumii}.}
  \tightlist
  \item
    Apri il tuo editor di testo preferito, come Gedit o Nano.
  \item
    Scrivi il seguente codice nel file:
  \end{enumerate}

\begin{Shaded}
\begin{Highlighting}[]
\BuiltInTok{print}\NormalTok{(}\StringTok{"Hello, World!"}\NormalTok{)}
\end{Highlighting}
\end{Shaded}

  \begin{enumerate}
  \def\labelenumii{\arabic{enumii}.}
  \setcounter{enumii}{2}
  \tightlist
  \item
    Salva il file con il nome \texttt{hello.txt}.
  \end{enumerate}
\item
  Rinomina il file (facoltativo): se desideri mantenere il file senza
  estensione \texttt{.txt}, puoi rinominarlo in \texttt{hello}
  utilizzando il comando \texttt{mv} nel terminale:
  \texttt{bash\ \ \ \ \ \ mv\ hello.txt\ hello}
\item
  Esegui il file Python:

  \begin{enumerate}
  \def\labelenumii{\arabic{enumii}.}
  \tightlist
  \item
    Apri il terminale del sistema operativo.
  \item
    Naviga fino alla directory in cui hai salvato il file. Ad esempio,
    se il file si trova nella cartella \texttt{Documenti}, puoi
    digitare:
  \end{enumerate}

\begin{Shaded}
\begin{Highlighting}[]
\BuiltInTok{cd}\NormalTok{ \textasciitilde{}/Documenti}
\end{Highlighting}
\end{Shaded}

  \begin{enumerate}
  \def\labelenumii{\arabic{enumii}.}
  \setcounter{enumii}{2}
  \tightlist
  \item
    Esegui l'interprete Python passando come argomento il file che hai
    creato:
  \end{enumerate}

\begin{Shaded}
\begin{Highlighting}[]
\ExtensionTok{python3}\NormalTok{ hello.txt}
\end{Highlighting}
\end{Shaded}
\item
  Visualizza il risultato:

\begin{Shaded}
\begin{Highlighting}[]
\ExtensionTok{Hello,}\NormalTok{ World!}
\end{Highlighting}
\end{Shaded}
\end{enumerate}

Con queste istruzioni, dovresti essere in grado di eseguire il programma
``Hello, World!'' utilizzando un file Python su Windows, macOS e Linux.

\subsection{IDE}\label{ide}

Utilizzo di un IDE (integrated development environment) installato sul
computer. Ecco alcuni dei più comuni e gratuiti.

\section{IDLE}

Incluso con l'installazione di Python.

\begin{enumerate}
\def\labelenumi{\arabic{enumi}.}
\tightlist
\item
  Avvia IDLE.
\item
  Crea un nuovo file (\texttt{File\ -\textgreater{}\ New\ File}).
\item
  Scrivi il programma:
\end{enumerate}

\begin{Shaded}
\begin{Highlighting}[]
\BuiltInTok{print}\NormalTok{(}\StringTok{"Hello, World!"}\NormalTok{)}
\end{Highlighting}
\end{Shaded}

\begin{enumerate}
\def\labelenumi{\arabic{enumi}.}
\setcounter{enumi}{3}
\tightlist
\item
  Salva il file (\texttt{File\ -\textgreater{}\ Salva}).
\item
  Esegui il programma (\texttt{Run\ -\textgreater{}\ Run\ Module}).
\end{enumerate}

\section{PyCharm}

Proprietario ma con una versione liberamente fruibile.

\begin{enumerate}
\def\labelenumi{\arabic{enumi}.}
\tightlist
\item
  Scarica e installa PyCharm da
  \href{https://www.jetbrains.com/pycharm/download/}{jetbrains.com/pycharm/download}.
\item
  Crea un nuovo progetto associando l'interprete Python.
\item
  Crea un nuovo file Python
  (\texttt{File\ -\textgreater{}\ New\ -\textgreater{}\ Python\ File}).
\item
  Scrivi il programma:
\end{enumerate}

\begin{Shaded}
\begin{Highlighting}[]
\BuiltInTok{print}\NormalTok{(}\StringTok{"Hello, World!"}\NormalTok{)}
\end{Highlighting}
\end{Shaded}

\begin{enumerate}
\def\labelenumi{\arabic{enumi}.}
\setcounter{enumi}{4}
\tightlist
\item
  Esegui il programma (\texttt{Run\ -\textgreater{}\ Run...}).
\end{enumerate}

\section{Visual Studio Code}

Proprietario ma liberamente fruibile.

\begin{enumerate}
\def\labelenumi{\arabic{enumi}.}
\tightlist
\item
  Scarica e installa VS Code da
  \href{https://code.visualstudio.com/}{code.visualstudio.com}.
\item
  Installa l'estensione Python.
\item
  Apri o crea una nuova cartella di progetto.
\item
  Crea un nuovo file Python
  (\texttt{File\ -\textgreater{}\ Nuovo\ file}).
\item
  Scrivi il programma:
\end{enumerate}

\begin{Shaded}
\begin{Highlighting}[]
\BuiltInTok{print}\NormalTok{(}\StringTok{"Hello, World!"}\NormalTok{)}
\end{Highlighting}
\end{Shaded}

\begin{enumerate}
\def\labelenumi{\arabic{enumi}.}
\setcounter{enumi}{5}
\tightlist
\item
  Salva il file con estensione \texttt{.py}, ad esempio
  \texttt{hello\_world.py}.
\item
  Esegui il programma utilizzando il terminale integrato
  (\texttt{Visualizza\ -\textgreater{}\ Terminale}) e digitando
  \texttt{python\ hello\_world.py}.
\end{enumerate}

\subsection{Esecuzione nel browser}\label{esecuzione-nel-browser}

Puoi eseguire Python direttamente nel browser, senza installare nulla.
Anche qui abbiamo diverse alternative, sia eseguendo il codice
localmente, che utilizzando piattaforme online.

\section{Repl.it}

\begin{enumerate}
\def\labelenumi{\arabic{enumi}.}
\tightlist
\item
  Visita \href{https://repl.it/}{repl.it}.
\item
  Crea un nuovo progetto selezionando Python.
\item
  Scrivi il programma:
\end{enumerate}

\begin{Shaded}
\begin{Highlighting}[]
\BuiltInTok{print}\NormalTok{(}\StringTok{"Hello, World!"}\NormalTok{)}
\end{Highlighting}
\end{Shaded}

\begin{enumerate}
\def\labelenumi{\arabic{enumi}.}
\setcounter{enumi}{3}
\tightlist
\item
  Clicca su ``Run'' per eseguire il programma.
\end{enumerate}

\section{Google Colab}

\begin{enumerate}
\def\labelenumi{\arabic{enumi}.}
\tightlist
\item
  Visita
  \href{https://colab.research.google.com/}{colab.research.google.com}.
\item
  Crea un nuovo notebook.
\item
  In una cella di codice, scrivi:
\end{enumerate}

\begin{Shaded}
\begin{Highlighting}[]
\BuiltInTok{print}\NormalTok{(}\StringTok{"Hello, World!"}\NormalTok{)}
\end{Highlighting}
\end{Shaded}

\begin{enumerate}
\def\labelenumi{\arabic{enumi}.}
\setcounter{enumi}{3}
\tightlist
\item
  Premi il pulsante di esecuzione accanto alla cella.
\end{enumerate}

\section{PyScript}

\begin{enumerate}
\def\labelenumi{\arabic{enumi}.}
\tightlist
\item
  Visita il sito ufficiale di \href{https://pyscript.net/}{PyScript} per
  ulteriori informazioni su come iniziare.
\item
  Crea un file HTML con il seguente contenuto:
\end{enumerate}

\begin{Shaded}
\begin{Highlighting}[]
\DataTypeTok{\textless{}!DOCTYPE}\NormalTok{ html}\DataTypeTok{\textgreater{}}
\DataTypeTok{\textless{}}\KeywordTok{html}\OtherTok{ lang}\OperatorTok{=}\StringTok{"en"}\DataTypeTok{\textgreater{}}
\DataTypeTok{\textless{}}\KeywordTok{head}\DataTypeTok{\textgreater{}}
    \DataTypeTok{\textless{}}\KeywordTok{meta}\OtherTok{ charset}\OperatorTok{=}\StringTok{"UTF{-}8"}\DataTypeTok{\textgreater{}}
    \DataTypeTok{\textless{}}\KeywordTok{title}\DataTypeTok{\textgreater{}}\NormalTok{Hello, World with PyScript}\DataTypeTok{\textless{}/}\KeywordTok{title}\DataTypeTok{\textgreater{}}
    \DataTypeTok{\textless{}}\KeywordTok{link}\OtherTok{ rel}\OperatorTok{=}\StringTok{"stylesheet"}\OtherTok{ href}\OperatorTok{=}\StringTok{"https://pyscript.net/latest/pyscript.css"}\DataTypeTok{\textgreater{}}
    \DataTypeTok{\textless{}}\KeywordTok{script}\OtherTok{ defer src}\OperatorTok{=}\StringTok{"https://pyscript.net/latest/pyscript.js"}\DataTypeTok{\textgreater{}\textless{}/}\KeywordTok{script}\DataTypeTok{\textgreater{}}
\DataTypeTok{\textless{}/}\KeywordTok{head}\DataTypeTok{\textgreater{}}
\DataTypeTok{\textless{}}\KeywordTok{body}\DataTypeTok{\textgreater{}}
    \DataTypeTok{\textless{}}\KeywordTok{py{-}script}\DataTypeTok{\textgreater{}}
\NormalTok{        print("Hello, World!")}
    \DataTypeTok{\textless{}/}\KeywordTok{py{-}script}\DataTypeTok{\textgreater{}}
\DataTypeTok{\textless{}/}\KeywordTok{body}\DataTypeTok{\textgreater{}}
\DataTypeTok{\textless{}/}\KeywordTok{html}\DataTypeTok{\textgreater{}}
\end{Highlighting}
\end{Shaded}

\begin{enumerate}
\def\labelenumi{\arabic{enumi}.}
\setcounter{enumi}{2}
\tightlist
\item
  Salva il file con estensione \texttt{.html} (ad esempio,
  \texttt{hello.html}).
\item
  Apri il file salvato in un browser web. Vedrai l'output
  \texttt{Hello,\ World!} direttamente nella pagina.
\end{enumerate}

\subsection{Jupyter Notebook}\label{jupyter-notebook}

Jupyter Notebook è un ambiente di sviluppo interattivo per la
programmazione che permette di creare e condividere documenti contenenti
codice eseguibile, visualizzazioni, testo formattato e altro ancora.
Originariamente sviluppato come parte del progetto IPython, Jupyter
supporta non solo Python, ma anche numerosi altri linguaggi di
programmazione attraverso i cosiddetti kernel tra cui R, Julia e Scala.

\section{Uso locale}

\begin{enumerate}
\def\labelenumi{\arabic{enumi}.}
\tightlist
\item
  Assicurati di avere Python e Jupyter installati sul tuo computer. Se
  non li hai, puoi installarli utilizzando Anaconda o pip:
\end{enumerate}

\begin{Shaded}
\begin{Highlighting}[]
\ExtensionTok{pip}\NormalTok{ install notebook}
\end{Highlighting}
\end{Shaded}

\begin{enumerate}
\def\labelenumi{\arabic{enumi}.}
\setcounter{enumi}{1}
\tightlist
\item
  Avvia Jupyter Notebook dal terminale:
\end{enumerate}

\begin{Shaded}
\begin{Highlighting}[]
\ExtensionTok{jupyter}\NormalTok{ notebook}
\end{Highlighting}
\end{Shaded}

\begin{enumerate}
\def\labelenumi{\arabic{enumi}.}
\setcounter{enumi}{2}
\tightlist
\item
  Crea un nuovo notebook Python.
\item
  In una cella di codice, scrivi:
\end{enumerate}

\begin{Shaded}
\begin{Highlighting}[]
\BuiltInTok{print}\NormalTok{(}\StringTok{"Hello, World!"}\NormalTok{)}
\end{Highlighting}
\end{Shaded}

\begin{enumerate}
\def\labelenumi{\arabic{enumi}.}
\setcounter{enumi}{4}
\tightlist
\item
  Premi \texttt{Shift\ +\ Enter} per eseguire la cella.
\end{enumerate}

\section{JupyterHub}

\begin{enumerate}
\def\labelenumi{\arabic{enumi}.}
\tightlist
\item
  Visita l'istanza di JupyterHub della tua istituzione o azienda
  (\href{https://jupyter.org/hub}{maggiori informazioni}).
\item
  Accedi con le tue credenziali.
\item
  Crea un nuovo notebook Python.
\item
  In una cella di codice, scrivi:
\end{enumerate}

\begin{Shaded}
\begin{Highlighting}[]
\BuiltInTok{print}\NormalTok{(}\StringTok{"Hello, World!"}\NormalTok{)}
\end{Highlighting}
\end{Shaded}

\begin{enumerate}
\def\labelenumi{\arabic{enumi}.}
\setcounter{enumi}{4}
\tightlist
\item
  Premi \texttt{Shift\ +\ Enter} per eseguire la cella.
\end{enumerate}

\section{Binder}

\begin{enumerate}
\def\labelenumi{\arabic{enumi}.}
\tightlist
\item
  Visita \href{https://mybinder.org/}{mybinder.org}.
\item
  Inserisci l'URL del repository GitHub che contiene il tuo notebook o
  il tuo progetto Python.
\item
  Clicca su ``Launch''.
\item
  Una volta avviato l'ambiente, crea un nuovo notebook o apri uno
  esistente.
\item
  In una cella di codice, scrivi:
\end{enumerate}

\begin{Shaded}
\begin{Highlighting}[]
\BuiltInTok{print}\NormalTok{(}\StringTok{"Hello, World!"}\NormalTok{)}
\end{Highlighting}
\end{Shaded}

\begin{enumerate}
\def\labelenumi{\arabic{enumi}.}
\setcounter{enumi}{5}
\tightlist
\item
  Premi \texttt{Shift\ +\ Enter} per eseguire la cella.
\end{enumerate}

Binder è un servizio simile a Colab, anche se quest'ultimo offre
strumenti generalmente più avanzati in termini di risorse computazionali
e collaborazione. Binder di contro è basato su GitHub e ciò può essere
utile in alcuni contesti.

\chapter{La struttura lessicale di
Python}\label{la-struttura-lessicale-di-python}

Per iniziare ad imparare Python come linguaggio, partiamo da una
semplificazione della stuttura lessicale, cioè dall'insieme di regole
sintattiche più significative, sia per comprendere le regole di
composizione di programmi comprensibili all'interprete, sia per
sfrutturne appieno tutte le potenzialità.

Ogni programma Python è costituito da una serie di file di testo
contenenti il codice sorgente con una certa codifica, il default è
l'UTF-8, ed ogni file si può vedere come una sequenza di istruzioni,
righe e token. Le istruzioni danno la granularità dell'algoritmo, le
righe definiscono come queste istruzioni sono distribuite nel testo e,
infine, i token sono gli elementi atomici che hanno un significato per
il linguaggio.

\section{Righe}\label{righe}

Le righe sono di due tipi: logiche e fisiche. Le seconde sono le più
facilmente individuabili nel testo di un programma, perché sono
terminate da un carattere di a capo. Una o più righe fisiche
costituiscono una riga logica che corrisponde ad una istruzione.

Vi sono due modi per dividere una riga logica in righe fisiche. Il primo
è terminare con il backslash (\VERB|\OperatorTok{\textbackslash{}}|,
poco usata la traduzione ``barra rovesciata'' o simili) tutte le righe
fisiche meno l'ultima (intendendo con ciò che il backslash precede l'a
capo):

\phantomsection\label{annotated-cell-32}%
\begin{Shaded}
\begin{Highlighting}[]
\NormalTok{x }\OperatorTok{=} \DecValTok{1} \OperatorTok{+} \DecValTok{2} \OperatorTok{+}\NormalTok{ \textbackslash{} }\hspace*{\fill}\NormalTok{\circled{1}}
     \DecValTok{3}

\ControlFlowTok{if}\NormalTok{ x }\OperatorTok{\textgreater{}} \DecValTok{5} \KeywordTok{and}\NormalTok{ \textbackslash{} }\hspace*{\fill}\NormalTok{\circled{2}}
\NormalTok{   x }\OperatorTok{\textless{}} \DecValTok{9}\NormalTok{: }\hspace*{\fill}\NormalTok{\circled{3}}
  \BuiltInTok{print}\NormalTok{(}\StringTok{"5 \textless{} x \textless{} 9"}\NormalTok{) }
\end{Highlighting}
\end{Shaded}

\begin{description}
\tightlist
\item[\circled{1}]
L'istruzione di assegnamento è spezzata su due righe fisiche
\item[\circled{2}]
L'istruzione condizionale ha due espressioni che devono essere entrambe
vere, ognuna su una riga fisica.
\item[\circled{3}]
Non importa quanto sono indentate le righe fisiche successive alla prima
e ciò può essere sfruttato per incrementare la leggibilità, ad esempio,
allineando le espressioni \texttt{x\ \textgreater{}\ 5} e
\texttt{x\ \textless{}\ 9} in colonna.
\end{description}

Il secondo è per mezzo di parentesi, giacché tutte le righe fisiche che
seguono una con parentesi tonda \VERB|\NormalTok{(}|, quadra
\VERB|\NormalTok{[}| o graffa \VERB|\NormalTok{\{}| aperta, fino a
quella con l'analoga parentesi chiusa, sono unite in una logica. Le
regole di indentazione, che vedremo nel seguito, si applicano solo alla
prima riga fisica.

Esempi:

\phantomsection\label{annotated-cell-33}%
\begin{Shaded}
\begin{Highlighting}[]
\NormalTok{x }\OperatorTok{=}\NormalTok{ (}\DecValTok{1} \OperatorTok{+} \DecValTok{2} \hspace*{\fill}\NormalTok{\circled{1}}
     \OperatorTok{+} \DecValTok{3} \OperatorTok{+} \DecValTok{4}\NormalTok{)}

\NormalTok{y }\OperatorTok{=}\NormalTok{ [}\DecValTok{1}\NormalTok{, }\DecValTok{2}\NormalTok{, }
           \DecValTok{3}\NormalTok{, }\DecValTok{4} \OperatorTok{+} \hspace*{\fill}\NormalTok{\circled{2}}
      \DecValTok{5}\NormalTok{] }

\NormalTok{z }\OperatorTok{=}\NormalTok{ [}\DecValTok{1}\NormalTok{, }\DecValTok{2} \hspace*{\fill}\NormalTok{\circled{3}}
\NormalTok{     , }\DecValTok{3}\NormalTok{, }\DecValTok{4}\NormalTok{]   }
\end{Highlighting}
\end{Shaded}

\begin{description}
\tightlist
\item[\circled{1}]
L'espressione è spezzata su due righe fisiche e le parentesi tonde
rappresentano un'alternativa all'uso del backslash.
\item[\circled{2}]
Questa riga e la successiva non hanno la stessa indentazione, anche se è
da evitare perché poco leggibile.
\item[\circled{3}]
La lista è spezzata su due righe fisiche e, anche qui, così è poco
leggibile.
\end{description}

\section{Commenti}\label{commenti}

Un commento inizia con un carattere cancelletto (\VERB|\CommentTok{\#}|)
e termina alla fine della riga fisica. I commenti non possono coesistere
con il backslash come separatore di riga logica, giacché entrambi devono
chiudere la riga fisica:

\phantomsection\label{annotated-cell-34}%
\begin{Shaded}
\begin{Highlighting}[]
\NormalTok{x }\OperatorTok{=} \DecValTok{1} \OperatorTok{+} \DecValTok{2} \OperatorTok{+}\NormalTok{ \textbackslash{} }\CommentTok{\# Commento }\hspace*{\fill}\NormalTok{\circled{1}}
     \DecValTok{3}

\ControlFlowTok{if}\NormalTok{ x }\OperatorTok{\textgreater{}} \DecValTok{5} \KeywordTok{and} \CommentTok{\# Commento \textbackslash{} }\hspace*{\fill}\NormalTok{\circled{2}}
\NormalTok{   x }\OperatorTok{\textless{}} \DecValTok{9}\NormalTok{: }
  \BuiltInTok{print}\NormalTok{(}\StringTok{"5 \textless{} x \textless{} 9"}\NormalTok{) }
\end{Highlighting}
\end{Shaded}

\begin{description}
\tightlist
\item[\circled{1}]
Il backslash deve terminare la riga fisica, quindi non può essere
seguito da un commento. Se necessario può andare o alla riga successiva,
scelta consigliata, o la precedente.
\item[\circled{2}]
Il commento rende il backslash parte di esso quindi non segnala più la
fine della riga fisica e, all'esecuzione, si avrà un
\href{https://docs.python.org/3/library/exceptions.html\#SyntaxError}{SyntaxError}
perché \texttt{and} deve essere seguito da un'espressione.
\end{description}

\section{Indentazione}\label{indentazione}

Indentazione significa che spazi o, in alternativa, tabulazioni
precedono un carattere che non sia nessuno dei due. Il numero di spazi
ottenuto dopo la trasformazione delle tabulazioni in spazi, si definisce
livello di indentazione. L'indentazione del codice è il modo che Python
utilizza per raggruppare le istruzioni in un blocco, ove tutte devono
presentare la medesima indentazione. La prima riga logica che ha una
indentazione minore della precedente, segnala che il blocco è stato
chiuso proprio da quest'ultima. Anche le clausole di un'istruzione
composta devono avere la stessa indentazione.

La prima istruzione di un file o la prima inserita al prompt
\VERB|\OperatorTok{\textgreater{}\textgreater{}\textgreater{}}| del REPL
non deve presentare spazi o tabulazioni, cioè ha un livello di
indentazione pari a 0.

Alcuni esempi:

\begin{itemize}
\item
  Definizione di una funzione:

\phantomsection\label{annotated-cell-82}%
\begin{Shaded}
\begin{Highlighting}[]
\KeywordTok{def}\NormalTok{ somma(a, b): }\hspace*{\fill}\NormalTok{\circled{1}}
\NormalTok{  risultato }\OperatorTok{=}\NormalTok{ a }\OperatorTok{+}\NormalTok{ b }\hspace*{\fill}\NormalTok{\circled{2}}
  \ControlFlowTok{return}\NormalTok{ risultato }
\end{Highlighting}
\end{Shaded}

  \begin{description}
  \tightlist
  \item[\circled{1}]
  Prima riga senza indentazione.
  \item[\circled{2}]
  Questa riga e la successiva appartengono allo stesso blocco e,
  pertanto, hanno la medesima indentazione.
  \end{description}
\item
  Ciclo e test di condizione:

\phantomsection\label{annotated-cell-83}%
\begin{Shaded}
\begin{Highlighting}[]
\NormalTok{x }\OperatorTok{=} \DecValTok{10}

\ControlFlowTok{if}\NormalTok{ x }\OperatorTok{\textless{}} \DecValTok{0}\NormalTok{: }\hspace*{\fill}\NormalTok{\circled{1}}
    \BuiltInTok{print}\NormalTok{(}\StringTok{"x è negativo"}\NormalTok{) }\hspace*{\fill}\NormalTok{\circled{2}}
\ControlFlowTok{elif}\NormalTok{ x }\OperatorTok{==} \DecValTok{0}\NormalTok{: }
      \BuiltInTok{print}\NormalTok{(}\StringTok{"x è zero"}\NormalTok{) }
\ControlFlowTok{else}\NormalTok{: }
        \BuiltInTok{print}\NormalTok{(}\StringTok{"x è positivo"}\NormalTok{) }
\end{Highlighting}
\end{Shaded}

  \begin{description}
  \tightlist
  \item[\circled{1}]
  Le tre clausole \VERB|\ControlFlowTok{if}| , \VERB|\NormalTok{then}| e
  \VERB|\ControlFlowTok{else}| hanno identica indentazione.
  \item[\circled{2}]
  I tre blocchi hanno come unico vincolo quello di avere un livello
  maggiore della riga precedente. I blocchi corrispondenti alle diverse
  clausole non devono avere lo stesso livello di indentazione, anche se
  è buona prassi farlo.
  \end{description}
\end{itemize}

\begin{tcolorbox}[enhanced jigsaw, breakable, opacityback=0, arc=.35mm, left=2mm, rightrule=.15mm, bottomtitle=1mm, title=\textcolor{quarto-callout-warning-color}{\faExclamationTriangle}\hspace{0.5em}{Attenzione}, colframe=quarto-callout-warning-color-frame, toprule=.15mm, coltitle=black, leftrule=.75mm, colback=white, bottomrule=.15mm, opacitybacktitle=0.6, colbacktitle=quarto-callout-warning-color!10!white, toptitle=1mm, titlerule=0mm]

Non si possono avere sia spazi che tabulazioni per definire il livello
di indentazione nello stesso file. Ciò perché renderebbe ambiguo il
numero di spazi che si ottiene dopo la trasformazione delle tabulazioni
in spazi. Quindi, o si usano spazi, scelta raccomandata, o tabulazioni.

\end{tcolorbox}

\section{Token}\label{token}

Le righe logiche sono composte da token che si categorizzano in parole
chiave, identificatori, operatori, delimitatori e letterali. I token
sono separati da un numero arbitrario di spazi e tabulazioni. Ad
esempio:

\begin{Shaded}
\begin{Highlighting}[]
\NormalTok{x }\OperatorTok{=} \DecValTok{1}  \OperatorTok{+}  \DecValTok{2}  \OperatorTok{+}  \DecValTok{3}

\ControlFlowTok{if}\NormalTok{    x }\OperatorTok{\textgreater{}} \DecValTok{5}    \KeywordTok{and}\NormalTok{    x }\OperatorTok{\textless{}} \DecValTok{9}\NormalTok{: }
  \BuiltInTok{print}\NormalTok{(}\StringTok{"5 \textless{} x \textless{} 9"}\NormalTok{) }
\end{Highlighting}
\end{Shaded}

\subsection{Identificatori}\label{identificatori}

Un identificatore è un nome assegnato ad un oggetto, cioè una variabile,
una funzione, una classe, un modulo e altro. Esso è \emph{case
sensitive} cioè \VERB|\NormalTok{python}| e \VERB|\NormalTok{Python}|
sono due identificatori diversi.

\begin{Shaded}
\begin{Highlighting}[]
\NormalTok{intero }\OperatorTok{=} \DecValTok{42}  \CommentTok{\# Identificatore di un numero intero}
\NormalTok{decimale }\OperatorTok{=} \FloatTok{3.14}  \CommentTok{\# Identificatore di un numero decimale}
\NormalTok{testo }\OperatorTok{=} \StringTok{"Ciao, mondo!"}  \CommentTok{\# Identificatore di una stringa}
\NormalTok{lista }\OperatorTok{=}\NormalTok{ [}\DecValTok{1}\NormalTok{, }\DecValTok{2}\NormalTok{, }\DecValTok{3}\NormalTok{]  }\CommentTok{\# Identificatore di una lista}
\NormalTok{dizionario }\OperatorTok{=}\NormalTok{ \{}\StringTok{"chiave"}\NormalTok{: }\StringTok{"valore"}\NormalTok{\}  }\CommentTok{\# Identificatore di un dizionario}

\KeywordTok{def}\NormalTok{ mia\_funzione(): }\CommentTok{\# Identificatore di funzione}
    \BuiltInTok{print}\NormalTok{(}\StringTok{"Questa è una funzione"}\NormalTok{)}

\CommentTok{\# Classe}
\KeywordTok{class}\NormalTok{ MiaClasse: }\CommentTok{\# Identificatore di classe}
    \KeywordTok{def} \FunctionTok{\_\_init\_\_}\NormalTok{(}\VariableTok{self}\NormalTok{, valore): }\CommentTok{\# Identificatore di metodo}
        \VariableTok{self}\NormalTok{.valore }\OperatorTok{=}\NormalTok{ valore  }\CommentTok{\# Identificatore per un attributo}
    
    \KeywordTok{def}\NormalTok{ metodo(}\VariableTok{self}\NormalTok{):}
        \BuiltInTok{print}\NormalTok{(}\StringTok{"Questo è un metodo della classe"}\NormalTok{)}

\ImportTok{import}\NormalTok{ math  }\CommentTok{\# Identificatore di un modulo}

\KeywordTok{def}\NormalTok{ mio\_generatore(): }\CommentTok{\# Identificatore di un generatore}
    \ControlFlowTok{yield} \DecValTok{1}
    \ControlFlowTok{yield} \DecValTok{2}
    \ControlFlowTok{yield} \DecValTok{3}

\NormalTok{mio\_oggetto }\OperatorTok{=}\NormalTok{ MiaClasse(}\DecValTok{10}\NormalTok{)  }\CommentTok{\# Identificatore per un\textquotesingle{}istanza}
\end{Highlighting}
\end{Shaded}

\subsection{Parole chiave}\label{parole-chiave}

Le parole chiave sono parole che non possono essere usate per scopi
diversi da quelli predefiniti nel linguaggio e, quindi, non possono
essere usate come identificatori. Ad esempio, \VERB|\VariableTok{True}|
che rappresenta il valore logico di verità, non può essere usato per
definire ad esempio una variabile.

Esistono anche delle parole chiave contestuali, cioè che sono tali solo
in alcuni contesti ed altrove possono essee usate come identificatori.
Usiamo il codice seguente per ottenere una lista di parole chiave e
parole chiave contestuali:

\begin{Shaded}
\begin{Highlighting}[]
\ImportTok{import}\NormalTok{ keyword}

\CommentTok{\# Si ottiene la lista delle parole chiave}
\NormalTok{parole\_chiave }\OperatorTok{=}\NormalTok{ keyword.kwlist}

\CommentTok{\# Si ottiene la lista delle parole chiave contestuali}
\NormalTok{parole\_chiave\_contestuale }\OperatorTok{=}\NormalTok{ keyword.softkwlist}

\CommentTok{\# Stampiamo la lista delle parole chiave}
\BuiltInTok{print}\NormalTok{(parole\_chiave)}

\CommentTok{\# Stampiamo la delle parole chiave contestuali}
\BuiltInTok{print}\NormalTok{(parole\_chiave\_contestuale)}
\end{Highlighting}
\end{Shaded}

Nella tabella seguente invece un elenco completo con breve descrizione:

\begin{longtable}[]{@{}
  >{\raggedright\arraybackslash}p{(\columnwidth - 2\tabcolsep) * \real{0.1852}}
  >{\raggedright\arraybackslash}p{(\columnwidth - 2\tabcolsep) * \real{0.8148}}@{}}
\toprule\noalign{}
\begin{minipage}[b]{\linewidth}\raggedright
Parola chiave
\end{minipage} & \begin{minipage}[b]{\linewidth}\raggedright
Descrizione
\end{minipage} \\
\midrule\noalign{}
\endhead
\bottomrule\noalign{}
\endlastfoot
Valori booleani & \\
\VERB|\VariableTok{False}| & Valore booleano falso \\
\VERB|\VariableTok{True}| & Valore booleano vero \\
Operatori logici & \\
\VERB|\KeywordTok{and}| & Operatore logico AND \\
\VERB|\KeywordTok{or}| & Operatore logico OR \\
\VERB|\KeywordTok{not}| & Operatore logico NOT \\
Operatori di controllo di flusso & \\
\VERB|\ControlFlowTok{if}| & Utilizzato per creare un'istruzione
condizionale \\
\VERB|\ControlFlowTok{elif}| & Utilizzato per aggiungere condizioni in
un blocco if \\
\VERB|\ControlFlowTok{else}| & Utilizzato per specificare il blocco di
codice da eseguire se le condizioni precedenti sono false \\
\VERB|\ControlFlowTok{for}| & Utilizzato per creare un ciclo for \\
\VERB|\ControlFlowTok{while}| & Utilizzato per creare un ciclo while \\
\VERB|\ControlFlowTok{break}| & Interrompe il ciclo in corso \\
\VERB|\ControlFlowTok{continue}| & Salta l'iterazione corrente del ciclo
e passa alla successiva \\
\VERB|\ControlFlowTok{pass}| & Indica un blocco di codice vuoto \\
\VERB|\ControlFlowTok{return}| & Utilizzato per restituire un valore da
una funzione \\
Gestione delle eccezioni & \\
\VERB|\ControlFlowTok{try}| & Utilizzato per definire un blocco di
codice da eseguire e gestire le eccezioni \\
\VERB|\ControlFlowTok{except}| & Utilizzato per catturare le eccezioni
in un blocco try-except \\
\VERB|\ControlFlowTok{finally}| & Blocco di codice che viene eseguito
alla fine di un blocco try, indipendentemente dal fatto che si sia
verificata un'eccezione \\
\VERB|\ControlFlowTok{raise}| & Utilizzato per sollevare un'eccezione \\
Definizione delle funzioni e classi & \\
\VERB|\KeywordTok{def}| & Utilizzato per definire una funzione \\
\VERB|\KeywordTok{class}| & Utilizzato per definire una classe \\
\VERB|\KeywordTok{lambda}| & Utilizzato per creare funzioni anonime \\
Gestione contesto di dichiarazione di variabili & \\
\VERB|\KeywordTok{global}| & Utilizzato per dichiarare variabili
globali \\
\VERB|\KeywordTok{nonlocal}| & Utilizzato per dichiarare variabili non
locali \\
Operazioni su moduli & \\
\VERB|\ImportTok{import}| & Utilizzato per importare moduli \\
\VERB|\ImportTok{from}| & Utilizzato per importare specifici elementi da
un modulo \\
\VERB|\ImportTok{as}| & Utilizzato per creare alias, ad esempio negli
import \\
Operatori di identità e appartenenza & \\
\VERB|\KeywordTok{in}| & Utilizzato per verificare se un valore esiste
in una sequenza \\
\VERB|\KeywordTok{is}| & Operatore di confronto di identità \\
Gestione delle risorse & \\
\VERB|\ControlFlowTok{with}| & Utilizzato per garantire un'azione di
pulizia come il rilascio delle risorse \\
Programmazione asincrona & \\
\VERB|\ControlFlowTok{async}| & Utilizzato per definire funzioni
asincrone \\
\VERB|\ControlFlowTok{await}| & Utilizzato per attendere un risultato in
una funzione asincrona \\
Varie & \\
\VERB|\KeywordTok{del}| & Utilizzato per eliminare oggetti \\
\VERB|\ControlFlowTok{assert}| & Utilizzato per le asserzioni, verifica
che un'espressione sia vera \\
\VERB|\ControlFlowTok{yield}| & Utilizzato per restituire un generatore
da una funzione \\
\VERB|\VariableTok{None}| & Rappresenta l'assenza di valore o un valore
nullo \\
Parole chiave contestuali & \\
\VERB|\NormalTok{match}| & Utilizzato nell'istruzione
\VERB|\NormalTok{match}| per il pattern matching \\
\VERB|\NormalTok{case}| & Utilizzato nell'istruzione
\VERB|\NormalTok{match}| per definire un ramo \\
\VERB|\NormalTok{\_}| & Utilizzato come identificatore speciale
nell'istruzione \VERB|\NormalTok{match}| per indicare un pattern di
default o ignorare valori \\
\VERB|\BuiltInTok{type}| & Utilizzato in specifici contesti per
dichiarazioni di tipo \\
\end{longtable}

Esempi di uso di parole chiave contestuali:

\begin{itemize}
\item
  \VERB|\NormalTok{match}|, \VERB|\NormalTok{case}| e
  \VERB|\NormalTok{\_}|:

\begin{Shaded}
\begin{Highlighting}[]
\CommentTok{\# Definiamo una funzione che utilizza il pattern matching}
\KeywordTok{def}\NormalTok{ process\_value(value):}
    \ControlFlowTok{match}\NormalTok{ value:}
        \ControlFlowTok{case} \DecValTok{1}\NormalTok{:}
            \BuiltInTok{print}\NormalTok{(}\StringTok{"Uno"}\NormalTok{)}
        \ControlFlowTok{case} \DecValTok{2}\NormalTok{:}
            \BuiltInTok{print}\NormalTok{(}\StringTok{"Due"}\NormalTok{)}
        \ControlFlowTok{case}\NormalTok{ \_:}
            \BuiltInTok{print}\NormalTok{(}\StringTok{"Altro"}\NormalTok{)}

\CommentTok{\# Utilizzo di \textasciigrave{}match\textasciigrave{} come identificatore per una variabile}
\NormalTok{match }\OperatorTok{=} \StringTok{"Questo è un identificatore valido"}

\CommentTok{\# Test della funzione}
\NormalTok{process\_value(}\DecValTok{1}\NormalTok{)  }\CommentTok{\# Output: Uno}
\NormalTok{process\_value(}\DecValTok{2}\NormalTok{)  }\CommentTok{\# Output: Due}
\NormalTok{process\_value(}\DecValTok{3}\NormalTok{)  }\CommentTok{\# Output: Altro}

\CommentTok{\# Stampa della variabile \textasciigrave{}match\textasciigrave{}}
\BuiltInTok{print}\NormalTok{(match)  }\CommentTok{\# Output: Questo è un identificatore valido}
\end{Highlighting}
\end{Shaded}
\item
  \VERB|\BuiltInTok{type}|:

\begin{Shaded}
\begin{Highlighting}[]
\ImportTok{from}\NormalTok{ typing }\ImportTok{import}\NormalTok{ TypeAlias}

\CommentTok{\# Dichiarazione di un alias di tipo}
\BuiltInTok{type}\NormalTok{ Point }\OperatorTok{=} \BuiltInTok{tuple}\NormalTok{[}\BuiltInTok{float}\NormalTok{, }\BuiltInTok{float}\NormalTok{]}

\CommentTok{\# Utilizzo dell\textquotesingle{}alias di tipo}
\KeywordTok{def}\NormalTok{ distanza(p1: Point, p2: Point) }\OperatorTok{{-}\textgreater{}} \BuiltInTok{float}\NormalTok{:}
    \ControlFlowTok{return}\NormalTok{ ((p1[}\DecValTok{0}\NormalTok{] }\OperatorTok{{-}}\NormalTok{ p2[}\DecValTok{0}\NormalTok{]) }\OperatorTok{**} \DecValTok{2} \OperatorTok{+}\NormalTok{ (p1[}\DecValTok{1}\NormalTok{] }\OperatorTok{{-}}\NormalTok{ p2[}\DecValTok{1}\NormalTok{]) }\OperatorTok{**} \DecValTok{2}\NormalTok{) }\OperatorTok{**} \FloatTok{0.5}

\CommentTok{\# Test della funzione con alias di tipo}
\NormalTok{punto1: Point }\OperatorTok{=}\NormalTok{ (}\FloatTok{1.0}\NormalTok{, }\FloatTok{2.0}\NormalTok{) }
\NormalTok{punto2: Point }\OperatorTok{=}\NormalTok{ (}\FloatTok{4.0}\NormalTok{, }\FloatTok{6.0}\NormalTok{)}

\BuiltInTok{print}\NormalTok{(distanza(punto1, punto2))  }\CommentTok{\# Output: 5.0}
\end{Highlighting}
\end{Shaded}
\end{itemize}

\subsection{Classi riservate di
identificatori}\label{classi-riservate-di-identificatori}

Alcune classi di identificatori (oltre alle parole chiave) hanno
significati speciali in Python. Queste classi sono identificate dai
pattern di caratteri di sottolineatura (underscore) all'inizio e alla
fine dei nomi. Tuttavia, l'uso di questi identificatori non impone
limitazioni rigide al programmatore, ma è importante seguire le
convenzioni per evitare ambiguità e problemi di compatibilità.

\begin{itemize}
\tightlist
\item
  \texttt{\_}

  \begin{itemize}
  \tightlist
  \item
    Non importato da \texttt{from\ module\ import\ *}: Gli
    identificatori che iniziano con un singolo underscore non vengono
    importati con un'istruzione di importazione globale.
  \item
    Pattern nei match: Nel contesto di un pattern di corrispondenza
    all'interno di un'istruzione \texttt{match}, \texttt{\_} è una soft
    keyword che denota un wildcard (carattere jolly).
  \item
    Interprete interattivo: L'interprete interattivo rende disponibile
    il risultato dell'ultima valutazione nella variabile \texttt{\_}.
    Questo risultato è memorizzato nel modulo builtins, insieme alle
    funzioni built-in come \texttt{print}.
  \item
    Altro uso: Altrove, \texttt{\_} è un identificatore regolare. Viene
    spesso usato per nominare elementi ``speciali'', ma non è speciale
    per Python stesso. Il nome \texttt{\_} è comunemente usato in
    congiunzione con l'internazionalizzazione (vedi la documentazione
    del modulo gettext per ulteriori informazioni su questa convenzione)
    ed è anche comunemente utilizzato per variabili non usate.
  \end{itemize}
\end{itemize}

Esempio:

\begin{Shaded}
\begin{Highlighting}[]
\CommentTok{\# Utilizzo di \_ come identificatore in vari contesti}

\CommentTok{\# Non importato con from module import *}
\NormalTok{\_private\_variable }\OperatorTok{=} \StringTok{"Questa variabile non verrà importata con \textquotesingle{}from module import *\textquotesingle{}"}

\CommentTok{\# Utilizzo di \_ come wildcard in un\textquotesingle{}istruzione match}
\KeywordTok{def}\NormalTok{ process\_value(value):}
    \ControlFlowTok{match}\NormalTok{ value:}
        \ControlFlowTok{case} \DecValTok{1}\NormalTok{:}
            \BuiltInTok{print}\NormalTok{(}\StringTok{"Uno"}\NormalTok{)}
        \ControlFlowTok{case}\NormalTok{ \_:}
            \BuiltInTok{print}\NormalTok{(}\StringTok{"Altro"}\NormalTok{)}

\NormalTok{process\_value(}\DecValTok{2}\NormalTok{)  }\CommentTok{\# Output: Altro}

\CommentTok{\# Uso di \_ nell\textquotesingle{}interprete interattivo}
\NormalTok{result }\OperatorTok{=} \DecValTok{5} \OperatorTok{+} \DecValTok{3}
\BuiltInTok{print}\NormalTok{(\_)  }\CommentTok{\# Output: 8 (nell\textquotesingle{}interprete interattivo)}

\CommentTok{\# Uso di \_ come variabile regolare}
\NormalTok{\_ }\OperatorTok{=} \StringTok{"Valore non usato"}

\CommentTok{\# Uso di \_ per internazionalizzazione}
\ImportTok{import}\NormalTok{ gettext}
\NormalTok{gettext.install(}\StringTok{\textquotesingle{}myapplication\textquotesingle{}}\NormalTok{)}
\BuiltInTok{print}\NormalTok{(\_(}\StringTok{\textquotesingle{}Hello, world\textquotesingle{}}\NormalTok{))}
\end{Highlighting}
\end{Shaded}

\begin{itemize}
\tightlist
\item
  \texttt{\_\_*\_\_}

  \begin{itemize}
  \tightlist
  \item
    Nomi definiti dal sistema: Questi nomi, informalmente noti come nomi
    ``dunder'', sono definiti dall'interprete e dalla sua
    implementazione (inclusa la libreria standard). I nomi di sistema
    attuali sono discussi nella sezione dei nomi dei metodi speciali e
    altrove. Altri potrebbero essere definiti nelle versioni future di
    Python. Qualsiasi uso di nomi \texttt{\_\_*\_\_}, in qualsiasi
    contesto, che non segua l'uso esplicitamente documentato, è soggetto
    a rottura senza preavviso.
  \end{itemize}
\end{itemize}

\subsubsection{Esempio}\label{esempio-1}

\begin{Shaded}
\begin{Highlighting}[]
\CommentTok{\# Utilizzo di nomi "dunder"}
\KeywordTok{class}\NormalTok{ MyClass:}
    \KeywordTok{def} \FunctionTok{\_\_init\_\_}\NormalTok{(}\VariableTok{self}\NormalTok{, value):}
        \VariableTok{self}\NormalTok{.\_\_value }\OperatorTok{=}\NormalTok{ value  }\CommentTok{\# Questo è un nome "dunder" per un attributo privato}
    
    \KeywordTok{def} \FunctionTok{\_\_str\_\_}\NormalTok{(}\VariableTok{self}\NormalTok{):}
        \ControlFlowTok{return} \SpecialStringTok{f"MyClass con valore }\SpecialCharTok{\{}\VariableTok{self}\SpecialCharTok{.}\NormalTok{\_\_value}\SpecialCharTok{\}}\SpecialStringTok{"}

\NormalTok{obj }\OperatorTok{=}\NormalTok{ MyClass(}\DecValTok{10}\NormalTok{)}
\BuiltInTok{print}\NormalTok{(obj)  }\CommentTok{\# Output: MyClass con valore 10}
\end{Highlighting}
\end{Shaded}

\begin{itemize}
\tightlist
\item
  \texttt{\_\_*}

  \begin{itemize}
  \tightlist
  \item
    Nomi privati della classe: I nomi in questa categoria, quando
    utilizzati nel contesto di una definizione di classe, vengono
    riscritti per utilizzare una forma mangled per evitare conflitti di
    nome tra attributi ``privati'' delle classi base e derivate.
  \end{itemize}
\end{itemize}

\subsubsection{Esempio}\label{esempio-2}

\begin{Shaded}
\begin{Highlighting}[]
\CommentTok{\# Utilizzo di nomi privati della classe}
\KeywordTok{class}\NormalTok{ BaseClass:}
    \KeywordTok{def} \FunctionTok{\_\_init\_\_}\NormalTok{(}\VariableTok{self}\NormalTok{):}
        \VariableTok{self}\NormalTok{.\_\_private\_attr }\OperatorTok{=} \StringTok{"Base"}

\KeywordTok{class}\NormalTok{ DerivedClass(BaseClass):}
    \KeywordTok{def} \FunctionTok{\_\_init\_\_}\NormalTok{(}\VariableTok{self}\NormalTok{):}
        \BuiltInTok{super}\NormalTok{().}\FunctionTok{\_\_init\_\_}\NormalTok{()}
        \VariableTok{self}\NormalTok{.\_\_private\_attr }\OperatorTok{=} \StringTok{"Derived"}

\NormalTok{base\_obj }\OperatorTok{=}\NormalTok{ BaseClass()}
\NormalTok{derived\_obj }\OperatorTok{=}\NormalTok{ DerivedClass()}

\CommentTok{\# Accesso ai nomi riscritti (name mangling)}
\BuiltInTok{print}\NormalTok{(base\_obj.\_BaseClass\_\_private\_attr)  }\CommentTok{\# Output: Base}
\BuiltInTok{print}\NormalTok{(derived\_obj.\_DerivedClass\_\_private\_attr)  }\CommentTok{\# Output: Derived}
\end{Highlighting}
\end{Shaded}

In sintesi, Python riserva certe classi di identificatori che hanno
significati speciali e seguono regole specifiche, principalmente per
garantire l'integrità del codice e la compatibilità tra versioni diverse
del linguaggio. Tuttavia, queste convenzioni non pongono limitazioni
rigide al programmatore, che può comunque utilizzare questi
identificatori secondo necessità, purché consapevole delle loro
implicazioni.

\subsection{Operatori}\label{operatori}

Gli operatori sono rappresentati da simboli non alfanumerici e, quando
applicati a uno o più identificatori, letterali o espressioni (definiti
genericamente operandi), producono un risultato. Attenzione a non
confondere la definizione di operatore come token, come considerata qui,
con quella di operatore come funzionalità algoritmica, poiché alcune
parole chiave sono operatori algoritmici e anche le funzioni possono
agire come operatori.

Esempi:

\begin{Shaded}
\begin{Highlighting}[]
\NormalTok{x }\OperatorTok{=} \DecValTok{5}
\NormalTok{y }\OperatorTok{=} \DecValTok{10}
\NormalTok{z }\OperatorTok{=}\NormalTok{ x }\OperatorTok{+}\NormalTok{ y }\CommentTok{\# Utilizza l\textquotesingle{}operatore + sugli identificatori x e y}

\BuiltInTok{sum} \OperatorTok{=} \DecValTok{3} \OperatorTok{+} \DecValTok{4} \CommentTok{\# Utilizza l\textquotesingle{}operatore + su letterali}

\NormalTok{result }\OperatorTok{=}\NormalTok{ (x }\OperatorTok{*}\NormalTok{ y) }\OperatorTok{+}\NormalTok{ (z }\OperatorTok{/} \DecValTok{2}\NormalTok{) }\CommentTok{\# Utilizza vari operatori su espressioni}
\end{Highlighting}
\end{Shaded}

In tabella l'elenco degli operatori:

\begin{longtable}[]{@{}
  >{\raggedright\arraybackslash}p{(\columnwidth - 4\tabcolsep) * \real{0.2222}}
  >{\raggedright\arraybackslash}p{(\columnwidth - 4\tabcolsep) * \real{0.1667}}
  >{\raggedright\arraybackslash}p{(\columnwidth - 4\tabcolsep) * \real{0.6111}}@{}}
\toprule\noalign{}
\begin{minipage}[b]{\linewidth}\raggedright
Tipo di operatore
\end{minipage} & \begin{minipage}[b]{\linewidth}\raggedright
Operatore
\end{minipage} & \begin{minipage}[b]{\linewidth}\raggedright
Descrizione
\end{minipage} \\
\midrule\noalign{}
\endhead
\bottomrule\noalign{}
\endlastfoot
Aritmetici & \VERB|\OperatorTok{+}| & Addizione \\
& \VERB|\OperatorTok{{-}}| & Sottrazione \\
& \VERB|\OperatorTok{*}| & Moltiplicazione \\
& \VERB|\OperatorTok{/}| & Divisione \\
& \VERB|\OperatorTok{//}| & Divisione intera \\
& \VERB|\OperatorTok{\%}| & Modulo \\
& \VERB|\OperatorTok{**}| & Esponenziazione \\
& \VERB|\OperatorTok{@}| & Matrice (operatore di moltiplicazione) \\
Confronto & \VERB|\OperatorTok{\textless{}}| & Minore \\
& \VERB|\OperatorTok{\textgreater{}}| & Maggiore \\
& \VERB|\OperatorTok{\textless{}=}| & Minore o uguale \\
& \VERB|\OperatorTok{\textgreater{}=}| & Maggiore o uguale \\
& \VERB|\OperatorTok{==}| & Uguale \\
& \VERB|\OperatorTok{!=}| & Diverso \\
Bitwise & \VERB|\OperatorTok{\&}| & AND bit a bit \\
& \VERB|\OperatorTok{\VerbBar{}}| & OR bit a bit \\
& \VERB|\OperatorTok{\^{}}| & XOR bit a bit \\
& \VERB|\OperatorTok{\textasciitilde{}}| & NOT bit a bit \\
& \VERB|\OperatorTok{\textless{}\textless{}}| & Shift a sinistra \\
& \VERB|\OperatorTok{\textgreater{}\textgreater{}}| & Shift a destra \\
Assegnazione & \VERB|\OperatorTok{:=}| & Operatore di assegnazione in
espressione (walrus o tricheco) \\
\end{longtable}

Esempio su \VERB|\OperatorTok{@}| che illustra un aspetto importante: il
comportamento degli operatori può (o meglio, deve) essere definito
quando si creano dei tipi di oggetto. Infatti, nel codice seguente, è
definita una matrice assieme a una delle operazioni matematiche più
comuni che è la moltiplicazione, implementata per mezzo di
\VERB|\FunctionTok{\_\_matmul\_\_}|:

\begin{Shaded}
\begin{Highlighting}[]
\KeywordTok{class}\NormalTok{ Matrice:}
    \KeywordTok{def} \FunctionTok{\_\_init\_\_}\NormalTok{(}\VariableTok{self}\NormalTok{, righe):}
        \VariableTok{self}\NormalTok{.righe }\OperatorTok{=}\NormalTok{ righe}
        \VariableTok{self}\NormalTok{.num\_righe }\OperatorTok{=} \BuiltInTok{len}\NormalTok{(righe)}
        \VariableTok{self}\NormalTok{.num\_colonne }\OperatorTok{=} \BuiltInTok{len}\NormalTok{(righe[}\DecValTok{0}\NormalTok{]) }\ControlFlowTok{if}\NormalTok{ righe }\ControlFlowTok{else} \DecValTok{0}

    \KeywordTok{def} \FunctionTok{\_\_matmul\_\_}\NormalTok{(}\VariableTok{self}\NormalTok{, altra):}
        \CommentTok{\# Controlla se le dimensioni sono compatibili per la moltiplicazione}
        \ControlFlowTok{if} \VariableTok{self}\NormalTok{.num\_colonne }\OperatorTok{!=}\NormalTok{ altra.num\_righe:}
            \ControlFlowTok{raise} \PreprocessorTok{ValueError}\NormalTok{(}\StringTok{"Non è possibile moltiplicare le matrici: "}
                             \StringTok{"dimensioni incompatibili."}\NormalTok{)}
        
        \CommentTok{\# Inizializza la matrice risultato con zeri}
\NormalTok{        risultato }\OperatorTok{=}\NormalTok{ [[}\DecValTok{0} \ControlFlowTok{for}\NormalTok{ \_ }\KeywordTok{in} \BuiltInTok{range}\NormalTok{(altra.num\_colonne)]}
                     \ControlFlowTok{for}\NormalTok{ \_ }\KeywordTok{in} \BuiltInTok{range}\NormalTok{(}\VariableTok{self}\NormalTok{.num\_righe)]}
        
        \CommentTok{\# Esegue la moltiplicazione delle matrici}
        \ControlFlowTok{for}\NormalTok{ i }\KeywordTok{in} \BuiltInTok{range}\NormalTok{(}\VariableTok{self}\NormalTok{.num\_righe):}
            \ControlFlowTok{for}\NormalTok{ j }\KeywordTok{in} \BuiltInTok{range}\NormalTok{(altra.num\_colonne):}
                \ControlFlowTok{for}\NormalTok{ k }\KeywordTok{in} \BuiltInTok{range}\NormalTok{(}\VariableTok{self}\NormalTok{.num\_colonne):}
\NormalTok{                    risultato[i][j] }\OperatorTok{+=}\NormalTok{ (}\VariableTok{self}\NormalTok{.righe[i][k] }\OperatorTok{*}
\NormalTok{                                        altra.righe[k][j])}
        
        \ControlFlowTok{return}\NormalTok{ Matrice(risultato)}

    \KeywordTok{def} \FunctionTok{\_\_repr\_\_}\NormalTok{(}\VariableTok{self}\NormalTok{):}
        \CommentTok{\# Rappresentazione leggibile della matrice}
        \ControlFlowTok{return} \StringTok{\textquotesingle{}}\CharTok{\textbackslash{}n}\StringTok{\textquotesingle{}}\NormalTok{.join([}\StringTok{\textquotesingle{} \textquotesingle{}}\NormalTok{.join(}\BuiltInTok{map}\NormalTok{(}\BuiltInTok{str}\NormalTok{, riga)) }\ControlFlowTok{for}\NormalTok{ riga }\KeywordTok{in} \VariableTok{self}\NormalTok{.righe])}

\CommentTok{\# Definizione di due matrici}
\NormalTok{A }\OperatorTok{=}\NormalTok{ Matrice([[}\DecValTok{1}\NormalTok{, }\DecValTok{2}\NormalTok{], [}\DecValTok{3}\NormalTok{, }\DecValTok{4}\NormalTok{]])}
\NormalTok{B }\OperatorTok{=}\NormalTok{ Matrice([[}\DecValTok{5}\NormalTok{, }\DecValTok{6}\NormalTok{], [}\DecValTok{7}\NormalTok{, }\DecValTok{8}\NormalTok{]])}

\CommentTok{\# Moltiplicazione di matrici utilizzando l\textquotesingle{}operatore @}
\NormalTok{C }\OperatorTok{=}\NormalTok{ A }\OperatorTok{@}\NormalTok{ B}

\BuiltInTok{print}\NormalTok{(}\StringTok{"Matrice A:"}\NormalTok{)}
\BuiltInTok{print}\NormalTok{(A)}

\BuiltInTok{print}\NormalTok{(}\StringTok{"Matrice B:"}\NormalTok{)}
\BuiltInTok{print}\NormalTok{(B)}

\BuiltInTok{print}\NormalTok{(}\StringTok{"Risultato di A @ B:"}\NormalTok{)}
\BuiltInTok{print}\NormalTok{(C)}
\end{Highlighting}
\end{Shaded}

Infine, \VERB|\OperatorTok{@}| è anche un delimitatore.

\subsection{Delimitatori}\label{delimitatori}

In Python, alcuni token servono come delimitatori nella grammatica del
linguaggio. I delimitatori sono caratteri che separano le varie
componenti del codice, come espressioni, blocchi di codice, parametri di
funzioni e istruzioni.

La seguente tabella include tutti i delimitatori e i principali
utilizzi:

\begin{longtable}[]{@{}
  >{\raggedright\arraybackslash}p{(\columnwidth - 2\tabcolsep) * \real{0.2619}}
  >{\raggedright\arraybackslash}p{(\columnwidth - 2\tabcolsep) * \real{0.7381}}@{}}
\toprule\noalign{}
\begin{minipage}[b]{\linewidth}\raggedright
Delimitatore
\end{minipage} & \begin{minipage}[b]{\linewidth}\raggedright
Descrizione
\end{minipage} \\
\midrule\noalign{}
\endhead
\bottomrule\noalign{}
\endlastfoot
\VERB|\NormalTok{(}| & Utilizzata per raggruppare espressioni, chiamate
di funzione e definizioni di tupla \\
\VERB|\NormalTok{)}| & Utilizzata per chiudere le parentesi tonde
aperte \\
\VERB|\NormalTok{[}| & Utilizzate per definire liste e accedere agli
elementi delle liste, tuple, o stringhe \\
\VERB|\NormalTok{]}| & Utilizzate per chiudere le parentesi quadre
aperte \\
\VERB|\NormalTok{\{}| & Utilizzate per definire dizionari e set \\
\VERB|\NormalTok{\}}| & Utilizzate per chiudere le parentesi graffe
aperte \\
\VERB|\NormalTok{,}| & Utilizzata per separare elementi in liste, tuple,
e argomenti nelle chiamate di funzione \\
\VERB|\NormalTok{:}| & Utilizzato per definire blocchi di codice (come
in \VERB|\ControlFlowTok{if}|, \VERB|\ControlFlowTok{for}|,
\VERB|\ControlFlowTok{while}|, \VERB|\KeywordTok{def}|,
\VERB|\KeywordTok{class}|) e per gli slice \\
\VERB|\NormalTok{.}| & Utilizzato per accedere agli attributi di un
oggetto. Può apparire in letterali decimnali e immaginari \\
\VERB|\OperatorTok{;}| & Utilizzato per separare istruzioni multiple
sulla stessa riga \\
\VERB|\OperatorTok{@}| & Utilizzato per dichiarare decoratori per
funzioni e metodi \\
\VERB|\OperatorTok{=}| & Operatore utilizzato per assegnare valori a
variabili \\
\VERB|\OperatorTok{{-}\textgreater{}}| & Annotazione del tipo di ritorno
delle funzioni \\
\VERB|\OperatorTok{+=}| & Assegnazione aumentata con addizione. Aggiunge
il valore a destra a quello a sinistra e assegna il risultato alla
variabile a sinistra. Come i successivi, è sia un delimitatore che un
operatore \\
\VERB|\OperatorTok{{-}=}| & Assegnazione aumentata con sottrazione \\
\VERB|\OperatorTok{*=}| & Assegnazione aumentata con moltiplicazione \\
\VERB|\OperatorTok{/=}| & Assegnazione aumentata con divisione \\
\VERB|\OperatorTok{//=}| & Assegnazione aumentata con divisione
intera \\
\VERB|\OperatorTok{\%=}| & Assegnazione aumentata con modulo \\
\VERB|\OperatorTok{@=}| & Assegnazione aumentata con moltiplicazione di
matrici \\
\VERB|\OperatorTok{\&=}| & Assegnazione aumentata con AND bit a bit \\
\VERB|\OperatorTok{\VerbBar{}=}| & Assegnazione aumentata con OR bit a
bit \\
\VERB|\OperatorTok{\^{}=}| & Assegnazione aumentata con XOR bit a bit \\
\VERB|\OperatorTok{\textgreater{}\textgreater{}=}| & Assegnazione
aumentata con shift a destra \\
\VERB|\OperatorTok{\textless{}\textless{}=}| & Assegnazione aumentata
con shift a sinistra \\
\VERB|\OperatorTok{**=}| & Assegnazione aumentata con esponenziazione \\
\end{longtable}

Una sequenza di tre punti, comunemente indicata come ellissi anche al
difuori dei linguaggi di
programmazione,\href{L'ellissi\%20è\%20usata,\%20ad\%20esempio,\%20in\%20C\%20per\%20dichiarare\%20funzioni\%20che\%20accettano\%20un\%20numero\%20variaible\%20di\%20parametri\%20e\%20i\%20Javascript\%20come\%20operatore\%20per\%20espandere\%20gli\%20array\%20o\%20le\%20proprietà\%20di\%20un\%20oggetto.}{1\^{}}
è trattata come un token a sé e corrisponde ad un oggetto predefinito
chiamato
\href{https://docs.python.org/3/reference/datamodel.html\#ellipsis}{Ellipsis},
con applicazioni in diversi contesti:

\phantomsection\label{annotated-cell-43}%
\begin{Shaded}
\begin{Highlighting}[]
\BuiltInTok{print}\NormalTok{(}\BuiltInTok{type}\NormalTok{(...))  }\CommentTok{\# \textless{}class \textquotesingle{}ellipsis\textquotesingle{}\textgreater{} }\hspace*{\fill}\NormalTok{\circled{1}}

\KeywordTok{def}\NormalTok{ funzione\_da\_completare():}
\NormalTok{    ... }\hspace*{\fill}\NormalTok{\circled{2}}

\KeywordTok{class}\NormalTok{ ClasseEsempio:}
    \KeywordTok{def}\NormalTok{ metodo\_da\_completare(}\VariableTok{self}\NormalTok{):}
\NormalTok{        ...}

\ImportTok{from}\NormalTok{ typing }\ImportTok{import}\NormalTok{ Callable}

\KeywordTok{def}\NormalTok{ funzione\_variadica(func: Callable[..., }\BuiltInTok{int}\NormalTok{]): }\hspace*{\fill}\NormalTok{\circled{3}}
    \ControlFlowTok{pass}

\ImportTok{import}\NormalTok{ numpy }\ImportTok{as}\NormalTok{ np}

\NormalTok{array }\OperatorTok{=}\NormalTok{ np.array([[[}\DecValTok{1}\NormalTok{, }\DecValTok{2}\NormalTok{, }\DecValTok{3}\NormalTok{], [}\DecValTok{4}\NormalTok{, }\DecValTok{5}\NormalTok{, }\DecValTok{6}\NormalTok{]], [[}\DecValTok{7}\NormalTok{, }\DecValTok{8}\NormalTok{, }\DecValTok{9}\NormalTok{], [}\DecValTok{10}\NormalTok{, }\DecValTok{11}\NormalTok{, }\DecValTok{12}\NormalTok{]]]) }

\BuiltInTok{print}\NormalTok{(array[..., }\DecValTok{1}\NormalTok{]) }\hspace*{\fill}\NormalTok{\circled{4}}
\CommentTok{\# Output: }
\CommentTok{\# [[ 2  5]}
\CommentTok{\#  [ 8 11]]}
\end{Highlighting}
\end{Shaded}

\begin{description}
\tightlist
\item[\circled{1}]
Otteniamo il tipo dell'oggetto ellissi.
\item[\circled{2}]
Utilizzo come segnaposto per indicare che la funzione è da completare.
Da notare che chiamare la funzione
\VERB|\NormalTok{funzione\_da\_completare()}| non dà errore.
\item[\circled{3}]
L'uso di \VERB|\NormalTok{Callable[..., }\BuiltInTok{int}\NormalTok{]}|
indica una funzione che può accettare un numero variabile di argomenti
di qualsiasi tipo e restituire un valore di tipo int.
\item[\circled{4}]
\VERB|\NormalTok{numpy}| è una libreria di calcolo matriciale molto
diffusa. L'ellissi è utilizzata per effettuare uno sezione complessa
della matrice secondo tutte le dimensioni precedenti all'ultima. In
altre parole, l'ellissi permette di selezionare interamente tutte le
dimensioni tranne l'ultima specificata.
\end{description}

Alcuni caratteri ASCII hanno un significato speciale come parte di altri
token o sono significativi per l'analizzatore lessicale:

\begin{longtable}[]{@{}
  >{\raggedright\arraybackslash}p{(\columnwidth - 2\tabcolsep) * \real{0.2706}}
  >{\raggedright\arraybackslash}p{(\columnwidth - 2\tabcolsep) * \real{0.7294}}@{}}
\toprule\noalign{}
\begin{minipage}[b]{\linewidth}\raggedright
Carattere
\end{minipage} & \begin{minipage}[b]{\linewidth}\raggedright
Descrizione
\end{minipage} \\
\midrule\noalign{}
\endhead
\bottomrule\noalign{}
\endlastfoot
\VERB|\CommentTok{\textquotesingle{}}| & Utilizzato per definire
stringhe di caratteri. \\
\VERB|\CommentTok{"}| & Utilizzato per definire stringhe di
caratteri. \\
\VERB|\CommentTok{\#}| & Simbolo di commento. Utilizzato per indicare un
commento, che viene ignorato dall'interprete Python. \\
\VERB|\OperatorTok{\textbackslash{}}| & Backslash. Utilizzato per
caratteri di escape nelle stringhe e per continuare le righe di codice
su più righe fisiche. \\
\end{longtable}

Alcuni caratteri ASCII non sono utilizzati in Python e la loro presenza
al difuori dei letterali di stringa e dei commenti genera un errore:
\texttt{\$}, \texttt{?}, \texttt{\textasciigrave{}}.

\subsection{Letterali}\label{letterali}

I letterali sono notazioni per valori costanti di alcuni tipi
predefiniti nel linguaggio. Esistono diversi tipi di letterali, ognuno
rappresenta un tipo di dato specifico e ha una sintassi particolare.

\subsubsection{Numerici}\label{numerici}

I letterali numerici includono interi, numeri a virgola mobile e numeri
complessi:

\begin{itemize}
\tightlist
\item
  Interi, possono essere scritti in base decimale, ottale, esadecimale o
  binaria:

  \begin{itemize}
  \tightlist
  \item
    Decimale: \VERB|\DecValTok{10}|,
    \VERB|\OperatorTok{{-}}\DecValTok{3}|.
  \item
    Ottale: \VERB|\BaseNTok{0o12}|,
    \VERB|\OperatorTok{{-}}\BaseNTok{0o7}|.
  \item
    Esadecimale: \VERB|\BaseNTok{0xA}|,
    \VERB|\OperatorTok{{-}}\BaseNTok{0x1F}|.
  \item
    Binario: \VERB|\BaseNTok{0b1010}|,
    \VERB|\OperatorTok{{-}}\BaseNTok{0b11}|.
  \end{itemize}
\item
  Virgola mobile, possono essere rappresentati con una parte intera e
  una decimale, oppure con notazione scientifica:

  \begin{itemize}
  \tightlist
  \item
    Virgola mobile: \VERB|\FloatTok{3.14}|,
    \VERB|\OperatorTok{{-}}\FloatTok{0.001}|.
  \item
    Notazione scientifica: \VERB|\FloatTok{1e10}|,
    \VERB|\OperatorTok{{-}}\FloatTok{2.5e{-}3}|.
  \end{itemize}
\item
  Complessi, appresentati da una parte reale e una parte immaginaria:
  \VERB|\DecValTok{3}\OperatorTok{+}\OtherTok{4j}|,
  \VERB|\OperatorTok{{-}}\DecValTok{1}\OperatorTok{{-}}\OtherTok{0.5j}|.
\end{itemize}

\subsubsection{Stringhe}\label{stringhe}

I letterali di stringa possono essere racchiusi tra virgolette singole o
doppie. Possono anche essere multi-linea se racchiusi tra triple
virgolette singole o doppie:

\begin{itemize}
\tightlist
\item
  Stringhe racchiuse tra virgolette singole o doppie:

  \begin{itemize}
  \tightlist
  \item
    Singole:
    \VERB|\CommentTok{\textquotesingle{}ciao\textquotesingle{}}|.
  \item
    Doppie: \VERB|\CommentTok{"mondo"}|.
  \end{itemize}
\item
  Stringhe multi-linea racchiuse tra triple virgolette singole o doppie:

  \begin{itemize}
  \tightlist
  \item
    Triple singole:
    \VERB|\CommentTok{\textquotesingle{}\textquotesingle{}\textquotesingle{}testo multi{-}linea\textquotesingle{}\textquotesingle{}\textquotesingle{}}|.
  \item
    Triple doppie: \VERB|\CommentTok{"""testo multi{-}linea"""}|.
  \end{itemize}
\end{itemize}

Le stringhe tra tripli apici possono avere degli a capo e degli apici
(non tripli) all'interno.

Esempio:

\begin{Shaded}
\begin{Highlighting}[]
\NormalTok{stringa\_multilinea }\OperatorTok{=} \StringTok{"""Questa è una stringa}
\StringTok{molto "importante"."""}

\BuiltInTok{print}\NormalTok{(stringa\_multilinea)}
\end{Highlighting}
\end{Shaded}

Tutte le stringhe sono codificate in Unicode, con il prefisso
\VERB|\NormalTok{b}| la stringa è di tipo byte ed è limitata ai 128
caratteri dell'ASCII. Se si prepone \VERB|\NormalTok{r}|, che sta per
\emph{raw} cioè grezzo, allora la codifica è sempre Unicode ma i
caratteri di escape non sono intepretati.\href{}{2\^{}}

\subsubsection{F-stringhe}\label{f-stringhe}

Le f-stringhe (stringhe formattate) sono racchiuse tra virgolette
singole, doppie o triple e sono precedute dal prefisso
\VERB|\NormalTok{f}| o \VERB|\NormalTok{F}|. Permettono di includere
espressioni Python all'interno.

Si possono avere stringhe formattate grezze ma non byte.

Esempio:

\begin{Shaded}
\begin{Highlighting}[]
\NormalTok{nome }\OperatorTok{=} \StringTok{"Python"}
\NormalTok{f\_stringa }\OperatorTok{=} \SpecialStringTok{f\textquotesingle{}Ciao, }\SpecialCharTok{\{}\NormalTok{nome}\SpecialCharTok{\}}\SpecialStringTok{!\textquotesingle{}}
\NormalTok{f\_stringa\_multi\_linea }\OperatorTok{=} \SpecialStringTok{f\textquotesingle{}\textquotesingle{}\textquotesingle{}Questo è}
\SpecialStringTok{un esempio di}
\SpecialStringTok{f{-}stringa multi{-}linea con }\SpecialCharTok{\{}\NormalTok{nome}\SpecialCharTok{\}}\SpecialStringTok{\textquotesingle{}\textquotesingle{}\textquotesingle{}}

\BuiltInTok{print}\NormalTok{(f\_stringa)}
\BuiltInTok{print}\NormalTok{(f\_stringa\_multi\_linea)}
\end{Highlighting}
\end{Shaded}

\section{Istruzioni}\label{istruzioni}

Un programma Pythom è una sequenza di istruzioni che si distinguono in
\textbf{semplici} o \textbf{composte}.

\subsection{Istruzioni semplici}\label{istruzioni-semplici-1}

Un'istruzione semplice è sempre contenuta in una riga logica, che può
presentare più istruzioni semplici separate da \VERB|\OperatorTok{;}|. È
permesso ma scaonsigliato perché in pochi casi porta a codice leggibile.

Un'espressione è una istruzione semplice ed, infatti, inserita nel REPL,
ne viene prodotto il risultato della valutazione. D'altronde, una
espressione è spesso utilizzata per chiamare funzioni che hanno effetti
collaterali, come, ad esempio, produrre un output:

\begin{itemize}
\item
  Produzione di output:

\begin{Shaded}
\begin{Highlighting}[]
\CommentTok{\# Funzione che stampa un messaggio}
\KeywordTok{def}\NormalTok{ stampa\_messaggio(messaggio):}
    \BuiltInTok{print}\NormalTok{(messaggio)}

\CommentTok{\# Istruzione di espressione che chiama la funzione }
\CommentTok{\# con un effetto collaterale (stampa del messaggio)}
\NormalTok{stampa\_messaggio(}\StringTok{"Ciao, mondo!"}\NormalTok{)}
\end{Highlighting}
\end{Shaded}
\item
  Modifica di parametri:

\begin{Shaded}
\begin{Highlighting}[]
\CommentTok{\# Funzione che modifica un argomento mutabile (lista)}
\KeywordTok{def}\NormalTok{ aggiungi\_elemento(lista, elemento):}
\NormalTok{    lista.append(elemento)}

\CommentTok{\# Lista iniziale}
\NormalTok{numeri }\OperatorTok{=}\NormalTok{ [}\DecValTok{1}\NormalTok{, }\DecValTok{2}\NormalTok{, }\DecValTok{3}\NormalTok{]}

\CommentTok{\# Istruzione di espressione che chiama la funzione }
\CommentTok{\# con un effetto collaterale (modifica dell\textquotesingle{}argomento)}
\NormalTok{aggiungi\_elemento(numeri, }\DecValTok{4}\NormalTok{)}

\BuiltInTok{print}\NormalTok{(numeri)  }\CommentTok{\# Output: [1, 2, 3, 4]}
\end{Highlighting}
\end{Shaded}
\item
  Modifica di variabili globali

\begin{Shaded}
\begin{Highlighting}[]
\CommentTok{\# Variabile globale}
\NormalTok{contatore }\OperatorTok{=} \DecValTok{0}

\CommentTok{\# Funzione che modifica una variabile globale}
\KeywordTok{def}\NormalTok{ incrementa\_contatore():}
    \KeywordTok{global}\NormalTok{ contatore}
\NormalTok{    contatore }\OperatorTok{+=} \DecValTok{1}

\CommentTok{\# Istruzione di espressione che chiama la funzione }
\CommentTok{\# con un effetto collaterale (modifica della variabile globale)}
\NormalTok{incrementa\_contatore()}

\BuiltInTok{print}\NormalTok{(contatore)  }\CommentTok{\# Output: 1}
\end{Highlighting}
\end{Shaded}
\item
  Lancio di eccezioni:

\begin{Shaded}
\begin{Highlighting}[]
\CommentTok{\# Funzione che solleva un\textquotesingle{}eccezione}
\KeywordTok{def}\NormalTok{ solleva\_eccezione(messaggio):}
    \ControlFlowTok{raise} \PreprocessorTok{ValueError}\NormalTok{(messaggio)}

\CommentTok{\# Istruzione di espressione che chiama la funzione con un effetto collaterale (sollevamento di un\textquotesingle{}eccezione)}
\ControlFlowTok{try}\NormalTok{:}
\NormalTok{    solleva\_eccezione(}\StringTok{"Qualcosa è andato storto!"}\NormalTok{)}
\ControlFlowTok{except} \PreprocessorTok{ValueError} \ImportTok{as}\NormalTok{ e:}
    \BuiltInTok{print}\NormalTok{(e)  }\CommentTok{\# Output: Qualcosa è andato storto!}
\end{Highlighting}
\end{Shaded}
\end{itemize}

Un assegnamento con \VERB|\OperatorTok{=}| è anch'esso un'istruzione
semplice e non può mai essere all'interno di una espressione, dove
invece si può usare l'operatore \emph{tricheco} \VERB|\OperatorTok{:=}|:

\begin{Shaded}
\begin{Highlighting}[]
\ControlFlowTok{if}\NormalTok{ (n }\OperatorTok{:=} \BuiltInTok{len}\NormalTok{(}\StringTok{"Python"}\NormalTok{)) }\OperatorTok{\textgreater{}} \DecValTok{5}\NormalTok{:}
  \BuiltInTok{print}\NormalTok{(}\SpecialStringTok{f"La lunghezza della stringa è }\SpecialCharTok{\{}\NormalTok{n}\SpecialCharTok{\}}\SpecialStringTok{"}\NormalTok{) }
  \CommentTok{\# Output: La lunghezza della stringa è 6}
\end{Highlighting}
\end{Shaded}

\subsection{Istruzioni composte}\label{istruzioni-composte}

Una istruzione composta è costituita da altre istruzioni (semplici o
composte). Il controllo dell'esecuzione delle istruzioni componenti
avviene per mezzo di una o più clausole che iniziano tutte con una
parola chiave, sono terminate da \VERB|\NormalTok{:}| e seguite da un
blocco di codice. Ogni blocco deve avere almeno una istruzione semplice,
ma può non avere una propria riga logica.

Alcuni esempi:

\begin{itemize}
\item
  Blocco di istruzioni separato su più righe con medesima indentazione:

\begin{Shaded}
\begin{Highlighting}[]
\ControlFlowTok{if}\NormalTok{ x }\OperatorTok{\textgreater{}} \DecValTok{0}\NormalTok{:}
  \BuiltInTok{print}\NormalTok{(}\StringTok{"x è positivo"}\NormalTok{)}
\NormalTok{  x }\OperatorTok{+=} \DecValTok{1}
  \BuiltInTok{print}\NormalTok{(}\SpecialStringTok{f"x ora è }\SpecialCharTok{\{}\NormalTok{x}\SpecialCharTok{\}}\SpecialStringTok{"}\NormalTok{)}
\end{Highlighting}
\end{Shaded}
\item
  Blocco come singola istruzione sulla stessa riga logica:

\begin{Shaded}
\begin{Highlighting}[]
\ControlFlowTok{if}\NormalTok{ x }\OperatorTok{\textgreater{}} \DecValTok{0}\NormalTok{: }\BuiltInTok{print}\NormalTok{(}\StringTok{"x è positivo"}\NormalTok{)}
\end{Highlighting}
\end{Shaded}
\item
  Diverse istruzioni semplici sulla stessa riga (non consigliato):

\begin{Shaded}
\begin{Highlighting}[]
\ControlFlowTok{if}\NormalTok{ x }\OperatorTok{\textgreater{}} \DecValTok{0}\NormalTok{: }\BuiltInTok{print}\NormalTok{(}\StringTok{"x è positivo"}\NormalTok{)}\OperatorTok{;}\NormalTok{ x }\OperatorTok{+=} \DecValTok{1}\OperatorTok{;} \BuiltInTok{print}\NormalTok{(}\SpecialStringTok{f"x ora è }\SpecialCharTok{\{}\NormalTok{x}\SpecialCharTok{\}}\SpecialStringTok{"}\NormalTok{)}
\end{Highlighting}
\end{Shaded}
\end{itemize}

\cleardoublepage
\phantomsection
\addcontentsline{toc}{part}{Appendici}
\appendix

\chapter*{Riferimenti}\label{riferimenti}
\addcontentsline{toc}{chapter}{Riferimenti}

\markboth{Riferimenti}{Riferimenti}

\phantomsection\label{refs}
\begin{CSLReferences}{1}{0}
\bibitem[\citeproctext]{ref-kernighan1973b}
Kernighan, Brian W. 1973. {«A Tutorial Introduction to the Programming
Language B»}. Murray Hill, NJ: Bell Laboratories.

\bibitem[\citeproctext]{ref-kernighan1988c}
Kernighan, Brian W., e Dennis M. Ritchie. 1988. \emph{The C Programming
Language}. 2nd ed. Englewood Cliffs, NJ: Prentice Hall.

\bibitem[\citeproctext]{ref-stroustrup1997}
Stroustrup, Bjarne. 1997. \emph{The C++ Programming Language}. 3rd ed.
Reading, MA, USA: Addison-Wesley.

\end{CSLReferences}


\backmatter

\end{document}
